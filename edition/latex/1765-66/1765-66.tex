
               \part*{Année 1765}\addcontentsline{toc}{part}{Année 1765}\chapter*{Janvier 1765}\addcontentsline{toc}{chapter}{Janvier 1765}



                     \begin{diary}{01 Janvier 1765}{}

                         Le verglas d'hier a causé beaucoup d'accidents
                           à Nancy. Il en a été de même
                           à Lunéville. \bigskip



                           L'hôtel de ville de Nancy
                           ayant fait graver par
                           Collin l'estampe de \emph{la construction du quartier
                              royal des casernes}.
                           Elle a été presentée ce matin
                           au roi de Pologne à
                              Lunéville. Sa Majesté en a été contente. \bigskip


                     \end{diary}

                     \begin{diary}{03 Janvier 1765}{}


                           M\up{lle}
                              Gronier a joué pour la première fois à
                           Nancy. On l'a trouvé d'une
                           figure charmante,
                           froide dans \emph{Zaïre}, mieux dans la
                           comédie. \bigskip


                         Jamais nouvel an n'a attiré tant de
                           monde à
                           Lunéville, comédies,
                           bals, grand jeu, grands repas. \bigskip


                         J'ai vu une lettre de M. Baligand à sa mère,
                           il a bien changé de ton et promet d'être un
                           tout autre homme. \bigskip


                     \end{diary}

                     \begin{diary}{07 - 08 Janvier 1765}{}




                           Pluie par huit et 8 \up{1}/\textsubscript{2} degrés de chaud
                        \bigskip


                     \end{diary}
                     \begin{diary}{09 Janvier 1765}{}

                         Résolu dans l'assemblée de
                              l'hôtel de ville de supprimer
                           la flèche du clocher de S.
                              Epvre \&\up{a}. On conservera
                           le cordon de la lanterne sur lequel sera établi un
                           petit dôme pour servir de pied à la croix. \bigskip


                         Assemblée particulière de l'Académie où etaient
                           Mrs Thibault, de Solignac, Du
                              Rouvrois, André
                           Cupers, de
                              Sivry,
                              de Tervenus, P. Husson et Durival
                              l'aîné. On est convenu de lire demain, partie
                           du discours sur la théologie,
                           quelques morceaux
                           de la traduction et du commentaire. Le
                                 discours sur le
                              luxe, qui a remporté l'autre
                           prix ne s'est point trouvé. Le P.
                              Simonin
                           jésuite de
                           Pont à Mousson en est
                           l'auteur. C'est M. Bergier
                           de Darney, curé au diocèse de
                              Besançon qui
                           a traduit \emph{la Théogonie} d'Hésiode.
                        \bigskip


                     \end{diary}
                     \begin{diary}{10 Janvier 1765}{}

                         L'assemblée publique de l'Académie s'est tenue
                           vers les 4 heures après midi. M.
                              Thibault n'y
                           était pas, je n'ai pu non plus m'y trouver. M.
                              le cardinal de Choiseul y était et neuf à
                           dix autres académiciens. On a lu le discours
                           préliminaire et quelques morceaux de traduction
                           et de commentaires de \emph{la Théogonie}. Le discours
                              sur le luxe a été lu aussi ; on l'a trouvé faible. \bigskip


                     \end{diary}

                     \begin{diary}{11 Janvier 1765}{}


                           Madame de Choiseul, abbesse de S. Louis est à
                           Nancy. On joue aujourd'hui
                           pour elle \emph{les
                              Folies amoureuses}. \emph{Annette et
                              Lubin}. \bigskip


                     \end{diary}

                     \begin{diary}{17 Janvier 1765}{}


                           M. de Marcol, procureur général
                           envoie chercher le syndic
                           des imprimeurs de Nancy,
                           touchant \emph{le Dictionnaire
                              philosophique}, et les impressions sans permissions.
                           Les imprimeurs et libraires, s'assemblent en conséquence. \bigskip


                     \end{diary}

                     \begin{diary}{18 Janvier 1765}{}


                           J'ai vu l'ardoisière
                           aujourd'hui. La roue qui
                           fait jouer six pompes. Les routes de communication \&\up{a}. \bigskip


                     \end{diary}

                     \begin{diary}{22 Janvier 1765}{}

                         Mort de J.B. Thimothée de Gautier de
                              Gignéville,
                           fils de M. de
                              Gignéville
                           chevalier de S. Louis. Sera
                           inhumé en l’église de S.
                              Pierre. \bigskip


                         On m'écrivait de Metz
                           le 17 qu'un conseiller
                           du parlement (\sout{Viar} de la Cour) avait été obligé de donner sa démission, pour friponnerie
                           au berlan faite il y a 18
                           mois ; l'avocat dans
                           la maison de qui on avait joué (Deschamps
                              de Villers) a donné acte de ne plus paraître
                           au barreau. \bigskip


                         J'ai parlé à M. de Marcol
                           procureur général sur la
                           librairie. Il a renouvellé les défenses aux
                           imprimeurs et libraires, et d'ouvrir leurs
                           ballots qu'en ma
                           présence. \bigskip


                     \end{diary}

                     \begin{diary}{23 Janvier 1765}{}


                           Assemblée particulière de l'Académie, où
                           étaient
                           M.rs Du Rouvrois directeur,
                              de Sivry
                           sous directeur,
                           de Tervenus, Gautier, André, PP. de Menoux et
                              Leslie,
                           Harmant et Cupers, de
                              Sozzi, de Nicéville,
                              Dhéguerti
                           et Durival l'aîné. \bigskip


                         On y a lu les statuts de l'Académie, et remarqué
                           les changements arrivés depuis leur date. Résolu
                           que quelques-uns de messieurs liraient de leurs ouvrages
                           à chaque assemblée ; qu'on écrirait à M. de Solignac
                           pour avoir les ouvrages qui doivent entrer dans
                           la suite des mémoires ; qu'on demanderait aussi le
                           catalogue général des livres à remettre à l'hôtel
                              de ville, et les noms de ceux qui avaient faits
                           des dons à la bibliothèque, pour imprimer \&\up{a}.
                           Il a été parlé des rétablissements des jetons.
                        \bigskip


                     \end{diary}

                     \begin{diary}{24 Janvier 1765}{}


                           Mon frère Claude m'écrit
                                 \og Le Conseil aulique
                              m'a fait présent d'un brevet de secretaire du
                              cabinet du roi, en considération de mes \emph{longs
                                 services, tant comme secrétaire de l'intendance
                                 qu'en mes qualités actuelles de greffier en chef du Conseil d’État et
                                    d'œconome-séquestre
                                 des
                                 bénéfices à la nomination de Sa Majesté}
                            \fg{}. \bigskip


                     \end{diary}
                     \begin{diary}{27 Janvier 1765}{}


                           M. l'intendant arrive à
                              Nancy à 8h. du
                           soir. Il était parti de Paris
                           le vendredi 25. \bigskip


                     \end{diary}

                     \begin{diary}{28 Janvier 1765}{}


                           Je reçois les trois volumes du recueil
                              (Héré) des bâtiments du roi, pour les
                           déposer aux archives de l'hôtel de
                              ville. \bigskip


                     \end{diary}

                     \begin{diary}{31 Janvier 1765}{}


                           Le roi de Pologne vient à
                              la Malgrange. \bigskip


                     \end{diary}
                  \chapter*{Février 1765}\addcontentsline{toc}{chapter}{Février 1765}



                     \begin{diary}{01 Février 1765}{}

                         J'ai fait ma cour au roi, et vu M.
                              le chancelier sur l'arrêt des fabriques
                              de Nancy. Sa Majesté Polonaise vient à Nancy
                           voir le cardinal de Choiseul, et madame Desarmoises. \bigskip


                     \end{diary}
                     \begin{diary}{02 Février 1765}{}


                           Il a neigé cette nuit, et un peu gelé. \bigskip


                     \end{diary}

                     \begin{diary}{05 Février 1765}{}


                           Le roi est reparti de
                              la Malgrange pour Lunéville
                           immédiatement après son dîner. \bigskip


                     \end{diary}

                     \begin{diary}{06 Février 1765}{}

                         Assemblée particulière de l'Académie, où étaient
                           Messieurs
                           Du Rouvrois, de Sivry, de
                           Tervenus, Gautier,
                           de Sozzi, Beauchamps, PP. Husson, de Menoux,
                              Leslie ;
                           Harmant, Dheguerti, André, Durival
                              l'aîné.
                           On y a lu un ouvrage envoyé par
                           un inconnu
                           sur la formation de la glace, soit dans les eaux mortes
                           soient sur les rivières. Un éloge du
                                 roi de Pologne où
                              le règne de Thelemaque dans Itaque (la Lorraine)
                           par M. l'abbé Millot, qui est
                           proposé pour
                           académicien. On est déterminé à le recevoir,
                           mais pour la règle on attendra la prochaine séance.
                           On a examiné si on était en état de continuer
                           l'impression des mémoires de l'Académie.
                           Non. Mais on fera des extraits des assemblées
                           publiques à envoyer aux autres académies,
                           et aux ouvrages périodiques. \bigskip


                     \end{diary}

                     \begin{diary}{08 - 09 Février 1765}{}

                         On a rempli en deux jours la
                           glacière de la
                              vènerie
                           et les deux de la
                              citadelle.
                           La glace
                           avait environ deux pouces d'épaisseur.
                        \bigskip


                     \end{diary}

                     \begin{diary}{09 Février 1765}{}

                         Je reçois l'arrêt du conseil du 5
                              janvier
                              1765 et les lettres patentes, pour un emprunt
                           de cent mille
                              livres à rentes viagères, et concession
                           du terrain des casernes et autres enfermés
                           dans la clôture du nouveau mur de ville. \bigskip


                     \end{diary}

                     \begin{diary}{18 Février 1765}{}


                           6 \up{1}/\textsubscript{2}
                              degrés de froid, à huit heures du matin.
                           C'est le plus grand de cet hiver. \bigskip


                         Sur l'avis que Conrad Rausch s'était éclipsé,
                           j'ai donné ordre au S.
                              Fleury commissaire de
                           police de constater ce qui existait dans la
                              manufacture de coton, de dire aux ouvriers
                           de continuer, et de remettre le tout à la garde
                           du nommé Laurent, maître drapier. \bigskip


                     \end{diary}

                     \begin{diary}{19 Février 1765}{}

                         Mardi gras.
                           Représentation au profit
                           des pauvres. 5 \up{1}/\textsubscript{2}
                              degrés de froid à 8 h
                              du matin. 0 à une heure après midi.
                           Les productions de la terre sont en sûreté
                           sous quelques pouces de neige. \bigskip


                     \end{diary}

                     \begin{diary}{28 Février 1765}{}

                         Je fais la revue des invalides de
                           la
                           subdélégation de Nancy.
                           M de Narbonne arrive. \bigskip


                     \end{diary}
                  \chapter*{Mars 1765}\addcontentsline{toc}{chapter}{Mars 1765}



                     \begin{diary}{02 Mars 1765}{}

                         Après beaucoup de séances sur
                           l'affaire
                           de Messieurs
                           Duhautoy, Coster, Sozzi \&\up{a}. le conseil
                           a rendu aujourdhui un arrêt
                           interlocutoire,
                           qui charge le S. Coster de
                           prouver que les sommes
                           dont il s'agit ont été versées dans la maison
                              d'Heudicourt. Dès cet après midi on avait
                           répandu à Nancy que M. Duhautoy avait
                           gagné son procès. \bigskip


                     \end{diary}

                     \begin{diary}{04 Mars 1765}{}


                           Mon frère me marque : \og l'abbé de l'Etanche
                              est mort ; on donne l'abbaye au prieur de la maison \fg{}. \bigskip


                     \end{diary}

                     \begin{diary}{05 - 06 Mars 1765}{}


                           Le nommé Calandre, couvreur, vole
                           des
                           plombs sur l’église S.
                              Sebastien. Je le fais
                           mettre en prison. Il avait vendu ce plomb
                           chez
                           la veuve Noël, place
                              S. Epvre. \bigskip


                     \end{diary}

                     \begin{diary}{09 Mars 1765}{}


                           Le conseil se sépare, après avoir
                           travaillé
                           longtemps de suite à Lunéville, au jugement
                           des affaires. \bigskip


                     \end{diary}

                     \begin{diary}{11 Mars 1765}{}


                           M. l'intendant, madame L'intendante, et leurs
                           enfants arrivent à Nancy, où ils
                           resteront
                           jusqu'au mois de mai.
                        \bigskip


                     \end{diary}

                     \begin{diary}{13 Mars 1765}{}


                           M. Alliot me fit part hier soir que
                              le roi
                           voulait donner cent mille livres de France à la ville
                              de Nancy, pour des objets ; de bien public. Et
                           aujourdhui je lui ai donné une note pour détourner
                              Sa Majesté Polonaise d'ordonner que cette somme soit
                           prêtée sous caution à 2 % de rente à
                           des bourgeois pauvres ; et donné un mémoire
                           pour employer les 5000\up{\#.} de rente savoir :
                           2000\up{\#.} à entretenir les bâtiments construits
                           par le roi. 500\up{\#.}
                           pour des fous, imbéciles, \&\up{a}.
                           1000\up{\#.} sur les accouchements. 600\up{\#.} pour
                           les accouchements doubles. 500\up{\#.} pour 5
                           renfermés à la
                           vènerie. 400\up{\#.} pour le jardin
                              de botanique. \bigskip


                     \end{diary}

                     \begin{diary}{14 Mars 1765}{}

                         Mort de J.B. Taillandier, sera inhumé à S. Sebastien.
                           Il avait été commis du trésorier de l'hôtel du duc Léopold. \bigskip



                           10 degrés de chaud. \bigskip


                     \end{diary}

                     \begin{diary}{18 Mars 1765}{}

                         On apprend que M. de Chateaufort
                           conseiller
                           à la Cour souveraine est mort à Paris, le 15 après
                           une maladie de 12 jours. Il laisse sa veuve
                           et 8 enfants la plupart estropiés, au moment
                           où il allait jouir d'appointements considérables
                           accordés par l'empereur, pour l'affaire de la
                           liquidation des dettes d’État. C'est M.
                              d'Ubexi
                           qui aura le cornet à sa place. \bigskip



                           Le curé de Commercy (M.
                           est mort aussi. \bigskip


                     \end{diary}


                     \begin{diary}{20 Mars 1765}{}

                         Assemblée particulière de l'Académie, où
                           étaient Messieurs
                           Thibault, Dheguerti, de
                              Tervenus,
                           les PP. de Menoux et Leslie, et Durival l'aîné. J'y ai lu un
                           memoire assez long qui m'a
                           été demandé par M.
                              l'intendant, pour servir
                           à la nouvelle
                           édition de la bibliothèque
                           historique du P. Le Long ;
                           et je l'ai envoyé ensuite à M.
                              l'intendant. \bigskip



                     \end{diary}

                     \begin{diary}{Encart}{} Lettre de M. de
                                 Belloy,
                              auteur de la tragédie
                              du \emph{Siège de Calais}, à
                              M. de Belloy, colonel au
                              corpus royal d'artillerie,
                              datée de Paris, le 2 avril 1765. \bigskip

                 Une conformité de noms a
                              causé, monsieur, une méprise
                              qui m'honore, puisqu'elle me
                              procure l'avantage de recevoir
                              une marque de l'aprobation
                              que vous voulés bien donner au
                              \emph{Siége de Calais}. Je suis l'auteur
                              de cette tragédie, et je n'ai pas
                              la gloire d'etre de votre famille
                              ni meme de votre province. Je
                              ferois bien plus de cas de l'honneur
                              de vous apartenir que du succès
                              passager d'un ouvrage qui ne doit sa fortune qu'aux
                              bontés
                              excessives du public. J'ai peint
                              l'amour de la patrie. Vos ancêtres
                              l'ont réalisé par les services
                              qu'ils ont rendus à leur souverain
                              et je sens qu'il est un peuple
                              difficile d'être au rang des
                              héros que de les faire parler
                              sur le théâtre. Personne ne
                              respecte plus que moi la vraie
                              noblesse ; celle que l'on porte dans
                              son sang, et que l'on soutient par
                              ses vertus. C'est la vôtre, M.
                              c'est celle que j'aurais ambition
                              et que le sort n'a pas faite pour
                              moi. \bigskip


                                 J'ai l'honneur d'être \&\up{a}.\bigskip

                \begin{flushright}
                                 signé Buirette de
                                    Belloy
                              \end{flushright}\end{diary}

                     \begin{diary}{22 Mars 1765}{}


                           On m'écrit de Lunéville que
                           le titre d'évêque,
                           in partibus est accordé à M.
                              l'abbé de Mareil,
                           grand prevost de S. Diez. \bigskip


                     \end{diary}

                     \begin{diary}{23 Mars 1765}{}


                           Le roi de Pologne arrive à la Malgrange.
                           On reçoit des
                           imprimés de la tragédie le \emph{Siège de
                                 Calais} de
                           M. de
                              Belloy. \bigskip


                     \end{diary}

                     \begin{diary}{24 Mars 1765}{}

                         Je vois le roi à la
                              Malgrange. Il est toujours
                           dans l'intention de remettre un fonds à l'hôtel
                              de ville, pour employer la rente à soulager ceux
                           qui se trouveraient dans le besoin. \bigskip


                         Conseil à la Malgrange. \bigskip


                     \end{diary}

                     \begin{diary}{25 Mars 1765}{}

                         Conseil encore l'après midi
                           Fleury est parti ce matin pour
                              Turin,
                           d'où à Parme. madame Sanlaville pour
                           Vienne. Le roi de Pologne est venu à Nancy
                           voir madame la marquise Dèsarmoises. \bigskip


                     \end{diary}

                     \begin{diary}{26 Mars 1765}{}


                           Le roi part a 4 h. après midi
                           pour Lunéville.
                           Le S. Le Kain arrivé de la veille
                           à onze heures
                           du soir joue le rôle de Zamore dans \emph{Alzire}
                        \bigskip


                     \end{diary}

                     \begin{diary}{27 Mars 1765}{}

                         Le lendemain Rhadamiste, la D.\up{lle} Grossier Zenobie. \bigskip


                     \end{diary}

                     \begin{diary}{28 Mars 1765}{}

                         Le \emph{Duc de
                              Foix}. \bigskip


                     \end{diary}

                     \begin{diary}{29 Mars 1765}{}


                           \emph{Iphigenie en Tauride}. Il y avait beaucoup
                           de monde quoique le prix des loges et de
                           l'amphithéâtre ait été augmenté. \bigskip


                     \end{diary}

                     \begin{diary}{30 Mars 1765}{}


                           \emph{Mitridate}. On a eu beaucoup de peine a
                           engager le S. Rocourt
                           (Sauxerotte) à jouer ce
                           rôle, parce que des arrangements étaient faits
                           pour partir le matin pour Bordeaux. \bigskip



                           M. Le Kain a joué Xipharès avec
                           beaucoup
                           de succès. Il y avait beaucoup de loges vides,
                           ce qui devait être une veille des rameaux.
                           Il est venu beaucoup de bourgeois de Lunéville
                           pour voir jouer M. Le Kain.
                              Madame de
                              Bouflers et quelques autres dames de la
                           Cour ont préféré de rester à la
                              Malgrange. \bigskip


                     \end{diary}

                     \begin{diary}{31 Mars 1765}{}

                         On parle de troubles en Bretagne. \bigskip


                     \end{diary}
                  \chapter*{Avril 1765}\addcontentsline{toc}{chapter}{Avril 1765}



                     \begin{diary}{01 Avril 1765}{}


                           M. l'intendant accorde à
                              Nicetti le remboursement
                           sur la ville des 500\up{\#.} de débit de M.\up{lle} Gronier,
                           et que les 354\up{\#.} qui revenaient à Brisson soient
                           employés au payement des frais de la comédie,
                           attendu l'insuffisance des recettes depuis la
                           séparation des deux troupes. \bigskip


                     \end{diary}
                     \begin{diary}{05 Avril 1765}{}

                         Nous nous sommes assemblés à l'hôtel de
                              ville aujourd'hui Vendredi saint. On n'a pas
                           jugé à propos de toucher à la taxe de la
                           viande. Nous avons signé 3
                           délibérations :
                           une pour la grande
                              pépinière à former ;
                           une pour disposer d'une partie des terrains
                           pour le jardin de
                              botanique, pour la
                              sécherie
                              des linges près des lavoirs publics ; la
                           3.\up{e} Pour accorder aux commissaires de police 2.\up{ds}
                           par livre du 20.\up{e} d'abonnement. \bigskip


                     \end{diary}

                     \begin{diary}{07 Avril 1765}{}


                           M. l'intendant arrive de
                              Neuviller à 6h.
                           du soir. \bigskip


                     \end{diary}

                     \begin{diary}{09 Avril 1765}{}

                         Aujourdhui à 6 h. \up{1}/\textsubscript{2} du matin M. l'intendant
                           m'a fait avertir de partir à 7 \up{1}/\textsubscript{2} avec lui
                           pour Lunéville, où le roi nous mandait, pour
                           consommer un dernier bienfait pour la ville
                           de Nancy. L'après midi dans le
                           cabinet
                           du roi, où étaient M. le
                              chancelier, M.
                              l'intendant et moi ; le roi fait relire son projet, explique ses
                           intentions ; tout est
                           discuté, M. le chancelier
                           chargé de former
                           les lettres patentes. M. l'intendant retourne
                           à Nancy. Le roi m'ordonne de rester à Lunéville. \bigskip


                     \end{diary}

                     \begin{diary}{11 Avril 1765}{}

                         Le lendemain M. le chancelier fait la minute
                           des lettres patentes. \bigskip


                     \end{diary}

                     \begin{diary}{12 Avril 1765}{}

                         Aujourdhui elles ont été mises sur le
                           parchemin, approuvées et applaudies
                           par le roi, scellées
                           l'après midi. Sa Majesté
                           fait don de cent milles livres de France
                           à la ville, (pour employer par les officiers
                           municipaux avec l'approbation de M.
                              l'intendant, la vente de 5000\up{\#.} à soulager
                           les habitants de Nancy, nobles, rentiers
                           artisans, journalier, de l'un et l'autre
                           , dans leurs besoins causés par des
                           malheurs imprévus \&. M.
                              Alliot me
                           donne le mandement pour toucher par le
                              trésorier de la ville
                           les cent mille
                              livres
                           sur le receveur général des fermes. \bigskip


                         Je visite l'après midi la manufacture
                              nouvelle établie à Lunéville au faubourg
                              de Viller.
                        \bigskip


                         En arrivant à Nancy je rends à M.
                              Richer
                           le mandement des cent milles livres. \bigskip


                     \end{diary}


                     \begin{diary}{13 Avril 1765}{}

                         Le samedi assemblée de l'hôtel de ville. Nous
                           faisons, la distribution de cette somme. M.
                              Mique y est pour 57000\up{\#} de Lorraine
                           qui complètent les 120000\up{\#.} de France
                           dont la ville contribue aux casernes. \bigskip


                         Le soir fort tard M. et madame de Stainville
                           arrivent de Metz et logent à l'hôtel du
                              commandant, où le nouvel ameublement
                           était préparé. \bigskip


                     \end{diary}

                     \begin{diary}{14 Avril 1765}{}

                         À M. le
                              chancelier. \bigskip


                        \begin{quote}\begin{flushright}
                                 Nancy ce 14 avril 1765. \end{flushright}\bigskip


                                 Mgr \bigskip


                              Je ne pourrai vous exprimer que très imparfaitement
                              les sentiments d'admiration, d'\sout{atendrisse} attendrissement et
                              d'amour dont tous les cœurs ont été pénétrés
                              en apprenant les dernières grâces que le roi
                              a \sout{bien voulu}
                              daigné répandre sur la ville de Nancy.
                              Le corps municipal infiniment honoré des
                              marques si distinguées de la confiance
                              que
                              Sa Majesté veut bien avoir en lui imaginait
                              toutes sortes de choses pour en témoigner sa
                              vive reconnaissance. Il a été besoin d'arrêt
                              des mouvements si naturels cependant et si justes, pour
                              se rapprocher des intentions du 1\up{er} prince dont
                              la bienfaisance et les bontés sont sans bornes.
                              La mémoire en passera d'âge en âge et
                              ne
                              s'effacera jamais. Je me suis chargé, Mgr,
                              de vous suplier très humblement de nous mettre aux pieds de Sa Majesté avec tous les
                              habitants de Nancy, qui ne
                              cesseront
                              \sout{jamais} de la bénir, et de faire
                              les vœux
                              les plus ardents pour que le ciel daigne
                              \sout{nous le} conserver le roi qu'il nous a
                              donné pour notre bonheur et pour la
                              gloire
                              de l'humanité. \bigskip

         Je suis avec le plus profond respect
                                 Mgr \begin{flushright}
                                 V.
                              \end{flushright}\end{quote}
                     \end{diary}

                     \begin{diary}{16 Avril 1765}{}

                         Je pars de Nancy
                           le mardi 16 pour S\up{t}.
                              Aubun, où je trouve ma
                              mère fort malade. \bigskip


                     \end{diary}

                     \begin{diary}{19 Avril 1765}{}

                         De retour à Nancy
                           le 19. Il y avait outre
                           M. de Stainville, M. le prince d'Anhalt et M.
                              le marquis d'Armentières. Ils avaient mangé
                           plusieurs fois chez
                           M. l'intendant. \bigskip


                     \end{diary}

                     \begin{diary}{20 Avril 1765}{}

                         Grand concert chez
                           madame de Stainville. \bigskip


                     \end{diary}

                     \begin{diary}{21 Avril 1765}{}

                         Concert à la comédie au profit des S.\up{rs}
                            musiciens de M. le prince. \bigskip


                     \end{diary}

                     \begin{diary}{24 Avril 1765}{}

                         Redoute. L'hôtel de ville a signé le contrat
                           de vente de la maison derrière celle de la
                              cure de S. Sebastien, à 9000\up{\#} au S. Brouck
                           pour y etablir une manufacture. \bigskip


                     \end{diary}


                     \begin{diary}{24 Avril 1765}{}

                         On commence a maçonner à l'aile
                           gauche
                           des casernes. \bigskip


                     \end{diary}

                     \begin{diary}{25 Avril 1765}{}


                           Pluie douce. Second concert à la
                              comédie
                           par deux excellents musiciens, les S.\up{rs}
                              Provére et
                        \bigskip


                     \end{diary}

                     \begin{diary}{26 Avril 1765}{}


                           M. de Mareil
                           grand doyen de S. Diey remit
                           les complimens sur le titre d'\emph{évêque} de
                           Sion
                           \emph{in partibus}
                        \bigskip


                         La seconde édition de l'\emph{Essai sur le luxe} de
                           M. de S.
                              Lambert,
                           avec des changements de
                           l'auteur paraît chez
                           Le Clerc à Nancy. \bigskip


                     \end{diary}

                     \begin{diary}{28 Avril 1765}{}


                           M. l'intendant donne une fête
                           brillante
                           à madame la comtesse de Stainville, grand souper,
                           grand bal ; le salon de l'intendance était
                           éclairé de cinq lustres et de girandoles. \bigskip


                     \end{diary}

                     \begin{diary}{30 Avril 1765}{}


                           Le roi de Pologne arrive à la
                              Malgrange aujourd'hui
                           mardi vers 5 heures du soir. \bigskip


                     \end{diary}
                  \chapter*{Mai 1765}\addcontentsline{toc}{chapter}{Mai 1765}



                     \begin{diary}{01 Mai 1765}{}

                         J'ai vu aujourd'hui Sa Majesté Polonaise
                           la comtesse
                              de Stainville \&\up{a}. a dîné à la Malgrange. \bigskip


                     \end{diary}

                     \begin{diary}{02 Mai 1765}{}


                           Le roi de Pologne vient à Nancy à
                           onze heures
                           du matin et dîne à midi chez
                           M. le comte
                              de Stainville, où il y avait une table de
                           24 couverts la garde avait pour la première fois, son habillement neuf.. On avait posé une face
                           de la décoration du feu d'artifice qui sera
                           tiré autour de la statue de Louis XV le
                           jour de S.\up{t} Stanislas. \bigskip


                         Assemblée particulière de l'Académie, où
                           étaient Messieurs
                           Cu Rouvrois, de Sivry, de
                              Solignac, P.
                           Leslie, abbé de Tervenus,
                              André
                           Thibault, Bagard, Durival
                              l'aîné, Durival
                              cadet, de
                              Nicéville. \bigskip



                           M. de Solignac a lu ce qui a été
                           écrit
                           de la séance du 17 avril, où on
                           avait
                           refusé d'admettre le S. Uriot,
                           parce qu'il
                           avait été comédien. M. Du
                              Rouvrois le
                           discours qu'il prononcera comme directeur
                           le jour de S. Stanislas. Le P.
                              Leslie a lu
                           le discours de réception de M. l'abbé Millot
                           un académicien de Lyon,
                           couronné par plusieurs
                           academies. M. de Solignac
                           l'éloge de feu
                           M. Henry
                           procureur du roi à Toul,
                           académicien
                           on a déterminé que la séance publique
                           commencerait par la lecture de cet ouvrage qu'ensuite M.
                              Bagard lirait ce qu'il a fait
                           sur les eaux de Luxeuil. Puis le discours
                              de M. l'abbé
                                 Millot, après M. Du
                              Rouvrois
                           prononcerait le sien. On
                           avait renouvelé
                           la prétention que les académiciens qui sont
                           de Nancy viendraient à
                              l'Académie et
                           travailleraient, ou seraient rayés du tableau
                           mais on s'en est tenu à ce qui avait été
                           réglé auparavant, en faveur de la liberté.
                           M. Le Bas, qui a écrit en faveur
                           des
                           naissances tardives a été proposé pour
                           académicien associé, sur quoi j'ai représenté
                           l'abus d'admettre à l'Académie
                           tous ceux qui
                           feraient imprimer \&\up{a}. \bigskip


                     \end{diary}

                     \begin{diary}{05 Mai 1765}{}


                           Le roi de Pologne vient vers 3 h \up{1}/\textsubscript{2} après midi
                           voir les casernes. Il y avait dans sa
                           voiture
                           M. le chancelier, M. le comte de Stainville
                           et M. de Sommievre. Après
                           avoir vu le plus
                           près possible les bâtiments il est entré au
                           jardin de botanique. On a
                           beaucoup crié
                           \og \emph{vive le roi} \fg{}
                        \bigskip


                         Hier mourrut
                           Mestivier, fameux
                           dans le contrôle des actes \bigskip


                     \end{diary}
                     \begin{diary}{06 Mai 1765}{}


                           Le roi est parti aujourd'hui de
                              la Malgrange
                           après son dîner. \bigskip


                     \end{diary}

                     \begin{diary}{07 Mai 1765}{}


                           Nanon, ma concierge d'Heillecourt, qui
                           va se marier, congédiée. Je lui ai payé
                           les 4 mois \up{1}/\textsubscript{4} de gages de
                           cette année. \bigskip


                     \end{diary}
                     \begin{diary}{08 Mai 1765}{}


                           L'Académie s'est assemblée à dix
                           heures du
                           matin chez
                           M. Du Rouvrois
                           premier président. De là
                           on est allé à la messe aux Cordeliers. Le P.
                              Gautier
                           jésuite a prononcé le panégyrique de
                           S. Stanislas. On est sorti à midi, et M.
                              le premier a donné un grand dîner. À
                           trois après midi l'assemblée publique dans
                           le grand salon de l'hôtel de
                              ville, où étaient
                           Messieurs
                           Du Rouvrois, directeur, de Sivry, sous directeur
                           de Solignac
                           secrétaire
                           André, Tervenus, P. Leslie
                           P. Husson, Bagard, Harmant, Cupers,
                           de Beauchamp, de Sozzi, Plaid,
                           Durival l'aîné. \bigskip



                           M. de Solignac a ouvert la séance
                           par
                           l'éloge de M. Henry
                           procureur du roi à Toul,
                           académicien mort au mois de janvier dernier
                           et qui avait été reçu il y a un an le 8.
                              mai.
                           Ensuite le S. Joseph Coster dont
                           la réception
                           avait été différée a enfin été reçu. Il a
                           traité dans son discours des
                           qualités qui font
                           de bon citoyen. M. de Sozzi a
                           lu le discours
                              de reception de M. l'abbé
                                 Millot de l'académie
                              de Lyon, il y est traité, lequel est plus difficile
                           d'éclairer les hommes que de les gouverner.
                           La séance a été terminée par un fort
                           bon discours de M. le
                              directeur, où il a fait entrer fort
                           adroitement l'éloge du roy de Pologne
                           et parlé de sa derniere fondation. Il n'a
                           pu par faiblesse de poitrine lire que le commencement.
                           La lecture a été faite par M. de
                              Sivry qui
                           était à côté de lui. Il y avait beaucoup de
                           monde, et en général la séance a bien été. \bigskip


                         L'usage était de faire un feu de joie chaque
                           année le jour de S. Stanislas sur la
                              place du
                              marché de la ville neuve. Dès l'année dernière
                           l'hôtel de ville prit la résolution
                           de supprimer
                           ce feu de joie qui inquiétait les propriétaires
                           de maisons ; et de faire éxécuter en place
                           un feu d'artifice sur la place royale.
                           Ce qui a eu lieu pour la première fois cette
                           année. Tout était préparé par une
                           décoration de boiserie peinte par Claudon
                           et qui ornait en dedans de la grille les
                           4 faces du piédestal de la statue de Louis XV.
                           Des transparents en bleu clair laissaient
                           voir à jour les chiffres du roi
                              de Pologne
                           et ces mots : \emph{vive Stanislas le
                              bienfaisant}.
                           Toutes les croisées de la place étaient
                           remplies du plus beau monde, et la place
                           de tout le peuple. À nuit fermée vers
                           neuf heures du soir, on fit faire un grand
                           cercle à environ 25 pas de distance de la grille, et le S. François, conseiller en l'hôtel
                              de ville, qui a tout conduit, fit partir les
                           artifices des 4 faces successivement, ce qui
                           dura un peu plus de demi-heure et fit grand plaisir, la nuit fut assez noire,
                           et l'air très calme et sans
                           pluie. Le
                           peuple a crié \emph{vive le roy} à
                           differentes
                           reprises et de bon cœur. \bigskip



                           M. et madame de Stainville étaient partis
                           de la ville pour Lunéville,
                           où ils
                           reviendront seulement demain. \bigskip


                     \end{diary}
                     \begin{diary}{11 Mai 1765}{}


                           Le chevalier Taylor, anglais, fameux
                           oculiste,
                           a prononcé à 8 h. du soir dans la salle des
                           redoutes, un discours sur son art, et à fait
                           des démonstrations sur l'œil. \bigskip


                     \end{diary}
                     \begin{diary}{12 Mai 1765}{}

                         Revue à 8 h. du matin à
                              la garenne, des
                           4 brigades du corps des grenadiers de France,
                           tous avec l'habillement neuf. \bigskip


                     \end{diary}
                     \begin{diary}{13 Mai 1765}{}

                         Un gros chien enragé, sans en être, en
                           a mordu
                           plusieurs dans les rues de
                              Nancy ce matin,
                           et un enfant de deux ans dans une allée ;
                           le peuple l'a poursuivi, un grenadier lui a
                           donné un coup de sabre sur le dos \bigskip


                         Ordonnance de police de tenir pendant
                           huit
                           jours les chiens enfermés, de livrer ceux qui
                           ont été mordus ou qui seraient soupçonnés au
                           maître des basses
                              œuvres qui a ordre de
                           faire
                           ses tournées après l'avertissement. \bigskip


                     \end{diary}
                     \begin{diary}{14 Mai 1765}{}

                         On m'écrit d'hier de Lunéville : \og M. Bresson (subdélégué
                              à Darney) a des lettres de
                              reconnaissance de noblesse \fg{}. \bigskip


                     \end{diary}


                     \begin{diary}{17 Mai 1765}{}


                           20 degrés de chaud
                           après midi.
                        \bigskip


                     \end{diary}
                     \begin{diary}{20 Mai 1765}{}


                           Le maréchal de Contade passe l'après midi allant
                           à Strasbourg. M. de Stainville avait fait laisser
                           la garde sous les armes jusques passé une heure.
                           Il comptoit donner à dîner au maréchal mais
                           il ne s'est point arrêté et ils ne se sont point vus.
                           Le soir arrivent M. l'archevêque de
                              Cambray,
                           madame l'abbesse de S. Louis, Messieurs
                           d'Armentières, prince
                              d'Anhalt, de
                           Chabot. \bigskip


                     \end{diary}
                     \begin{diary}{21 Mai 1765}{}

                         À dix heures du matin les 4
                           brigades du
                           corps des grenadiers de
                              France étaient sur le
                           pré, dans une même ligne. On a exercé
                           au feu, le prince duc des Deux Ponts, est arrivé
                           de Lunéville allant à
                              Paris et a été présent
                           jusqu'à la fin, c'est lui qu'on a salué en
                           défilant. On a d'abord tiré par compagnie
                           par deux, par trois compagnies, par
                           brigade ; ensuite le feu de billaude. Tout
                           était fini à onze heures et demie. Le temps
                           était clair et le vent un peu à l'Est. \bigskip


                     \end{diary}
                     \begin{diary}{22 Mai 1765}{}

                         Assemblée particulière de l'Académie, où étoient
                           Messieurs
                           Du Rouvrois, de Sivry, de
                           Solignac, de
                              Tervenus, André,
                              Durival l'aîné, et pour
                           la première fois M. Coster.
                              M. Du Rouvrois
                           directeur a lu une lettre écrite à l'Académie
                           par M. de Laverdy, contrôleur général pour inviter l'Académie à concourir au succès du \emph{Journal
                              de commerce} (ou gazette.) et de lui faire part
                           directement de toutes les lumières qui lui
                           viendraient là dessus ; sur quoi M.
                              Coster s'est
                           engagé de lire chaque quinzaine à l'Académie
                           quelque chose sur cet objet. Le reste de la
                           seance a été employé à différents
                           raisonnements
                           sur l'agriculture \&\up{a}. \bigskip


                     \end{diary}
                     \begin{diary}{24 Mai 1765}{}

                         Grande rumeur parmi les grenadiers, à
                           l'occasion de 3\up{\#.} 10\up{s.} qu'on
                           voulait leur retenir sur
                           leur décompte ; menaces, lettre insolente écrite
                           à M. de Stainville qu'ils ont
                           plus d'amis que lui
                           à la Cour de France. Ils avaient mis à quelques
                           casernes, \emph{maison à louer}, et le coup de la
                           retraite était pour eux le signal de la générale.
                           La retenue n'a pas eu lieu. \bigskip


                     \end{diary}
                     \begin{diary}{25 Mai 1765}{}

                         J'ai dîné chez
                           M. l'éveque de Toul à sa maison
                           de Nancy. \bigskip


                         Il a plu la nuit \sout{dernière} du 23 au 24 après une assez longue
                           sécheresse, qui faisait souffrir les prés. \bigskip


                     \end{diary}
                     \begin{diary}{26 Mai 1765}{}

                         Aujourd'hui, fête de la Pentecôte,
                              M. l'abbé
                              de Mareil, a été sacré évêque de Sion,
                           par M. l'archevêque de
                              Besançon, M. l'évêque
                              de Metz, et le
                              suffragant de Besançon,
                           dans
                           la chapelle du roi à Lunéville, avec
                           beaucoup d'éclat et un grand concours
                           de spectateurs, qui s'y étaient rendus de
                           toutes parts et de tous états. \bigskip


                     \end{diary}


                     \begin{diary}{27 Mai 1765}{}


                           Le cardinal de Choiseul vient à
                              Nancy. \bigskip


                     \end{diary}
                     \begin{diary}{28 Mai 1765}{}


                           M. de Sion et M. l'intendant son neveu
                           y arrivent l'après midi. \bigskip


                     \end{diary}
                     \begin{diary}{29 Mai 1765}{}


                           M. de Sion part pour Metz. M. Le Pelletier
                              de Beauprés passe et voit M. l'intendant.
                           Peu après passe aussi M. de Blair
                           nouvel intendant de Strasbourg. \bigskip


                         Assemblée particulière de l'Académie pour
                           examiner un ouvrage imprimé chez
                           Le Seure,
                           sous le titre de \emph{\emph{Recueil sur
                                 diverses matières}}.
                           M. de Solignac l'avait
                           adressé à M. Du Rouvrois
                           comme l'ouvrage d'un jeune homme qui avait
                           des talents et le désir d'être de l'Académie. On
                           n'a pas été longtemps sans reconnaître dans ce
                           prétendu jeune homme un roi
                              de 88 ans.
                           Il y avait à l'assemblée Messieurs
                           Du Rouvrois,
                           de Sivry, Thibault, évêque de Toul, de
                              Niceville
                           Bagard, Harmant, Cupers,
                              Durival l'aîné,
                           Andre, PP. Leslie et Husson,
                              Coster
                        \bigskip


                         J'ai été obligé de quitter aprés la
                           lecture de
                           quelques chapitres pour aller à l'étuve des
                              blés. J'y
                           ai verifié que le magasin de
                           conservation a 6 pi. 1 po.
                           de hauteur, sur
                           sept pieds de large dans un
                           sens, et 5 pi. 6 po.
                           dans l'autre. \bigskip


                         3 resaux de blés tirés du magasin
                              de la
                              poissonnerie pesaient chacun 180\up{l}. déduction, faite du poids des sacs. Les 3 mesurés,
                           dans un bichet d'un pied cube ont donné
                           10 pi \up{1}/\textsubscript{2} en sorte que le
                           resal de Nancy a
                           trois pieds et demi cubes. \bigskip


                         Hier mourut le S. Mengin Arnould,
                           chancelier, capitaine de la milice
                              bourgeoise.
                           M. de Stainville n'a pas
                           voulu permettre
                           qu'on lui rendit les honneurs militaires suivant
                           l'usage. \bigskip


                         Hier matin il deserta deux
                           grenadiers ;
                           cette nuit autant. Le mécontentement subsiste,
                           toute la maréchaussée et plusieurs détachements
                           sont aux environs et le tout en vain. \bigskip


                     \end{diary}
                     \begin{diary}{30 Mai 1765}{}


                           M. l'intendant part pour
                              Neuviller, après
                           avoir vu mettre le blé à l'étuve, et visité
                           les casernes. On pose
                           les bois de toiture
                           d'un pavillon. On maçonne à la fois aux
                           deux aîles, et on commence à éléver les
                           corniches du corps du milieu. \bigskip


                         On a poussé la chaleur jusqu'à 40
                           degrés
                           à l'étuve de Duhamel.
                        \bigskip


                     \end{diary}
                  \chapter*{Juin 1765}\addcontentsline{toc}{chapter}{Juin 1765}




                     \begin{diary}{01 Juin 1765}{}

                         Aujourd'hui matin on a passé par les
                           courroyes deux
                              grenadiers. L'un pour avoir
                           en lâche donné des coups de sabre au visage
                           de son camarade. L'autre pour vol de chambrée.
                           Ils auront des congés infamants. \bigskip


                         La grande affaire des chanoines
                              réguliers
                           contre l'évêque et les curés a été jugée à
                           la Cour souveraine. Les chanoines réguliers ont perdu. \bigskip


                         Les charpentiers ont mis après midi la
                           première cheville aux
                              casernes, avec beaucoup
                           de cérémonie, des violons, des boites \&\up{a} \bigskip


                         Un convoi considérable de vieux
                           canons de
                           Metz est parti pour Lunéville. \bigskip


                     \end{diary}

                     \begin{diary}{04 Juin 1765}{}

                         On m'écrit de Lunéville
                           le 4 à 10 h. du soir :
                           Il s'agit de savoir qui du parlement ou de M.
                              le comte de Stainville aura le pas, à la procession
                           de la Fête-Dieu. Le roi de Pologne ne veut pas
                           prononcer et renvoie à faire juger la difficulté
                           par le roi son gendre
                        \bigskip


                     \end{diary}

                     \begin{diary}{05 Juin 1765}{}

                         Le lendemain, après la séparation de
                           l'assemblée
                           de l'hôtel de ville, vers onze
                           heures du matin, un
                           huissier en robe, est venu me dire de la part de
                           la Cour souveraine qu'elle
                           n'irait point à la
                           procession. Sur quoi j'ai convoqué extraordinairement
                           l'hôtel de ville à 3 h. après midi. La résolution a été prise unanimêment de ne point non plus
                           aller à la procession. Messieurs de l'hôtel
                              de ville
                           en corps sont venus m'en faire part aussitôt
                           dans ma chambre où j'étais retenu par la goutte.
                           Pendant qu'ils y étaient encore M.
                              de Stainville
                           m'a envoyé dire d'aller lui parler. Je m'en suis
                           excusé sur mon indisposition. Un moment après
                           M. Dorly son secrétaire est
                           venu et m'a pris
                           en particulier, pour me dire que M. de
                              Stainville
                           était informé que le magistrat avait fait dire
                           à Messieurs de la primatiale que le corps de ville
                           n'irait point à la procession. Qu'il me priait
                           d'engager ces messieurs à y aller, de ne point
                           mettre d'humeur dans cette affaire, que M. de
                              Stainville était commandant général de la
                           province
                           par des provisions enregistrées \&\up{a}. J'ai répondu
                           que le corps de ville et
                           les autres suivaient la
                           Cour, et que nous ayant avertis qu'elle n'irait
                           pas à la procession c'était une deffense d'y aller ;
                           que nous n'avions point d'humeur ; que M. le
                              comte de Stainville était commandant
                           général des
                           troupes, mais non de la justice, de la police,
                           des villes. M. Dorly s'est retiré,
                           et j'ai fait
                           part de la conversation à Messieurs de l'hôtel de ville
                           qui se sont aussitôt séparés. \bigskip



                           M. le cardinal de Choiseul a fait
                           placer dans le
                           chœur un fauteuil pour M. de
                              Stainville et des
                           chaises pour les officiers, mais sans la participation
                           du chapitre. \bigskip





                           M. de Stainville avait écrit
                           deux lettres à M.
                              le premier président de la cour, pour avoir par écrit
                           le refus d'aller à la procession ; mais M.
                              Du Rouvrois
                           lui a fait répondre verbalement. Au reste la
                              Cour souveraine a fait un arrêté et veut porter ses
                           plaintes au roi de France comme Sa Majesté Polonaise l'a permis. \bigskip


                     \end{diary}

                     \begin{diary}{06 Juin 1765}{}

                         Dès les 6 heures le corps des
                           grenadiers de
                           France faisait ses dispositions pour border la
                           haye dans toutes les vues ou passerait la
                           procession. Les paroisses, confréries et moines
                           se sont rendus à la
                              primatiale. La procession
                           n'a commencé à en sortir qu'à 9 heures un
                           quart, passant par la rue
                              de la congrégation,
                           devant l'hôtel de ville,
                              la rue de la
                              poissonnerie
                           devant les minimes
                              stations, aux
                              petites Carmelites
                              station rue S. Joseph, l'hôpital S. Charles station
                           S. Roch station d'où à la
                              primatiale vers 10 h \up{1}/\textsubscript{2}.
                           Aucun corps de judicature
                           ne s'y est trouvé,
                           et les marchands ont refusé de porter le dais.
                           M. le comte de Stainville
                           suivait immédiatement
                           le S. Sacrement porté par le cardinal, il y
                           avait
                           aux côtés et derrière M. de
                              Stainville un gros
                           d'officiers, soit de son corps soit de l’état-major
                           de la place. Il avait fait inviter la noblesse,
                           mais personne de ce corps ne s'y est rendu.
                           Le peuple a été choqué et affligé de ne voir
                           point la Cour et les autres juridictions, et la procession si écourtée et si
                           differente des
                           autres fois. \bigskip


                     \end{diary}
                     \begin{diary}{10 Juin 1765}{}

                         Mort de Marianne Legal de Vissemberg,
                           fille de M. de Vissemberg
                           chevalier de S. Louis.
                           On inhume aussi
                           dit la Jeunesse, concierge et
                           ensuite portier
                           à l'intendance, fils d'un homme
                           qui l'avait
                           été dans l'ancien château de Nancy. \bigskip


                     \end{diary}
                     \begin{diary}{11 Juin 1765}{}


                           M. Stadler, aide de camp, écuyer,
                           de M. le
                              comte de Stainville, épouse M.\up{lle} Billecard,
                           fille du notaire. \bigskip


                     \end{diary}

                     \begin{diary}{12 Juin 1765}{}

                         Assemblée particulière de l'Académie où
                           étaient Messieurs
                           Du Rouvrois directeur, de Sivry
                           sous directeur, de Tervenus, Bagard, Cupers,
                           P. Husson, André, abbé Foliot,
                              Coster
                           Durival l'ainé. M. Du Rouvrois à lu la lettre
                           qu'il avait écrite à M. de
                              Solignac, à l'occasion
                           de l'ouvrage du prétendu jeune homme (le roi de Pologne)
                           cette lettre tendait à faire rendre à l'Académie
                           les 2000\up{\#.} qu'on lui a retranchés. \bigskip



                           M. Coster a lu ensuite un mémoire excellent
                           où il commence par réfuter différents auteurs
                           qui ont parlé de la Lorraine et fait l'éloge
                           de mon mémoire sur les duchés de Lorraine et de
                                 Bar. Il parle ensuite de l'agriculture, du
                           commerce, de la population. \bigskip


                     \end{diary}



                     \begin{diary}{13 Juin 1765}{}

                         La procession de la petite Fête
                           Dieu s'est faite
                           à l'ordinaire. M. de
                              Stainville a accordé un
                           détachement pour celle de la
                              primatiale. Mais
                           il en a refusé pour celle de l'hôtel de ville,
                           qui s'est faite sans cela avec beaucoup d'ordre. \bigskip


                     \end{diary}

                     \begin{diary}{14 Juin 1765}{}

                         Après midi assemblée au bureau de l'aumône
                              publique, les
                           9 directeurs y étaient. On a signé
                           les comptes de 1763 et de 1764 examinés auparavant
                           par Messieurs
                           de Maisonneuve et François. \bigskip


                         Convenu de presenter requête à l'empereur pour
                           certains ouvrages d'aumônes du temps du duc Léopold. \bigskip


                     \end{diary}
                     \begin{diary}{23 - 24 Juin 1765}{}

                         Cette nuit le feu ayant pris à Pont à
                              mousson, dans
                           une écurie, ou il y avait des chevaux du régiment
                              de Schomberg, 30 ont été brulés. M. le comte de
                              Stainville en fait venir de Mirecourt, pareil
                           nombre du régiment de Custine, pour que Schomberg soit
                           en état de paraître à la revue de Compiègne. \bigskip


                     \end{diary}

                     \begin{diary}{25 Juin 1765}{}

                         La fenaison ouverte sur le ban de
                              Nancy. \bigskip



                           Le roi de Pologne arrive à la
                              Malgrange vers 7 h. du
                           soir avant son départ de Lunéville il y avait eu
                           un grand orage. \bigskip


                     \end{diary}

                     \begin{diary}{26 Juin 1765}{}

                         Conseil à la Malgrange l'après midi. \bigskip


                         Assemblée particulière de l'Académie, où étaient
                           Messieurs
                           Du Rouvrois, de Solignac, Thibault, de
                           Tervenus,
                           Bagard, Devaux,
                              André, P.
                              Husson, Coster et Durival l'aîné. M.
                              Coster a lu un discours
                              sur l'agriculture en Lorraine, pour étre envoyé
                           au controlleur
                              général. M. Bagard
                           un discours
                              sur les eaux thermales. Le roi aurait desiré
                           que M. le premier président allat demain à la Malgrange
                           pour les affaires de l'Académie ; il s'en est excusé
                           sur ce qu'une de ses filles est actuellement inoculée. \bigskip


                     \end{diary}

                     \begin{diary}{27 Juin 1765}{}


                           M. l'abbé
                            comte de Lignéville,
                           écolâtre de la
                              primatiale est
                              mort. La place
                           d'écolâtre donnée à M. l'abbé de
                              Tervenus, qui
                           l'était déjà de l'ancien chapitre
                              de S. Georges.
                           Le canonicat vacant donné
                           à M. l'abbé de
                              Turique. \bigskip


                         J'ai vu le roi. Il m'a parlé des casernes
                           de la bibliothèque, de sa dernière fondation. Il
                           doit venir demain voir la bibliothèque à
                              l'hôtel de ville. \bigskip


                     \end{diary}

                     \begin{diary}{28 Juin 1765}{}


                           Le roi vient à la bibliothèque l'après midi ; ensuite
                           aux nouvelles casernes. Il était de très bonne
                           humeur et dans sa gaieté a lâché de bons mots. \bigskip


                     \end{diary}

                     \begin{diary}{29 Juin 1765}{}

                         Ce matin à la prière de M. le comte de
                              Stainville
                           le roi de Pologne a accordé 50 mille livres
                              de France pour continuer les casernes. Sa Majesté m'a
                           fait ordonner de lui aller parler. Je l'ai vu.
                           avant dîner. L'après dîner Sa Majesté m'a dit
                           ce qu'elle avait fait pour l'accélération du
                           bâtiment, mais qu'elle entendait que se serait sans diminution de ce que la France avait
                           accordé. \bigskip


                         Conseil l'après midi à la Malgrange. \bigskip


                     \end{diary}
                  \chapter*{Juillet 1765}\addcontentsline{toc}{chapter}{Juillet 1765}


                     \begin{diary}{01 Juillet 1765}{}

                         Grand exercice dans la prairie, des
                           4 brigades des grenadiers de France
                        \bigskip


                     \end{diary}

                     \begin{diary}{03 Juillet 1765}{}


                           Le roi part pour Commercy, à 1 h.
                           après midi. À 8 h. \up{1}/\textsubscript{2}. du
                           soir madame la
                              duchesse de Grammont arrive de Metz. \bigskip


                     \end{diary}

                     \begin{diary}{04 Juillet 1765}{}

                         Grand exercice à feu dans la prairie,
                              madame
                              de Grammont et le cardinal
                              de Choiseul y
                           étoient \bigskip


                     \end{diary}

                     \begin{diary}{05 Juillet 1765}{}


                           M. le comte de Stainville qui
                           devait partir
                           la nuit dernière pour Paris,
                           n'a pu partir
                           qu'à une heure après midi, parce que madame
                              de Stainville avait eu la fièvre. \bigskip



                           M. Gandoger a été admis au collège royal
                              de médecine sans examen et dispense des
                           droits. Les S.\up{rs}
                           Cupers, Platel et Gormand
                           s'y opposaient. M. l'intendant nommé
                           honoraire. Il est parti ce soir pour Neuviller. \bigskip


                     \end{diary}

                     \begin{diary}{06 Juillet 1765}{}


                           Madame la duchesse de Grammont est partie
                           à 8 h \up{1}/\textsubscript{2} du matin pour
                              Plombières.
                           Elle fit hier donner une bouteille de vin
                           à chaque grenadier. Il y avait encore
                           eu grand exercice à feu devant elle
                           dans la prairie. \bigskip



                           J.B. Salmon reçu maître de latin par
                           l'hôtel de ville. \bigskip


                     \end{diary}




                     \begin{diary}{07 Juillet 1765}{}

                         Hier le pain blanc fut taxé 2\up{s.} la livre
                           le bis à 1\up{s.} 4\up{d}
                           \up{1}/\textsubscript{2} à commencer d'aujourd'hui.
                        \bigskip


                     \end{diary}

                     \begin{diary}{08 Juillet 1765}{}

                         Assemblée à la communauté de prêtres,
                           où
                           s'est trouvé pour la première fois comme
                           écolâtre M. l'abbé de Tervenus,
                           on y a reçu
                           M. l'abbé Bagard, qui avait
                           été admis dès
                           le mois d'août de l'année dernière.
                           Il a été
                           question des comptes du S. abbé
                              Thouvenin
                           de sa présence ; mais à l'ordinnaire rien de décidé. \bigskip


                     \end{diary}

                     \begin{diary}{09 Juillet 1765}{}

                         Les 1.\up{re} et
                              3.\up{e} brigades des grenadiers
                           de France sont parties à 3 \up{1}/\textsubscript{2} h. ce matin.
                           Il y a encore eu de la désertion. \bigskip


                         J'ai en vertu d'un arrêt du conseil
                           fait faire la livraison du jardin de
                           l'abbé Grandpair, près
                              la porte S.
                              Georges.
                           Il contient 5 jours 2 ommées 2 toises 14 pieds 2 pouces de Lorraine
                        \bigskip


                     \end{diary}

                     \begin{diary}{10 Juillet 1765}{}

                         Arrivé à 10 h. du matin d'un bataillon
                           de gardes lorraines, qui a relevé à la garde
                           des grenadiers de France
                        \bigskip


                         Assemblée de l'Académie ou étaient Messieurs
                           Du Rouvrois
                           de Solignac, de Beauchamps, Thibault, de
                           Tervenus,
                           André, Bagard, Cupers,
                              P. Leslie, P. Husson, Coster,
                           Durival l'aîné. \bigskip



                           M. de Tervenus a lu l'éloge historique de
                              D. Remy
                                 Cellier, \sout{abbé} prélat de Flavigny. \bigskip


                     \end{diary}


                     \begin{diary}{11 Juillet 1765}{}

                         Les 2.\up{e} et 4.\up{e} brigades des grenadiers de
                           France partent de Nancy à 3 h
                              \up{1}/\textsubscript{2} du matin. \bigskip


                         On fait la visite des corps de garde,
                           des
                           casernes ; tout est terminé avec satisfaction. \bigskip


                     \end{diary}

                     \begin{diary}{12 Juillet 1765}{}

                         On passe par les verges une fille libertine
                           sur le rempart avant
                           à la garde montante. Il y avait longtemps
                           que cela ne s'était vu à Nancy. \bigskip


                     \end{diary}

                     \begin{diary}{13 Juillet 1765}{}

                         Orage, mêlé de grêle, à Lunéville, Flavigny
                           et autres endroits. \bigskip


                     \end{diary}

                     \begin{diary}{14 Juillet 1765}{}

                         Dans les environs de Bar une bête féroce,
                           qu'on présume n'être autre chose qu'un loup,
                           a mordu 20 à 25 personnes et dévoré
                           plusieurs enfants. Un laboureur auprès de
                               Void a aussi été attaqué et mordu par
                           un loup. Le roi de Pologne a eu ces jours
                           derniers une légère indisposition à Commercy. \bigskip


                     \end{diary}

                     \begin{diary}{15 Juillet 1765}{}

                         Nous avons été obligés, par la
                           faiblesse
                           du bataillon des gardes lorraines, de faire monter
                           la garde à 36 bourgeois, à commencer
                           d'aujourd'hui, pour garder les portes de la
                              place du marché, et des portes S. Nicolas
                           S. Georges,
                              S.\up{te}
                              Catherine et S.\up{t} Stanislas. \bigskip


                     \end{diary}

                     \begin{diary}{17 Juillet 1765}{}

                         Je suis allé à Lunéville. Il y avait une affaire assès vive entre M. de La Salle aide major
                           des gardes du corps et M. de S.
                              Simon, qui
                           lui avait ordonné les arrêts, et à M. de
                              Bray, qui n'avaient point obéis. \bigskip


                     \end{diary}

                     \begin{diary}{20 Juillet 1765}{}

                         Revenu le
                              20 à Nancy. On venait
                           d'y
                           amener deux accusés. \bigskip


                         Les corps de fer substitués à celui de
                           plomb,
                           à l'aqueduc de
                              Boudonville, ont rendu l'eau
                           à la ville vielle qui en
                           manquait depuis
                           deux mois. \bigskip



                           La jeune veuve Dujard, qui
                           était avec M.
                              Hocquet fils son frère, dans un cabriolet,
                           a été volée sur le grand chemin, entre
                           Nancy et Champigneulle, par un homme
                           armé de pistolets. \bigskip


                     \end{diary}

                     \begin{diary}{22 Juillet 1765}{}


                           Mon jeune frère m'écrit
                           de Commercy :
                           \og le roi ne souffre
                              d'aucune indisposition,
                              extraordinaire, cependant il est fort affaissé
                              depuis 3 jours et ne marche qu'avec
                              une extreme difficulté. On voudrait qu'il
                              fut à Lunéville. On
                              pense même à le
                              déterminer à abandonner le projet du
                              voyage de Versailles.
                              Quoique ce prince
                              sente bien son affaiblissement il n'en
                              paraît nullement affecté  \fg{}. Il ajoute par postscript : \og Le roi a recouvert plus de
                              force et a les mains plus fraîches \fg{}. \bigskip


                     \end{diary}

                     \begin{diary}{22 Juillet 1765}{}

                         Mort de M. Dalmas, à 4 h. après midi,
                           il n'a été que huit jours malade.
                           Il avait 75 ans. Inhumé dans l’église S. Epvre. \bigskip


                     \end{diary}

                     \begin{diary}{23 Juillet 1765}{}

                         On m'écrit de Commercy
                           le 23 : \og le roi
                              dort bien et mange à son ordinaire, mail il
                              y a affaiblissement sensible dans les nerfs des
                              jambes et du bras gauche, et dans le ressort des
                              principaux viscères, puisque souvent il ne
                              sent pas couler les urines ... Il paraît peu
                              affecté de sont état, et ne change rien à
                              ses exercices ordinaires \fg{}. \bigskip


                     \end{diary}

                     \begin{diary}{24 Juillet 1765}{}


                           L'hôtel de ville a fait cession au
                              S. Claude
                              Mique son architecte, du terrain qui
                           se trouvera entre le jardin de
                              botanique
                           et la sécherie des lavoirs
                              publics, entre le
                           nouveau mur de ville et l'ancien. Le terrain
                           contient 3 jours 2 ommées 20 toises 2 pieds 11 pouces.
                           Concession aussi de 2 lignes d'eau de diamètre
                           à Michel Godechaux. Autant à
                              M. Hoffman. \bigskip


                     \end{diary}

                     \begin{diary}{25 Juillet 1765}{}


                           Mon frère m'écrit de
                              Commercy
                           le 25 à
                           4 h. après midi : \og Le roi
                              est aujourd'hui dans un
                              état aussi satisfaisant qu'on pouvait le desirer.
                              Il est allé à une grande messe à la paroisse,
                              avec plus de nerf qu'on n'en espérait dèsormais ; en sorte que nous sommes rassurés
                              pour le moment.
                              Un courrier de la reine
                              apporta hier la nouvelle
                              de la mort de l'infant dom
                                 Philipe.
                                 M. le prince
                                 de Beauvau est parti immédiatement après dîner,
                              pour retourner à Compiègne \fg{}. \bigskip


                     \end{diary}

                     \begin{diary}{26 Juillet 1765}{}

                         Il m'écrit du 26 : \og après un sujet d'alarmes tel
                              que celui que nous avons eu, on pense toujours
                              qu'il y aurait danger plus imminent que jamais
                              dans l'entreprise du voyage, et je crois qu'on a
                              écrit en conséquence pour déterminer la reine
                              à engager le roi son
                                 père à ne pas en faire cette
                              année \fg{}. \bigskip


                     \end{diary}

                     \begin{diary}{27 Juillet 1765}{}

                         Je vais à Neuviller et j'en reviens le
                              28. \bigskip


                     \end{diary}

                     \begin{diary}{28 Juillet 1765}{}


                           Mon jeune frère m'écrit
                           de Commercy
                           \og Sa Majesté
                              fit hier (27) le tour du canal et
                              des bosquets dans
                              sa carriole, et a un reste d'affaissement près,
                              elle est mieux aujourd'hui que nous n'osions
                              l'espérer. \fg{}
                        \bigskip


                     \end{diary}

                     \begin{diary}{29 Juillet 1765}{}

                         Il arrive de Pont à mousson un convoy de 400
                           bombes à conduire par 27 voitures à Strasbourg. \bigskip


                     \end{diary}

                     \begin{diary}{30 Juillet 1765}{}

                         Mort de l'abbé Joseph-Bernard de
                              Willemin chanoine de la
                              primatiale.
                           On casse la tête à un
                           déserteur des gardes Lorraines. \bigskip


                     \end{diary}


                     \begin{diary}{31 Juillet 1765}{}

                         On m'écrit d'hier de Commercy : \og la santé
                              du roi se soutient, et
                              à l'affaiblissement près des jambes et des reins, on peut dire qu'elle est
                              aussi bonne qu'elle fut jamais. La reine insiste
                              pour la suppression du voyage de Versailles,
                              mais le prince n'y veut pas entendre \fg{}. \bigskip


                         Le canonicat de l'abbé Willemin est donné
                           à M. de Marcol chanoine de
                              S. Diey ; et ce
                           dernier bénéfice destiné à l'abbé
                              Journu. \bigskip


                         Assemblée particulière de l'Académie, où étaient
                           Messieurs
                           Du Rouvrois, de Solignac, de
                              Tervenus,
                           André, P.
                              de Husson, Durival
                              l'aîne. On
                           y a lu quelque chose d'un ouvrage de
                                 l'abbé
                                 Sigorque contre les lettres de la montagne. \bigskip



                           L'Académie a refusé la dédicace
                           d'un
                           livre de l'abbé S.
                              Mihiel, chanoine de Bouxieres,
                           sous ce titre : \emph{Ortograf des dames}.
                           Cet ouvrage avait concouru inutilement
                           au prix. \bigskip


                     \end{diary}
                  \chapter*{Août 1765}\addcontentsline{toc}{chapter}{Août 1765}



                     \begin{diary}{01 Août 1765}{}


                           M. le marquis d'Hericourt, commandant le
                           régiment du roi vient à Nancy
                           prendre
                           connaissance des logements militaires. \bigskip


                     \end{diary}

                     \begin{diary}{03 Août 1765}{}

                         Cette nuit après de grandes chaleurs
                           et beaucoup de sécheresse il a tombé un
                           peu de pluie. Et cet après midi de la grêle
                           melée de pluie. \bigskip


                         On m'écrit de Commercy
                           le 2 que le
                              roi continue a jouir d'une bonne santé
                           et qu'il sera difficile de le disuader du
                           voyage de Versailles. Qu'il
                           y a un loup
                           d'Ardennes aux environs de Sampigny et
                           de Mescring qui y répand l'alarme : il a
                           déjà attaqué et mordu grièvement un homme
                           et une femme. \bigskip


                     \end{diary}

                     \begin{diary}{04 Août 1765}{}


                           M. l'intendant va à Commercy avec M.
                              l'abbé de S. Mihiel son frère. \bigskip


                     \end{diary}

                     \begin{diary}{05 Août 1765}{}


                           Madame la duchesse de Grammont part de
                           Nancy pour Commercy l'après midi.
                           Mort de madame

                              Rorté, agée de 84 ans, mère du marquis Dessalles. \bigskip



                           L'abesse de S. Louis de Metz
                           arrive à Commercy
                           et madame la duchesse de Grammont presque en
                           même temps. \bigskip


                     \end{diary}

                     \begin{diary}{06 Août 1765}{}

                         Le lendemain madame la princesse de Saxe
                           coadjutrice de Remiremont y arrive aussi. \bigskip


                     \end{diary}

                     \begin{diary}{06 Août 1765}{}


                           M. Thibault de Monbois, maître des comptes,
                           épouse M.\up{lle} Friant, élève de S. Cyr. \bigskip


                     \end{diary}

                     \begin{diary}{07 Août 1765}{}

                         Une voleuse a été pendue. Elle
                           avait servi
                           dans plusieurs maisons de Nancy
                           et accusait
                           tous ses maîtres. Elle avait volé chez
                           M. Mengin
                           sous prétexte d'y faire sa déclaration de grossesse. \bigskip


                         Un courrier arrive à Commercy, portant
                           la nouvelle que la reine
                              de France partira
                           de Compiègne
                           le 17 pour venir en Lorraine
                           voir le roi son père.
                        \bigskip


                     \end{diary}

                     \begin{diary}{08 Août 1765}{}

                         Cette nouvelle nous fut apportée le 6
                           par M. l'intendant et
                              M. l'abbé de S.
                              Mihiel, qui partiront demain pour Neuviller. \bigskip


                     \end{diary}

                     \begin{diary}{09 Août 1765}{}

                         Mort de madame Catalde, supérieure des
                           dames prêcheresses. \bigskip



                           La duchesse de Grammont
                           arrive à Lunéville
                           à 7 h \up{1}/\textsubscript{2} du soir \bigskip


                     \end{diary}

                     \begin{diary}{10 Août 1765}{}

                         Elle part le lendemain, entrainant les
                           dames,
                           les officiers, total 32 chevaux. \bigskip


                     \end{diary}

                     \begin{diary}{14 Août 1765}{}

                         On n'a pu dissuader le roi de son voyage
                           ordinaire de Bonsecours, pour la Notre-Dame de
                           demain ; et Sa Majesté est arrivée à la
                              Malgrange
                           cet après midi, vers 5 h \up{1}/\textsubscript{2}. M. le chancelier
                           était dans la même voiture. La suite fort
                           peu nombreuse. \bigskip


                     \end{diary}



                     \begin{diary}{15 Août 1765}{}


                           M. le chancelier a reçu à
                              la Malgrange,
                           un courrier de France à une heure du matin,
                           touchant les dispositions du voyage de la
                              reine. Il avait déjà envoyé à M. l'intendant
                           à Neuviller, la lettre et
                           les états de M. le
                              comte de S. Florentin. À 4 h. du matin
                           M. le chancelier est parti
                           pour Lunéville
                           où il a tenu l'audience des sceaux, et il
                           est arrivé à 10 h \up{3}/\textsubscript{4} aux
                              minimes de
                              Bonsecours,
                           au moment que Sa Majesté allait
                           s'y mettre à table. Sa Majesté a mangé au
                           réfectoire et était en voiture pour Commercy
                           un peu avant midi. Elle a reçu avec
                           bonté toutes les personnes que l'affection,
                           où la curiosité avaient amenées. Il y a
                           eu beaucoup de peuple à son passage et à
                           son départ. On n'a remarqué à l'extérieur
                           aucun signe de dérangement et d'altération
                           dans la santé du roi. Mais
                           hier à
                           son arrivé elle souffrait d'une espèce
                           de rétention d'urine, occasionnée par le
                           voyage. \bigskip


                     \end{diary}

                     \begin{diary}{16 Août 1765}{}

                         On m'écrit de Commercy que Sa Majesté Polonaise
                           y est arrivé hier en bonne santé, et
                           y a trouvé le duc de Fleury,
                           gouverneur
                           de la province, qui y restera pendant tout le voyage de la reine. \bigskip


                     \end{diary}

                     \begin{diary}{17 Août 1765}{}


                           M. l'intendant a passé à
                              Nancy venant
                           de Neuviller, avec M l'abbé de S. Mihiel
                           son frère. Ils sont partis après dîner
                           pour Commercy, avec mon frère le commissaire
                           M. le duc de Fleury et M. de La Galaizière
                              intendant, doivent aller à Saudrupt demain
                           au devant de la
                           reine. \bigskip


                     \end{diary}

                     \begin{diary}{18 Août 1765}{}

                         On m'écrit de Commercy
                           \og M. l'intendant
                              arriva hier vers 8 h \up{1}/\textsubscript{2} il est parti cet après
                              midi avec M. le duc de Fleury,
                              pour se rendre
                              à Bar. Notre frère les dévance de quelques
                              heures pour aller s'assurer des dispositions
                              arrêtées pour le service. La reine dînera
                              à Saudrupt, et arrivera
                                 demain vers 4 ou 5 h.
                              On croit que le roi ira
                              à sa rencontre jusqu'à
                              S. Aubin ; on voudrait
                              qu'il n'alla qu'à la
                                 fontaine royale. Toutes les maisons des
                              bourgeois sont remplies \fg{}. \bigskip


                     \end{diary}

                     \begin{diary}{19 Août 1765}{}


                           L'hôtel de ville qui avait remis son
                           repas
                           de la S. Roch
                           du 16, l'a donné aujourd'hui
                           aux capucins. Les deux procureurs et généraux
                           y étaient. En tout 23 personnes. \bigskip


                     \end{diary}

                     \begin{diary}{20 Août 1765}{}

                         Assemblée du bureau de l'aumône, où
                           étaient Messieurs
                           Du Rouvrois, de Dombâle, de
                              Tervenus et de
                              Bressey, et moi. On a
                           proposé
                           de préter à la ville 8000\up{\#.} sur environ 8600 que l'empereur a fait payer au bureau, d'arrerages
                           d'aumônes, en remboursant quand on avertirait
                           à 3 mois d'avance, et payant les intérêts à
                           5 p % en attendant. Messieurs
                           de Tervenus et de
                              Bressey ont pretendu par des raisonnements de
                           théologiens que cela n'était pas permis. On a
                           remis l'examen à une assemblée plus nombreuse. \bigskip


                     \end{diary}

                     \begin{diary}{20 Août 1765}{}

                         Lettre de mon jeune frère, écrite de Commercy. \og
                              La reine arriva
                              hier vers six heures du
                              soir. Le roi étoit
                              allé à sa rencontre jusqu'à
                              S. Aubin, où il attendit
                              plus de trois heures.
                              C'est là que s'est faite la première entrevue
                              de leurs Majestés dans la maison de Schmidt ...
                              Bien en a pris à M.
                                 l'intendant d'être allé
                              en avant, car sans cela la reine, qui s'attendait
                              que le roi son père
                              lui donnerait à dîner à
                              Saudrupt, n'y aurait
                              rien trouvé. Ce ne fût
                              que par hazard que M.
                                 l'intendant
                              apprit
                              bien avant dans la nuit du \sout{samedi au}
                              dimanche au lundi qu'il n'y avait
                              aucune disposition
                              faite pour recevoir la
                                 reine ... Le lundi
                              à midi il a fait servir sept tables. Notre
                                 frère a fait en cette occasion l'office de
                              contrôleur de la bouche, et
                              il est resté pour
                              arrêter les comptes et solder. madame de Najac
                              et M. sont logés dans mon
                              corridor, madame
                                 Thibault et M.\up{lle} Marchand en sont tous
                              à portée.  \fg{}
                        \bigskip


                     \end{diary}

                     \begin{diary}{22 Août 1765}{}


                           Madame la princesse de Saxe a passé à
                           Nancy, venant de Commercy et retournant
                           à Remiremont. \bigskip


                     \end{diary}
                     \begin{diary}{23 Août 1765}{}

                         On a su par le courrier d'Allemagne de
                           ce matin que l'empereur était mort. Cette
                           nouvelle m'a été confirmée l'après midi
                           par mon frère qui
                           arrivait de Commercy.
                           M. le marquis du Chastelet en avait eu une
                           lettre de M. de Lomont. \bigskip



                           La reine est allée à
                              la fontaine royale. Elle
                           a pris trois oiseaux à la pipée, leur a fait
                           laver les ailes et leur a rendu la liberté. \bigskip


                     \end{diary}

                     \begin{diary}{24 Août 1765}{}

                         Je suis allé à Commercy, où j'ai appris les
                           particularités de la mort de l'empereur, arrivé
                           à Inspruck, où le duc Léopold son père était
                           né
                           en 1679. J'ai vu M. le chancelier ; M.
                              l'intendant qui allait partir pour Neuviller.
                           Le roi avait ordonné à
                              M. Alliot de me présenter
                           à la reine, mais je
                           n'ai pas eu de temps, et
                           il fallait d'abord voir madame de Noailles. Le roi et
                           la reine mangeaient à
                           une même table, l'un
                           vis à vis de l'autre. Point d'autre homme que
                           le roi ; environ 15 dames.
                              Madame la duchesse de
                              Duras est jeune et belle. Les 3 évêques étaient
                           à Commercy ; il y est venu
                           des jésuites de Nancy.
                           La reine était
                           sur le point d'aller au pavillon
                           lorsque je suis reparti pour revenir à Nancy.
                           La nuit m'a pris en sortant de Toul. \bigskip


                     \end{diary}


                     \begin{diary}{25 Août 1765}{}


                           Après
                              midi
                           25 \up{1}/\textsubscript{2} degrés de chaud. \bigskip


                         À Commercy la ville avait fait élever
                           un arc de triomphe, avec des devises et des
                           inscriptions emblématiques, en
                              face du
                              château. Cet édifice fut illuminé. Le 24
                           la reine s'avança
                           dans la rue au dehors
                           du château, et sa présence inattendue combla
                           la satisfaction des bourgeois. L'affluence
                           des gens de la campagne et des villes voisines
                           \sout{aup} a donné un spectacle intéressant
                           toute la journée. La multitude a rempli
                           jusqu'à la nuit les cours, les salles et les
                           jardins du château ; elle vient encore d'assiéger
                           les fenêtres de la
                              reine, qui s'en fait voir.
                           Il y a ordre de laisser entrer tout le monde
                           pendant le dîner. \bigskip



                           Clement
                           Hogard
                           incendié à Ville en Vermois,
                           22 chevaux brulés \&\up{a}. \bigskip


                     \end{diary}

                     \begin{diary}{26 Août 1765}{}


                           Madame la duchesse de Villars a passé à
                           7 h. du soir. \bigskip


                     \end{diary}

                     \begin{diary}{27 Août 1765}{}


                           Commercy. La reine a continué de manger
                           hier et aujourd'hui en public, et on remarque
                           toujours la même satisfaction dans le père et la
                              fille
                        \bigskip


                     \end{diary}

                     \begin{diary}{28 Août 1765}{}


                           15 degrés de chaud à 7 h. du matin
                           25 à midi. \bigskip



                           Le duc de Fleury passe incognito
                           \sout{Le Sain} pour retourner à Commercy. \bigskip


                     \end{diary}


                     \begin{diary}{29 Août 1765}{}


                           Madame la comtesse de Noailles, madame la duchesse de
                              Duras et M. le marquis de Saulx arrivés de la
                           veille partent pour Lunéville, après avoir
                           vu l'intendance la rotonde, l'hôtel de
                              ville \&\up{a}. \bigskip


                         La chaleur était accablante,
                           mais après
                           un grand vent qui élevait la poussière aux
                           rues, il a tombé un peu de pluie vers
                           3 h \up{1}/\textsubscript{2}
                           après midi. \bigskip


                     \end{diary}

                     \begin{diary}{31 Août 1765}{}

                         Les 1.\up{er} et
                              3.\up{e} bataillons du régiment
                              du roi arrivent vers dix heures du
                           matin. Ils avaient fait
                           halte près de la
                           belle croix pour changer de linge \&\up{a}. \bigskip


                     \end{diary}
                  \chapter*{Septembre 1765}\addcontentsline{toc}{chapter}{Septembre 1765}




                     \begin{diary}{01 Septembre 1765}{}

                         Mort de Masson, du café
                              royal. Il laisse
                           six filles et un fils. \bigskip


                     \end{diary}

                     \begin{diary}{02 Septembre 1765}{}

                         Arrivent à 9 h \up{1}/\textsubscript{2} du matin les 2.\up{d}
                           et 4.\up{e} bataillons du régiment du roi.
                           À midi le cardinal de Rochechouard, la
                              duchesse d'Aiguillon, la
                              marquise de
                              Valbelle, et le marquis de Bart, qui viennent
                           de Commercy voir Nancy \&\up{a}. \bigskip


                     \end{diary}

                     \begin{diary}{03 Septembre 1765}{}


                           M. le comte de Guerchy, colonel
                           du régiment du
                           roi arrive à 8 h. du soir. \bigskip


                     \end{diary}

                     \begin{diary}{04 Septembre 1765}{}

                         Le lendemain il fait la revue de son régiment.
                        \bigskip



                           M. de La Sône 1.\up{er} médecin de la
                              reine passe
                           venant de Lunéville et
                           retournant à Commercy.
                           Je lui ai donné à dîner, à M. de La
                              Chataigneraye
                           qui était avec lui, et à M.
                              Bagard. Ils ont
                           vu Nancy et surtout
                              le jardin botanique. \bigskip


                         On a decidé dans l'assemblée de
                              l'hôtel de
                              ville que le S.
                              Lhuillier serait commissaire de police
                           adjoint pour la paroisse
                              Notre-Dame et le S. La
                              Riviere
                           pour la paroisse S. Roch. \bigskip


                     \end{diary}

                     \begin{diary}{05 Septembre 1765}{}

                         Je pars pour Commercy avec M.
                              Breton
                           conseiller pour la noblesse et Richer
                           trésorier de
                           l'hôtel de ville. Nous allâmes le
                           soir même
                           à la fontaine royale, que le
                              roi a fort embellie. Girardet l'a peint en
                           perspective
                           et le tableau sera mis demain dans l'appartement
                           de la reine, qui veut
                           en faire aussi le dessin. \bigskip


                     \end{diary}

                     \begin{diary}{06 Septembre 1765}{}

                         Après avoir vu M. le chancelier et M.
                              l'intendant, et les beautés que le roi a ajouté
                           à Commercy, nous allons voir
                              M. le duc de
                              Fleury, qui a causé longtemps avec nous. Nous
                           l'avons prié de se laisser peindre, pour avoir
                           son portrait à l'hôtel de
                              ville ; de permettre
                           qu'on lui envoie quelques liqueurs de Lunéville ;
                           on a donné à M. Louis son
                           secrétaire, au
                           lieu de liqueurs 6 louis, au valet de chambre
                           2. aux
                           laquais 1.
                           Nous avons ensuite vu
                           le roi. Il voulait me
                           présenter à la reine,
                           ce qui n'a pu se faire faute de temps,
                           parce que la reine
                           s'est arrêtée beaucoup
                           aux ursulines, où le roi et elle avaient
                           entendu la messe. Repartis a 2 h \up{1}/\textsubscript{2}. après
                           midi et arrivés à Nancy à 8.
                        \bigskip


                     \end{diary}

                     \begin{diary}{07 Septembre 1765}{}

                         J'ai vu M. le
                              comte de Guerchi logé chez
                           M. de Croismare. Il va lundi 9 à
                           Commercy, et en partira le
                           même jour pour
                           Versailles, étant pressé
                           de retourner en Angleterre. \bigskip


                         Les S.\up{rs}
                           Lhuillier et La Riviere ont preté
                           serment en qualité de commissaires de police
                           adjoints. \bigskip



                           M. le comte de Stainville arrive
                           de Metz à
                           3 h \up{1}/\textsubscript{2}
                           après midi \bigskip


                     \end{diary}


                     \begin{diary}{07 Septembre 1765}{}

                         Je reçois à dix heures du soir, par
                              M. de
                              Lenoncourt une lettre de M. le chancelier,
                           portant que le roi a
                           permis à M. le cardinal
                              de Rochechouard
                           de faire imprimer les pièces jointes, qui sont une
                           lettre circulaire de l'assemblée générale du
                           clergé de France, et les actes de
                              l'assemblée
                              de 1765 sur la religion \&\up{a}. Les pièces ont
                           été foudroyées par le Parlement de Paris. \bigskip


                     \end{diary}

                     \begin{diary}{08 Septembre 1765}{}

                         Je conviens avec Lefevre qu'il en tirera
                           2000 exemplaires, mêmes caractères et format. \bigskip



                           Le comte de Stainville part à 5 h.
                           du matin
                           pour Commercy. \bigskip


                     \end{diary}

                     \begin{diary}{09 Septembre 1765}{}


                           M. le comte de Guerchi part pour
                              Paris le
                           matin passant par Commercy. \bigskip



                           M. le comte de Stainville arrivé
                           de Commercy
                           le soir. \bigskip


                     \end{diary}

                     \begin{diary}{10 Septembre 1765}{}


                           La reine de France a
                           quitté le séjour de
                           Commercy qu'elle trouve
                           charmant, vers
                           neuf heures du matin. Les pleurs qu'elle
                           a versé en allant à sa voiture en ont fait
                           répandre à tout le monde. La scène s'est
                           renouvellée à S. Aubin, dans
                           le bâtiment
                           neuf de Schmidt, où le roi de Pologne était allé
                           attendre la reine sa
                              fille ; et c'est là que
                           s'est faite la séparation la plus touchante. \bigskip



                           Le comte de Stainville va de
                              Nancy inspecter
                           à Lunéville
                           le régiment ; gardes-lorraines. \bigskip


                     \end{diary}

                     \begin{diary}{11 Septembre 1765}{}


                           Le S. Nicolas-Joseph Gormand,
                           médecin ordinnaire
                           du roi, et secrétaire du collège royal des médecins,
                           est mort à cinq heures du matin, sera
                           demain inhumé dans l’église du S. Sacrement.
                           On prétend qu'il avait une collection de
                           mémoires contre tout le monde. \bigskip


                     \end{diary}

                     \begin{diary}{12 Septembre 1765}{}


                           Le roi de Pologne arrive de
                              Commercy.
                           À son passage par Nancy les
                           cloches ont sonné
                           en volée, et il s'est trouvé beaucoup de peuple
                           dans les rues criant \og vive le roi \fg{}. Sa Majesté est
                           arrivée à la Malgrange un
                           peu avant cinq
                           heures après midi. \bigskip


                         Les régiments d'hussards de Nassau
                           arrive aussi,
                           et un convoi de bombes pour Strasbourg. \bigskip


                     \end{diary}

                     \begin{diary}{13 Septembre 1765}{}


                           Le roi vient à 3 h. après midi voir les
                              casernes. La naissance du fronton était
                           posée, et on commençait la première
                           assise de pierre de savonnières pour le
                           tympan. \bigskip


                     \end{diary}

                     \begin{diary}{14 Septembre 1765}{}

                         On donne le premier exemplaire de la
                           réimpression des \emph{Actes de l'assemblée
                                 générale
                                 du clergé de France de 1765} et de la lettre
                           circulaire du 27 aout ; M. le cardinal
                              de Rochouard avait demandé au roi de
                           faire faire cette réimpression, et Le
                              Seur
                           en était chargé. Il y en a 500 exemplaires
                           pour l'évêque de Toul. 500
                           pour celui de Metz, et 1000 pour celui
                              de Verdun.
                           L'imprimeur en a tiré 500 de plus
                           pour son compte. \bigskip


                     \end{diary}

                     \begin{diary}{15 Septembre 1765}{}


                           Le roi de Pologne part de la
                              Malgrange
                           pour Lunéville à 1 h \up{1}/\textsubscript{2} après midi. \bigskip



                           M. le comte de Stainville à 11 h.
                           du
                           soir pour Paris. \bigskip


                     \end{diary}

                     \begin{diary}{16 Septembre 1765}{}

                         Je suis allé à 9h. du matin, avec
                              Vannson.
                           et La Rose sergents de ville à
                           l'imprimerie de
                           Marie-Marthe-Scholastique
                              Baltazard, où
                           j'ai trouvé 4 ouvriers, occupés à un in 4\degre.
                           qui était déjà à la page 416 et dont la
                           Fff était sous
                           presse. C'est la suite de la
                           troisième partie de la \emph{réponse au livre
                              intitulé : \emph{Extrait des assertions dangereuses
                                 et pernicieuses en tout genre que les soi-disans
                                 jésuites \&\up{a}}} dont j'ai trouvé plusieurs exemplaires
                           dans une chambre derrière. La
                                 D.\up{lle} Baltazar
                           était absente, après l'avoir attendu quelques
                           temps je me suis retiré avec quelques feuilles
                           de l'ouvrage et un feuillet de la copie. Je l'ai
                           fait venir l'après midi vers 4 h. Elle est convenue
                           qu'elle n'avait ni permission ni privilège pour
                           cet ouvrage mais que les jésuites lui avaient
                           dit qu'il avaient permission : sur quoi j'ai ordonné de m'apporter les exemplaires de la
                           3.\up{e} partie, les feuilles de la suite, et la
                           forme qui était sur la presse ; ce que la
                                 D.\up{lle}
                              Baltazar a éxécuté elle même, mais la
                           forme s'est rompue en sortant de la maison.
                           C'est le P. Figean qui corrige
                           les épreuves. \bigskip


                     \end{diary}

                     \begin{diary}{17 - 18 Septembre 1765}{}

                         J'en ai rendu compte à M. le chancelier.
                           Les jésuites de leur côté avaient presentés un
                           mémoire au roi à cette
                           occasion, prétendant
                           que je savais que Sa Majesté avait permis l'impression
                           de cet ouvrage. Sur quoi M.
                              Alliot m'écrivait
                           qu'il était nécessaire que j'allasse à Lunéville
                           en parler au roi. \bigskip


                     \end{diary}

                     \begin{diary}{19 Septembre 1765}{}

                         Mais M. le chancelier lui ayant fait
                           sentir le danger de ces sortes d'écrits sur
                           lesquels les parlements ont les yeux ouverts,
                           il m'a écrit que Sa Majesté permettait que l'ouvrage
                           dont il s'agit soit achevé. \bigskip


                     \end{diary}

                     \begin{diary}{20 Septembre 1765}{}

                         En conséquence j'ai envoyé chercher la
                           D.\up{lle}
                              Baltazar et lui ai rendu ses imprimés
                           après avoir pris d'elle la déclaration par
                           écrit qu'elle m'avait point montré de
                           permission. Je lui ai réitéré qu'en pareille
                           circonstance je n'en croirais que le roi
                           lui même, ou M. le
                              chancelier, tout le
                           monde n'étant pas fait pour donner
                           à connaître les volontés du souverain. \bigskip


                     \end{diary}



                     \begin{diary}{22 Septembre 1765}{}

                         On a publié l'arrêt du 26 juin et les lettres patentes
                           du 15 août concernant les maitres
                           d'écoles de
                           Nancy. Et défense d'aller dans les vignes jusqu'à
                           la vendange sans être accompagné de gardes. \bigskip


                     \end{diary}

                     \begin{diary}{23 Septembre 1765}{}

                         On a célébré aux Cordelier un service solennel
                           pour l'âme du feu empereur
                              François I. Il y est
                           venu beaucoup de monde de la campagne. Le
                              marquis Des Armoises, et le 1.\up{er} président y étaient. Les
                           permissionaires de l'empereur, plusieurs officiers lorrains
                           au service imperial \&\up{a}. Il y avait une garde
                           nombreuse du régiment ; du roy, et beaucoup d'officiers. \bigskip


                     \end{diary}

                     \begin{diary}{24 - 26 Septembre 1765}{}

                         Je suis allé à Neuviller avec M. le changeur, pour
                           parler de la pépinière de la
                              province, à former
                           auprès de Nancy. M. le chancelier y est venu
                           le 26, convenu qu'on enverra un plan
                           et le
                           projet d'arrêt à rendre. M.
                              l'intendant m'a montré
                           une lettre de main propre du duc de
                              Choiseul,
                           touchant le fonds promis pour les casernes. Revenu l'après midi à Nancy. \bigskip


                     \end{diary}

                     \begin{diary}{27 Septembre 1765}{}


                           Le S. Huart (dit Desruisseaux de
                              Commercy)
                           repart demain pour Florence,
                           où il est
                           trésorier des bâtiments. Il avait avait
                           apporté
                           nombre d'exemplaires d'un in folio qu'il a fait
                           imprimer des portraits des ducs et duchesses
                           de Lorraines. \bigskip


                     \end{diary}
                  \chapter*{Octobre 1765}\addcontentsline{toc}{chapter}{Octobre 1765}


                     \begin{diary}{02 Octobre 1765}{}


                           M. l'intendant arrivé d'avant d'hier est
                           reparti
                           \sout{avant hier} pour Neuviller aujourd'hui.
                           Hier nous examinâmes le terrain où
                           sera
                           la pépinière. \bigskip


                         Je vais à Fleville. Madame la marquise de
                              Langalleri, suisse, fort connue de M. de
                              Voltaire y étoit. \bigskip


                         A dix heures du soir je reçois un
                           paquet
                           de M. le chancelier, par un
                           cavalier
                           de maréchaussée. Il renfermait toutes les
                           expéditions de la coadjutorerie de la dignité
                           de grand prévôt,
                           comte de S. Diey, pour le S.
                              Barthelemy-Louis-Martin de Chaumont de
                              La Galaizière (fils de M. le chancelier) ensuite
                           du consentement donné par le S. Dieudonné
                              de Chaumont de Mareil, titulaire actuel.
                           Le brevet de don du roi de
                              Pologne
                           du
                              7 Janvier 1765  ; les lettres de Rome  ; et la
                           procuration du coadjuteur datée de Metz
                           le 1.\up{er}
                              octobre pour en poursuivre l'enregistrement
                           à la Cour souveraine. \bigskip


                     \end{diary}

                     \begin{diary}{03 Octobre 1765}{}

                         Le lendemain écrit à M. le chancelier et
                           touchant la clause du brevet qui porte
                           que le coadjuteur pourra siéger à la
                              Cour
                              souveraine lorsque le grand prévôt comte de
                              S. Diez n'y sera pas. Ce qui pourrait
                           faire difficulté. \bigskip


                         Assemblée particulière de l'Académie,
                           où étaient Messieurs
                           Du Rouvrois, de Tervenus, Bagard,
                              abbé Gautier, Thibault, Durival l'aîné.
                           Il ne s'est trouvé que cinq ouvrages pour
                           concourir au prix des arts. Rien sur les
                           belles lettres. Un homme de Champigneulle
                           a fait marcher dans la place sa voiture
                           qu'un homme enfermé sous le siège d'honneur
                           fait aller par des rouages fort simples,
                           avec les mouvements de recul et de côté \bigskip


                         J'ai vu aujourd'hui le baron de Burkana,
                           né à Alep en Sirie, voyageur
                           célèbre qui parle
                           toutes les langues. Il porte toujours l'habit turc,
                           et paraît âgé d'environ 55 ans. Il porte
                           un livre assez épais, où il fait inscrire depuis
                              1747 les témoignages de toutes les personnes qui
                           l'ont vu ou connu, ministres, magistrats,
                           savans \&\up{a} de tous les pays. Il m'avait demandé
                           aussi d'écrire sur son livre qui est rempli
                           d'éloges. Je lui ai dit que je n'étais pas le premier
                           magistrat à Nancy et qu'il
                           convenait qu'il
                           s'adressât d'abord à M. Du
                              Rouvrois
                           premier président
                        \bigskip


                     \end{diary}

                     \begin{diary}{04 Octobre 1765}{}

                         Je reçois à 8 h du matin réponse de
                              M.
                              le chancelier touchant le
                              coadjuteur de S. Diez.
                           Elle porte en substance que le coadjuteur a
                           sans contredit le même droit que le titulaire
                           pourvu qu'ils ne l'exercent pas ensemble. Qu'au
                           surplus il n'était pas nécessaire qu'il fût fait
                           dans la requête une mention expresse de la
                           séance. \bigskip


                     \end{diary}

                     \begin{diary}{04 Octobre 1765}{}

                         On commença hier à Lunéville le deuil
                           de l'empereur pour
                           trois semaines. \bigskip


                     \end{diary}

                     \begin{diary}{05 Octobre 1765}{}

                         On a publié l'ouverture des vendanges
                           sur le ban de Nancy, pour les jeudis 10 et vendredi 11.
                        \bigskip


                         On m'écrit de Lunéville que la nuit du 4
                              au 5, le vent qui a été
                           impétueux \sout{à Nancy},
                           a renversé les caisses et déraciné plusieurs
                           arbres du bosquet. \bigskip



                           M. le chancelier m'écrit,
                           touchant la
                           coadjutorerie de S. Diez.
                           L'arrêt était levé,
                           on fait insinuer les bulles. \bigskip


                     \end{diary}

                     \begin{diary}{06 Octobre 1765}{}


                           M. l'intendant a passé ce matin
                           allant
                           à Frescati. Il fera de là une
                           tournée
                           vers Morhange, ira à
                              Strasbourg \&\up{a}. \bigskip


                     \end{diary}

                     \begin{diary}{09 Octobre 1765}{}

                         Les trois maisons de jésuites de
                              Nancy, célébrent
                           dans l’église du
                              noviciat, un
                           service pour l'empereur.
                        \bigskip


                     \end{diary}

                     \begin{diary}{11 Octobre 1765}{}


                           Les 1.\up{er} et
                                 3.\up{e}
                              bataillons de Picardie arrivent à Nancy.
                           M. Potier, commissaire ordonnateur à
                              Nancy, ses lettres
                           seront datées du 1.\up{er}
                              août. \bigskip


                     \end{diary}

                     \begin{diary}{16 Octobre 1765}{}

                         On a trouvé dans un puits à la
                              cloche le corps du
                           nommé Ment architecte, que l'on cherchait depuis 10
                           Jours \bigskip


                     \end{diary}

                     \begin{diary}{17 Octobre 1765}{}

                         Service solennel que l'empereur régnant fait
                           faire aux Cordeliers de
                              Nancy pour l'empereur François I son
                           père. Le catafalque et le
                           reste
                           étaient magnifiques  ; le premier président. M. le marquis
                              Des Armoise, madame
                                 Des Armoise, beaucoup
                           de haute noblesse, et autres personnes qualifiées
                           y ont assistés. La princesse a fait distribuer 3000\up{\#.}
                           aux pauvres domestiques du feu empereur. \bigskip


                     \end{diary}

                     \begin{diary}{18 Octobre 1765}{}

                         Aujourd'hui autre service solennel,
                           avec la
                           même pompe, de la part
                           de madame la princesse
                              Charlotte. Il y a eu le même concours. \bigskip


                         J'envoie à M. l'intendant le
                           projet d'arrêt
                           de la pépinière. \bigskip



                           M. le chancelier, M. l'intendant
                           madame l'intendante
                           \&\up{a}. sont allés à S. Diey pour l'installation
                           de M. l'abbé de S.
                              Mihiel, à la grande prévôté. \bigskip


                     \end{diary}

                     \begin{diary}{19 Octobre 1765}{}


                           M. Chapuis père, s'est trouvé
                           mal en entrant
                           à l'hôtel de ville. On l'a
                           reporté chez lui \bigskip


                         La chambre a accordé à M. Sirejean,
                              fils, un
                           filet d'eau d'une ligne et demie de diamètre,
                           pour la maison du reclu qu'il a acheté de l'hôpital
                              S. Julien ; ayant prouvé par une ancienne file
                           encore existante que cette maison en tirait
                           anciennement du bouge des princes qui en est
                           très à portée. Cela n'eut pas lieu alors par
                           la faute de M. Sirejean. \bigskip


                     \end{diary}

                     \begin{diary}{20 Octobre 1765}{}

                         Il y a eu aujourd'hui assemblée publique
                           de
                           l'Académie, mais je ne m'y
                           suis pas trouvé, étant allé à Heillecourt dès le
                           matin, pour ma santé. \bigskip


                     \end{diary}

                     \begin{diary}{25 Octobre 1765}{}

                         On s'impatiente beaucoup à
                              Nancy de ne
                           voir point arriver Fleury et sa
                           troupe. On
                           fait cent conjectures sur ce retard. Qu'il
                           a péri en mer, qu'il a été emmené en Afrique
                           par les corsaires \&\up{a}. \bigskip


                     \end{diary}

                     \begin{diary}{31 Octobre 1765}{}

                         J'écris à M. le chancelier sur la plainte
                           du S. Pitoux, et je propose
                           d'abolir à S.
                              Nicolas comme dans les autres villes, le
                           crieur qui va pendant la
                           nuit en certains
                           de temps de l'année crier pour les trepassés. \bigskip



                           \sout{Il} Il y a quelques jours que l'arrêt pour
                              la
                              \emph{pépinière royale} de Nancy est rendu. Il y
                           en a un exemplaire pour enregistrer à la
                              chambre des comptes. \bigskip


                     \end{diary}
                  \chapter*{Novembre 1765}\addcontentsline{toc}{chapter}{Novembre 1765}




                     \begin{diary}{02 Novembre 1765}{}

                         On m'écrit de Lunéville le 2 novembre.
                           Un prétendu comte de
                              Steinbock, livonien,
                           s'est annoncé ici sous ce titre, a été présenté
                           de même par M. le comte de
                              Zuily, et a dîné
                           à la table du roi. On a appris par l'arrivée
                           du courrier que ce n'est qu'un aventurier,
                           qui a excroqué 40 louis à Paris,
                           de l'aumônier
                              de l'ambassadeur de Suède, sur une lettre de
                           change qu'on dit fausse. L'exempt de la maréchaussée
                           a en conséquence pris possession de son
                           appartement à l'arbre d'or, et le conduira
                           aux tours Notre-Dame s'il ne réalise les 40 louis.
                           Il est accompagné d'une fille qui se dit sa femme.
                           Le roi est furieux qu'on
                           l'ait fait manger
                           avec lui. C'est un grand jeune homme âgé
                           d'environ 30 ans, cheveux blonds, portant
                           divers uniformes en parlant plusieurs langues. \bigskip


                     \end{diary}

                     \begin{diary}{04 Novembre 1765}{}

                         Du 4 à Lunéville. le
                              prétendu comte de
                              Steinbock a laissé ici des effets pour nantissement
                           des 40
                              louis empruntés, et est parti cet après
                           dîner, pour retourner dit-il à Paris. Les nouvelles
                           de la santé de M. le
                              Dauphin sont un peu meilleures. \bigskip



                           Le S. Collancy, intéressé dans
                           les domaines,
                           mort à Nancy
                           le 3. \bigskip


                     \end{diary}

                     \begin{diary}{07 Novembre 1765}{}


                           La femme de M. Guire, premier
                           commis
                           de la chancellerie, meurt à Lunéville. On m'écrit du 7 de
                              Lunéville que l'étranger
                           qui a été arrêté était véritablement comte
                              de Steinbock. \bigskip


                     \end{diary}

                     \begin{diary}{08 Novembre 1765}{}


                           Fleury, directeur de la comédie arrive,
                           le reste de la troupe l'avait précédé de
                           plusieurs jours. \bigskip


                     \end{diary}

                     \begin{diary}{09 Novembre 1765}{}

                         Acte public de M.Gandoger, au collège
                              royal des médecins. \bigskip


                     \end{diary}

                     \begin{diary}{10 Novembre 1765}{}

                         J'ai fait mettre en prison Hœner, pour avoir
                           imprimé : \emph{Histoire de Sinal}, traduite
                           d'un
                           manuscrit hébreu, à Smirne
                           1765, in
                              12 de
                           14 pages. C'est une sanglante satyre contre
                           Messieurs
                           de Bains et de Serignac officiers au régiment
                           du roi. L'auteur est M. de
                                 \sout{Fontet} autre officier
                           âgé d'environ 20 ans. \bigskip


                         La comédie a donné pour sa rentrée
                              \emph{le
                              Philosophe marié} et \emph{la Jeune
                              indienne}.
                           Il y avait beaucoup de monde, et le public
                           a été content. \bigskip



                           M. de Fontet mis en prison. Il nie.
                        \bigskip


                     \end{diary}

                     \begin{diary}{11 Novembre 1765}{}

                         Service aux Minimes pour le feu empereur.
                           On apprend la mort du duc de
                              Cumberland,
                           oncle du roi d'Angleterre.
                        \bigskip


                     \end{diary}


                     \begin{diary}{12 Novembre 1765}{}

                         Services pour l'empereur, à S. Sébastien, à
                           S. Epvre et à la Congrégation. \bigskip


                         J'ai le soir fait sortir Hœner de prison. \bigskip


                     \end{diary}

                     \begin{diary}{13 Novembre 1765}{}

                         Assemblée particulière de l'Académie, où
                           étaient Messieurs
                           Du Rouvrois, de Sivry, de
                              Beauchamps, André,
                              abbé de Tervenus,
                           abbé Gautier, P. Husson, Cupers, Coster et
                           Durival l'aîné. On y a lu
                              un ouvrage sur
                              les inconvénients de la littérature, qui a été
                           admis au concours, et distribué des mémoires
                           à examiner. Il a été résolu d'indiquer
                           des sujets à traiter, sans exclure les autres. \bigskip


                     \end{diary}

                     \begin{diary}{14 Novembre 1765}{}


                           La Cour souveraine a fait sa
                           rentrée aujourd'hui.
                           C'est M. de La Milliere
                           avocat général qui a fait
                           la harangue. Le sujet
                           est : quelles belles lettres
                           sont nécessaires au magistrat. \bigskip



                           Mon frère le commissaire
                           envoie
                           un
                              mémoire militaire à la Cour. \bigskip


                     \end{diary}

                     \begin{diary}{15 Novembre 1765}{}

                         Services pour l'empereur aux Capucins, aux
                              Augustins et aux Minimes de Nancy. \bigskip


                     \end{diary}

                     \begin{diary}{16 Novembre 1765}{}

                         On apprend que M. le Dauphin a
                           été
                           administré
                           le 13. \bigskip


                     \end{diary}

                     \begin{diary}{18 Novembre 1765}{}


                           La primatiale
                           ayant annoncé un service solennel
                           à demain pour l'empereur, avait invité la
                              Cour souveraine d'y assister. Aujourd'hui à dix
                           heures du matin l'abbé de
                              Bressey est venu
                           me demander d'ordonner qu'il fait sonner
                           dans toutes les églises de Nancy, aujourd'hui
                           à midi, et a 4 et 6 heures après midi. Et
                              demain à 7 h et à 10 h. du matin. J'ai
                           envoyé mon
                              secrétaire à M. le premier président
                           qui m'a fait répondre que la Cour souveraine n'irait
                           point à ce service. À l'égard de la sonnerie
                           que cela ne le regardait pas.
                        \bigskip


                         Services pour l'empereur aux Dominiquains. \bigskip


                     \end{diary}

                     \begin{diary}{19 Novembre 1765}{}


                           La primatiale en a fait
                           aussi aujourd'hui.
                           On n'a point sonné dans les autres églises. \bigskip



                           M. l'évêque de Toul, sur les ordres
                           du roy
                              de Pologne, fait faire dans toutes les églises,
                           pendant 3 jours, les prières des 40 heures
                           pour la santé de M. le
                              Dauphin ; on les
                           commence demain. \bigskip


                     \end{diary}

                     \begin{diary}{20 Novembre 1765}{}

                         Service aux Cordeliers par les pensionnaires
                           de l'empereur. \bigskip


                         Les nouvelles de la porte sur
                              M. le Dauphin
                           sont à peu près comme les précédentes.
                           On parle de 5 membres du parlement
                              de
                              Bretagne
                           arrêtés, et d'une commission de
                           3 conseillers d'états et de 12 maitres des
                           requêtes qui doivent aller dans ce pays là. \bigskip


                     \end{diary}


                     \begin{diary}{20 Novembre 1765}{}

                         On ne jouera point la comédie
                           pendant ces trois jours de 40 heures. \bigskip


                     \end{diary}

                     \begin{diary}{21 Novembre 1765}{}

                         Services pour l'empereur aux Tiercelins ; et de
                           la part des pénitents
                           aux Cordeliers. \bigskip


                         Il a un peu gelé cette nuit, et le
                           matin
                           le thermomètre était au degré glace. \bigskip


                     \end{diary}

                     \begin{diary}{22 Novembre 1765}{}

                         Service aux Carmes pour l'empereur.
                           Les prières des 40 h. pour M. le
                              Dauphin, ayant fini aujourd'hui, on a
                           repris la comédie. M. le premier président en
                           avait été prévenu. \bigskip


                         On a fait hier à Lunéville des prières
                           de 40 h. pour M. le
                              Dauphin. Le roi
                              de Pologne y a assisté. \bigskip


                     \end{diary}

                     \begin{diary}{24 Novembre 1765}{}


                           M. l'intendant qui était
                           depuis deux à trois
                           jours à Nancy, en est parti
                           aujourd'hui pour Lunéville. \bigskip


                     \end{diary}

                     \begin{diary}{25 Novembre 1765}{}

                         Les nouvelles arrivées par la
                              porte d'aujourd'hui
                           en donne d'assez satisfaisantes de M. le
                              Dauphin
                           des 21 et 22. \bigskip


                         Mort de M.  baron de Philibert
                        \bigskip


                     \end{diary}

                     \begin{diary}{26 Novembre 1765}{}

                         Services pour l'empereur à l'hôpital S. Julien,
                           et aux Dominiquains de
                              Nancy. \bigskip



                           Madame l'intendante arrive de Neuviller pour
                           passer l'hiver à Nancy. \bigskip


                     \end{diary}

                     \begin{diary}{27 Novembre 1765}{}

                         Assemblée particulière de
                              l'Académie, où
                           étaient Messieurs
                           Du Rouvrois, de Sivry, Thibault,
                           de Tervenus, abbé Gautier, P.
                              Husson, Cupers,
                           André, Coster, Durival
                              l'aîné. On a parlé
                           du discours préliminaire à imprimer à la tête
                           du \emph{catalogue des livres
                                 de la bibliothèque}. M.
                              Coster en avait composé un. M. de Solignac
                           en l'abrégeant avait retranché surtout l'historique
                           des fonds destinés en achats de livres. On a
                           mis en déliberation si on imprimerait ce
                           discours ou si le catalogue resterait nu. On a
                           été d'avis d'abord de le lire au roi. Puis on a
                           ouvert un autre avis, sur ce que j'ai soutenu
                           qu'on n'avait pas besoin de permission, et que
                           l'hôtel de ville qui fait imprimer
                           le catalogue
                           pouvait y faire mettre un préliminaire ;
                           qu'il n'y avait rien là qui put blesser le roi.
                           Que moi même je pourrais faire imprimer.
                           On s'est séparé sans conclure clairement. \bigskip


                     \end{diary}

                     \begin{diary}{28 Novembre 1765}{}

                         On a trouvé ces jours-ci, dans les
                           fossés de la
                              ville vielle de Nancy, auprès de l'ardoisière, un
                           boulet de fer, ayant douze pouces de diamètre,
                           et pesant suivant le certificat de La
                              Kafouse
                           d'aujourd'hui 257\up{livres}. \bigskip


                         Services pour l'empereur à la visitation. \bigskip


                     \end{diary}

                     \begin{diary}{30 Novembre 1765}{}

                         On m'écrit du Lunéville
                           le 30 que M.
                              le marquis du Chastelet est mort chez
                           son
                              frère à Loisey. \bigskip


                     \end{diary}


                     \begin{diary}{Encart}{} Vers de M. de
                                 Solignac à M.
                                 Durival le jeune en lui renvoyant
                              le manuscrit d'un ouvrage sur la
                                 Finance. \bigskip


                              \begin{verse}Grand merci Messer
                                       Durival,\\de vos écrits sur la finance.\\Ils font espérer que la
                                       France,\\mourant faute de cordial,\\peut un jour par vôtre science\\revenir en convalescence,\\et reprendre un air triomphal.\\Mais si sortant de l'hôpital,\\de vôtre ainé le
                                       commissaire,\\elle adopte l'art militaire,\\je la tient quitte de tout mal.\\\end{verse}
        \bigskip


                           \end{diary}

                  \chapter*{Décembre 1765}\addcontentsline{toc}{chapter}{Décembre 1765}



                     \begin{diary}{02 Décembre 1765}{}

                         Service pour l'empereur aux Pénitents blancs
                           de la miséricorde de Notre-Dame de consolation de Rome,
                           en leur église ville
                              vielle. \bigskip


                     \end{diary}

                     \begin{diary}{04 Décembre 1765}{}

                         Autres services à la paroisse Notre-Dame de la part
                           des prêtres de l'oratoire. \bigskip



                           Le S. Aufresne joue dans \emph{le Duc de Foix}
                        \bigskip


                     \end{diary}

                     \begin{diary}{05 Décembre 1765}{}

                         Aujourd'hui il a joué \emph{le Misantrope}. Son
                           jeu est la nature. 3 degrés
                              \up{1}/\textsubscript{2} de froid le
                              matin. \bigskip


                     \end{diary}

                     \begin{diary}{06 Décembre 1765}{}

                         Assemblée au bureau de l'aumône, où
                           étaient Messieurs
                           Du Rouvrois, de Riocourt, de
                              Dombâle, François, de Moret,
                              de Bressey,
                           de Maisonneuve, Durival. \bigskip


                     \end{diary}

                     \begin{diary}{07 Décembre 1765}{}

                         Je reçois le serment d'une maîtresse
                           d'école
                           pour les filles de la paroisse S.
                              Fiacre. \bigskip



                           M. le chancelier a écrit au
                              collège de
                              médecine
                           que l'intention du roi
                           était que le S. Platel
                           en fut secrétaire. \bigskip


                     \end{diary}


                     \begin{diary}{11 Décembre 1765}{}


                           L'hôtel de ville à la recommandation de M. de
                              Millet reçoit M.\up{lle} Georgin suisse de la
                              paroisse
                              de S. Epvre délibére de céder à l'ordre des
                              avocats le grenier sur la voûte qui sert de
                           passage de la carrière au rempart, pour
                           y tenir des conférences et placer ses livres.
                           M. l'intendant
                           approuve la délibération. \bigskip


                         Assemblée particulière de l'Académie, où étaient Messieurs Du Rouvrois, de Tervenus, de
                              Niceville,
                           Cupers, abbé Gautier, Coster et
                              Durival l'aîné.
                           On a examiné définitivement les ouvrages
                           qui ont concouru pour les prix. Sur les
                           lettres un discours sur les inconvénients de
                              la littérature a été rejeté. Il était le seul
                           pour cette partie. Les deux prix ont été
                           ainsi distribués. 300\up{\#.} a une jeune fille
                           auteur d'un ouvrage sur la manière
                              de
                              tenir les livres de commerce. Et à titre
                           d'encouragement 100\up{\#.} à l'auteur d'un ouvrage
                              de mécanique pour sonner en volée, brider
                              et débrider les cloches sans monter au clocher.
                           100\up{\#.} à l'auteur d'une petite voiture portée
                           sur trois roues dont celle de devant qui
                           fait aller les autres, a le mouvement horizontal
                           et le vertical, ce qui est nouveau. Et 100\up{\#.} a
                           celui qui a tenté de graver l'effigie du roi,
                           en médaillon, avec l'abondance au revers
                           et cette légende. \emph{Semper sic
                              manera spargit}. \bigskip


                         Je
                           reçois par M. Duplessis une lettre
                           de
                           M. le comte de Guerchy
                           écrite de Londres
                           le 3 décembre sur l'affaire de
                              M. de Fontelle. \bigskip


                     \end{diary}

                     \begin{diary}{12 Décembre 1765}{}

                         Je réponds à M. le comte de
                              Guerchi, et lui
                           rends compte de ce qui s'est passé de ma
                           connaissance relativement à l'affaire de M.
                              de Fontelle. J'ai remis ma lettre à M. Duplessis
                           major pour l'envoyer. \bigskip


                     \end{diary}


                     \begin{diary}{Encart}{} Vers d'Aufresne, pere de
                              l'acteur,
                              à M. de Voltaire
                           \bigskip


                              \begin{verse}Servet eut tort et fut un
                                    sot\\d'oser dans un siécle falot\\s'avouer antitrinitaire ;\\et notre illustre atrabilaire\\eut tort d'employer le fagot\\pour réfuter son adversaire.\\Et tort notre antique senat\\d'avoir prété son ministére\\à ce dangereux coup d'eclat.\\Quant au censeur épistolaire,\\qui dans son pétulant effort\\vient reveiller le chat qui dort\\et dans un ample commentaire\\prôner ce qu'il auroit dû faire,\\je laisse à juger s'il a tort.\\Quant à vous célébre Voltaire,\\vous avés tort, c'est mon avis.\\Vous vous
                                    plaisés en ce païs :\\messagés ce qu'on y revère\\vous avés à satieté\\les biens où la raison aspire,\\l'opulence, la liberté\\la paix qu'en cent lieux on désire ;\\des droits à l'immortalité,\\cent fois plus qu'on ne pourroit dire\\l'on a du goût, l'on nous admire\\Tronchin veille à votre santé.\\Cela vaut bien en verité\\qu'on immole à sa sureté\\le triste plaisir de médire.\\\end{verse}
        \bigskip


                           \end{diary}


                     \begin{diary}{16 Décembre 1765}{}


                           M. Gandoger, commence son cours
                           d'anatomie, dans l'ancienne salle
                              de
                              l'Académie. \bigskip


                     \end{diary}

                     \begin{diary}{18 Décembre 1765}{}

                         Mort de Marie-Catherine de Fumeron,
                           douairière de feu M. Jaques Dalmas,
                           commissaire ordonnateur des guerres. Sera
                           inhumée dans l’église de S.
                              Epvre. \bigskip



                           Adjudication à l'hôtel de ville de
                           la ferme
                           des bois, cercles, cuveaux \&\up{a}. \bigskip


                     \end{diary}

                     \begin{diary}{19 Décembre 1765}{}


                           Le chevalier de Bouflers arrive de Fontainebleau,
                           et n'apporte que de mauvaises nouvelles
                           de M. le Dauphin. \bigskip


                     \end{diary}

                     \begin{diary}{20 Décembre 1765}{}

                         Le bulletin du 16
                              matin arrivé
                           aujourd'hui ne laisse plus d'espérance. \bigskip


                     \end{diary}

                     \begin{diary}{21 Décembre 1765}{}

                         Nous n'avons reçu aujourd'hui aucune nouvelle
                           sur M. le Dauphin.
                           Grande incertitude si on
                           jouera la comédie, ou si on fera cesser les
                           spectacles. M. l'intendant
                           me renvoie le
                              directeur, je l'envoie à M. le premier président
                           Il m'écrit, Fleury y retourne.
                           On joue. \bigskip


                         Mort de Jean-Adolphe-Nicolas Lorin,
                           directeur
                           des domaines de Lorraine. Il sera
                           inhumé dans l’église de S.
                              Roch. Il est
                           regretté. Il ne laisse qu'une fille. \bigskip


                     \end{diary}

                     \begin{diary}{22 Décembre 1765}{}

                         Mort de
                           comte de Lupcourt. \bigskip


                     \end{diary}

                     \begin{diary}{23 Décembre 1765}{}

                         Nous n'avons appris qu'aujourd'hui que M.
                              le Dauphin était mort à Fontainebleau
                           le
                              20 à 8 h. du matin. \bigskip


                         Assemblée des directeurs de la fondation
                           de l'abbé de Bousey pour le
                              refuge.
                           Les six filles sorties seront remplacées
                           par les ci-après.


                                 Anne Marie Mouchot


                                 Anne Craincourt


                                 Anne Moitrier


                                 Catherine Pile dite la
                                    fortune


                                 Marguerite Thoilley


                                 la D\up{lle}. de
                                    Potès.


                           Convenu que les 2 premières places
                           qui viendraient à vaquer pendant l'année
                           seront pour ... Legros, et
                              Anne Pogny.
                           Qu'on priera M. le chancelier de faire exprimer
                           dans les lettres de cachet pour le 2.\up{d}
                           quartier, qui est celui des pensions, différent
                           du 1.\up{er} quartier où sont les plus libertines.
                        \bigskip


                     \end{diary}

                     \begin{diary}{25 Décembre 1765}{}

                         Jour de Noël, 4 degrés de froid, à 8 h. du matin.
                           Le
                              soir
                           à zéro. \bigskip


                     \end{diary}

                     \begin{diary}{26 Décembre 1765}{}

                         Il neige. \bigskip


                     \end{diary}

                     \begin{diary}{Encart}{}
                              décembre 1765
                              chevalier De Bouflers
                           \bigskip


                              Le roy de Pologne, duc de
                              Lorraine et de
                              Bar convaincu de l'incapacité du marquis
                                 de Boufflers, a resolu de confier la compagnie
                                 de ses gardes à un officier digne de ce poste
                              important ; il a jeté les yeux sur le chevalier de
                                 Boufflers, dont l'experience, la gravité
                              la sagesse et surtout l'assiduité lui sont
                              connues, pour lui donner la survivance de
                              son frere. \bigskip


                              Sa Majesté prie M. le duc de Choiseul
                              d'obtenir en conséquence du
                                 chevalier de
                                 Boufflers un brevet de colonel, afin de
                              perpetuer l'heureux accord, qui a toujours
                              existé entre le service de Lorraine et le
                              service de France. \bigskip

         On sera peut-étre étonné que le
                                 roy de
                                 Pologne à son âge, nomme un survivant
                              à un officier de vingt neuf ans. On répond
                              que le besoin que ses gardes ont d'un
                              chef, fait passer sur toutes les objections D'ailleurs
                              l'embonpoint de Sa Majesté
                                 Polonaise et la maigreur du marquis de
                                 Boufflers compensent assés la difference
                              d'age. On pourroit trouver encore une
                              autre compensation dans les vœux que la
                                 France et la
                                 Lorraine font pour la vie
                              du roy de Pologne, et
                              ceux que toutes
                              les troupes font pour la mort du
                                 marquis
                                 de Boufflers
                           \bigskip


                              Le chevalier de
                                 Boufflers a fait la
                              guerre comme volontaire pendant quatre
                              mois ; il a extrémement fatigué le
                                 prince Ferdinant, toute la derniere
                              campagne ; c'est un sujet propre à
                              rétablir dans les troupes, cette gaieté
                              françoise que le marquis de Boufflers
                              attriste par sa sévérité, et cet ancien
                              esprit de la nation, auquel le
                                 marquis
                                 de Boufflers a porté tant d'atteintes. \bigskip

         Il aime la table, le jeu, les femmes et les
                              chevaux ; il ne cesse de boire à la santé
                              de M. le duc de Choiseul et
                              de le bénir
                              dans toutes ses chansons./.
                           \bigskip

        \end{diary}



                     \begin{diary}{27 Décembre 1765}{}


                           M. le chancelier écrivit hier
                           au procureur général
                              de la Cour souveraine, que l'intention du roi est
                           qu'il soit sursi, à toutes marques extérieures de
                           deuil, jusqu'à ce qu'on sache ce qui se fera en France.
                           Que Sa Majesté ne veut pas qu'il soit sonné pendant
                           40 jours, parce que cela ne se pratique pas en France.
                           Qu'elle a donné ses ordres à M.
                              l'intendant pour faire
                           cesser les spectacles dans tous ses états ; et qu'elle
                           fera savoir par la même voie quand elle
                           permettra qu'ils recommencent, sur quoi elle
                           se conformera à ce qui aura lieu dans le
                           royaume. Qu'elle n'a pas encore fixé le jour que
                           sa cour et ses sujets prendront le deuil, car
                           pour sa personne il n'en est pas question.
                           Que celui qui a commencé le 26 est
                           pour
                           le duc de Cumberland,
                           et qu'il durera vraisemblablement
                           jusqu'à mardi. \bigskip


                         Assemblée particulière de l'Académie, où
                           étaient Messieurs
                           Du Rouvrois, l'intendant, de
                              Sivry
                           Thibault, de Tervenus, P.
                              Leslie, Bagard,
                              Cupers,
                           André, P.
                              Husson, de Nicéville,
                              Coster, Durival l'aîné
                           Trois personnes avaient été proposées savoir : \begin{itemize}\item
                                 M. Falois fils de l'avocat du roi aux
                                    requêtes du
                                    palais, il va en Saxe, et a un ouvrage
                                    sur les
                                    fortifications
                                 approuvé par l'Académie des sciences. \item
                                 M. Le Bas médecin, qui a
                                 une grande dispute
                                 avec M. Louis ; sur les accouchements prématurés. \item
                                 M. Mittié, fils du chirurgien, qui va être
                                 reçu médecin à Paris.
                                 On a différé la réception de tous trois, sur tout parce qu'ils ne sont
                                 point connus par des ouvrages imprimés,
                                 ou que par leur profession. \end{itemize}
                        \bigskip



                           L'Académie a délibéré que M. le premier président
                           comme directeur écrirait au roi pour
                           savoir s'il agréerait une députation de
                           l'Académie au nouvel an. \bigskip


                     \end{diary}

                     \begin{diary}{30 Décembre 1765}{}

                         On prend le deuil à Lunéville, en pleureuses
                           pour M. le Dauphin. \bigskip



                           Le roi a fait répondre qu'il
                           ne recevait
                           point de compliments et de députations. \bigskip


                     \end{diary}

                     \begin{diary}{31 Décembre 1765}{}


                           5 degrés de froid à 8 h. du matin. \bigskip


                         Adjudication à l'hôtel de ville du droit
                           de jauge, pour deux
                           années. \bigskip


                         Mort de M.
                              de Gignéville, officier,
                           fils du maître des comptes. \bigskip



                     \end{diary}

               \part*{Année 1766}\addcontentsline{toc}{part}{Année 1766}\chapter*{Janvier 1766}\addcontentsline{toc}{chapter}{Janvier 1766}



                     \begin{diary}{01 Janvier 1766}{}


                           7 degrés de froid à 8 h. du matin
                           6 à 9 h.
                           Le feu à une cheminée
                           chez
                           M. de Malleloy
                           à midi. \bigskip



                           La princesse Christine passe
                           venant
                           de Remiremont et va à
                              Versailles. \bigskip


                     \end{diary}

                     \begin{diary}{02 Janvier 1766}{}


                           7 degrés de froid à 8 h \up{1}/\textsubscript{2} du matin
                        \bigskip


                     \end{diary}

                     \begin{diary}{03 Janvier 1766}{}


                           Id. \bigskip


                         On me mande de Versailles
                           le 30 décembre.
                           Le roi va demain à
                              Choisy, il n'y aura point
                           par conséquent après demain de procession de
                           cordons bleus. Il n'y aura
                           point non plus de
                           révérence. \bigskip


                         Et de Lunéville qu'hier matin on a célébré
                           à la paroisse un service pour M. le Dauphin.
                           L'après midi le conseil a commencé
                           à l'assemblée. \bigskip


                         Les glacières de Nancy remplies hier et
                           aujourd'hui. La glace à 8 et 12 pouces d'épaisseur. \bigskip


                     \end{diary}

                     \begin{diary}{06 Janvier 1766}{}


                           6 degrés
                              \up{1}/\textsubscript{2}. de froid à
                              8 h. du
                              matin. \bigskip


                     \end{diary}

                     \begin{diary}{07 Janvier 1766}{}


                           7 degrés
                              \up{1}/\textsubscript{2}. de froid à
                              8 h. du
                              matin. 4 à
                           2 h. après
                              midi. \bigskip


                         Le feu a pris cet après midi dans le
                           cabinet
                           de jour de madame l'intendante. Il a été aussitôt
                           éteint. \bigskip


                     \end{diary}

                     \begin{diary}{07 Janvier 1766}{}

                         On a pendu aujourd'hui une vielle
                           femme
                           qui volait depuis lontemps dans les églises
                              à Saint-Nicolas et ailleurs. Le cordon a séparé
                           sa tête de son corps ; ils ont tombé en
                           même temps. \bigskip


                     \end{diary}

                     \begin{diary}{08 Janvier 1766}{}

                         A onze heures du soir hier
                           8 degrés
                              \up{1}/\textsubscript{2} de
                              froid. aujourd'hui à 4 h. du matin
                           7 degrés à 9 h.
                           5 à midi
                              4, et à
                           3 h. du
                              soir
                           3 \up{1}/\textsubscript{2}.
                        \bigskip


                     \end{diary}

                     \begin{diary}{09 Janvier 1766}{}

                         Le vent est revenu au Nord-Est pendant
                           la
                           nuit, et il y avait 9 degrés de
                              froid à
                           7 h. du
                              matin
                           7 degrés à 9 h. Il avait
                           hier tombé de la neige. \bigskip


                         Assemblée publique de l'Académie où étaient
                           Messieurs
                           Du Rouvrois
                           directeur
                           de Sivry
                           sous directeur
                           de Solignac
                           secrétaire
                           Thibault, de Tervenus,
                           P. Leslie, P. Husson, abbé
                              Gautier, André
                           Bagard, Coster, Harmant,
                              Durival l'aîné.
                           M. le premier président à ouvert sa séance par
                           un discours qui renferme un éloge funèbre
                           de M. le Dauphin.
                           Ensuite il a rendu
                           compte de la disposition des deux prix
                           qui ont été données. M. de
                              Tervenus à
                           lu l'éloge de Dom Remy
                              Cellier. Après
                           la séance M. Du Rouvrois a
                           été continué
                           directeur par acclamation. \bigskip


                         Signé un acte touchant les 3000\up{\#}
                           de fonds de l'École
                              des filles de Bonsecours, prété à M. de Marcol, conseiller. \bigskip




                         Ensuite du consentement de l'hôtel de ville,
                           l'ordre des avocats
                           ayant obtenu pour
                           sa bibliothèque et tenir les assemblées des
                           conférences le dessus de la voute du passage
                           qui va de la carrière
                           au rempart près du
                           palais ; les grains du
                           magasin d'abondance
                           qui y étaient ont été transportés ces jours
                           derniers dans la salle des cerfs, où était
                           cy-devant
                           la bibliothèque, avec partie de ceux de
                           la monnaie. \bigskip


                     \end{diary}

                     \begin{diary}{10 Janvier 1766}{}


                           Froid 10 degrés
                              \up{3}/\textsubscript{4} à 7 heures du
                              matin.
                           10 degrés à 9 h. \bigskip


                     \end{diary}

                     \begin{diary}{11 Janvier 1766}{}


                           Aujoud'hui à 8 h.
                              du matin
                           8 degrés
                              \up{1}/\textsubscript{2}
                        \bigskip


                         Service que sont célébrer les PP. Bénédictins
                           au monastère du S.
                              Sacrement, pour
                           M. le Dauphin. Le 8 ils en avaient
                           fait un dans leur église de S.
                              Leopold. \bigskip



                           Le théâtre se rouvre
                           aujourd'hui, avec la
                           permission du roi, et
                              le S. Aufresne joue
                           dans \emph{Mithridate}
                        \bigskip


                     \end{diary}

                     \begin{diary}{15 Janvier 1766}{}

                         Dégel. \bigskip


                     \end{diary}
                     \begin{diary}{16 Janvier 1766}{}

                         Le froid reprend 6 degrés
                              \up{1}/\textsubscript{2} à 7 h. du matin
                        \bigskip


                         Assemblée particulière de l'Académie. \bigskip


                     \end{diary}

                     \begin{diary}{17 Janvier 1766}{}


                           8 degrés de froid. Le
                              roi de Pologne a écrit à
                           M. le cardinal de Choiseul, pour faire célébrer
                           à la primatiale un
                           service pour M. le Dauphin.
                        \bigskip


                     \end{diary}

                     \begin{diary}{18 Janvier 1766}{}


                           Froid 8 degrés
                              \up{3}/\textsubscript{4}
                        \bigskip


                         Les officiers du régiment du roi ont fait célébrer
                           ce matin aux jardins avec beaucoup de
                           solennité et de décorations un service pour
                           M. le Dauphin. L'abbé de S. Mihiel part pour
                              Paris. \bigskip


                     \end{diary}

                     \begin{diary}{19 Janvier 1766}{}


                           Froid 6. degrés
                        \bigskip


                     \end{diary}

                     \begin{diary}{20 Janvier 1766}{}


                           3 degrés
                              \up{1}/\textsubscript{2} de froid
                        \bigskip


                     \end{diary}

                     \begin{diary}{22 Janvier 1766}{}

                         On fait des services pour M. le Dauphin,
                           dans les sept paroisses de Nancy. \bigskip


                     \end{diary}

                     \begin{diary}{27 Janvier 1766}{}

                         On adresse à la Cour de France et à la Chambre des
                              Comptes des lettres de cachet, pour se trouver
                           le 3 février au service de
                              la primatiale
                        \bigskip



                           Dom de Lisle abbé de S. Leopold
                           mort le
                        \bigskip


                         Je prends possession avec le S.
                              Poirson
                           notaire des terres
                           que j'ai achetées de la veuve Gauvin à Heillecourt
                        \bigskip


                     \end{diary}

                     \begin{diary}{28 Janvier 1766}{}

                         Il y a eu aujourd'hui matin à
                              la Cour
                              souveraine audience extraordinaire
                           sur l'affaire des fabriques de
                              Nancy.
                           Au lieu d'enregistrer l'arrêt du Conseil
                           et les lettres patentes on
                           a rendu un
                           grand arrêt. \bigskip


                     \end{diary}

                     \begin{diary}{29 Janvier 1766}{}

                         Assemblée particulière de l'Académie où étaient Messieurs
                           de Sivry, le P.
                              Leslie, Coster,
                              abbé
                              de Tervenus, Bagard, Durival
                           l'aîné. On
                           y a lu des vers de M. de Sozzi
                           au roi sur le
                           tort que le long deuil fait aux ouvriers de Lyon.
                           On a agité si l'Académie
                           demanderait un
                           rang à la primatiale,
                           le jour du service de
                           M. le Dauphin. \bigskip



                           L'hôtel de ville de Lunéville fait célébrer
                           dans l’église
                              paroissiale un service pour M.
                              le Dauphin. \bigskip


                         Le baromètre au plus haut possible.
                           3 degrés de froid
                           le
                           matin. \bigskip


                         La livre de pain augmentée d'un denier
                              \up{1}/\textsubscript{2}. \bigskip


                     \end{diary}

                     \begin{diary}{31 Janvier 1766}{}

                         Un huissier de la Cour souveraine en robe vient
                           m'avertir pour l'hôtel de ville
                           d'assister lundi
                           au service de la
                              primatiale à 10 h. du matin. \bigskip



                           Mon frère Jean, qui a été
                           fort mal des
                           hémorroïdes et saigné 5 fois en un jour
                           m'écrit d'hier qu'il est mieux. \bigskip


                     \end{diary}
                  \chapter*{Février 1766}\addcontentsline{toc}{chapter}{Février 1766}



                     \begin{diary}{01 Février 1766}{}


                           Le roi de Pologne arrive à
                              la Malgrange
                           vers quatre heures après midi. \bigskip


                     \end{diary}

                     \begin{diary}{03 Février 1766}{}

                         La cérémonie d'aujourd'hui ne s'est
                           pas
                           passée aussi tranquillement qu'on aurait pu
                           le croire. Le service pour M.
                              le Dauphin
                           devait commencer à 10 h. précises. Un
                           détachement du régiment du
                              roi était au portail
                           des gardes du corps de Sa Majesté Polonaise dans l’église
                              primatiale. Le
                              bailliage, la maitrise
                              des eaux et forêts
                           l'hôtel de ville, et la justice
                           consulaire étaient
                           déjà placés ; la Chambre des
                              Comptes est
                           venue peu après prendre place dans le
                           chœur, et on n'attendait plus que la Cour
                              souveraine mais il s'était élevé une difficulté.
                           M. le cardinal de Choiseul
                           officiant a prétendu
                           que le P. Coster
                           jésuite qui devait prononcer
                           l'oraison funébre lui adresserait la parole,
                           sinon qu'il monterait à l'autel et continuerait.
                           La Cour voulait que ce fut à elle que l'orateur
                           s'adressat, sinon il serait décrété. Cela
                           a retardé jusqu'à 10 h \up{1}/\textsubscript{2}. Enfin la Cour souveraine
                           s'est présentée à la porte principale. Les
                           gardes du corps ont refusé de laisser passer
                           la maréchaussée qui escortait la Cour ; et la Cour n'a
                           voulu entrer qu'avec la maréchaussée. Elle s'est
                           retirée, les stales qui lui étaient préparées
                           sont restées vides. M. le
                              chancelier et M. l'intendant
                           étaient dans une tribune
                           pratiquée dans un pilier. Les gens de la
                           Cour distribués à différents endroits. La
                              musique du roi dans une tribune faite
                           exprès sous l'orgue. Le service a commencé.
                           Le P. Coster a prononcé l'oraison
                           funèbre, qui
                           a duré plus de cinq quarts d'heure. Après la
                           messe les obsèques, où ont paru 4 abbés
                           crossés et mitrés. On n'est sorti qu'après
                           une heure un quart. Le catafalque était
                           une pyramide fort haute et fort large
                           qui empêchait aux \up{3}/\textsubscript{4}
                           de voir. \bigskip


                         Il n'y avait là que le fauteuil du
                           roi,
                           Sa Majesté Polonaise était restée presque seule à
                           la Malgrange. \bigskip


                         Le soir Sa Majesté l'ayant bien voulu on a
                           représenté \emph{le Père de famille}. \bigskip


                         L'oraison funèbre avait été
                           lue au roi à
                           Lunéville par M. Coster l'académicien frère
                           de l'auteur à l'endroit d'un très bel éloge
                           de Sa Majesté Polonaise Elle dit au S.
                              Coster : \og dit à
                              ton frère d'ôter cela et de le réserver pour
                              mon oraison funèbre. \fg{}
                        \bigskip


                     \end{diary}

                     \begin{diary}{04 Février 1766}{}


                           Sa Majesté est repartie pour Lunéville après
                           son dîner. Elle a hier querellé M. le
                              premier
                              président de la Cour et n'a pas voulu voir les deux conseillers qui
                           l'accompagnaient.
                           Elle m'a dit qu'elle sait bon gré aux
                           autres corps de s'être trouvés à la cérémonie. \bigskip


                     \end{diary}

                     \begin{diary}{05 Février 1766}{}

                         À l'assemblée de l'hôtel de ville aujourd'hui
                           on a reçu Sigisbert
                              Poinsignon archer de ville
                           à la place de Poirson, mort. \bigskip


                         Après midi est venue la fâcheuse
                           nouvelle
                           que le roi de Pologne à 7 h. du matin à Lunéville
                           avait eu sa robe de chambre brulée sur lui \&\up{a}.
                           Ce qui a fait partir d'ici M.
                              l'intendant, \sout{madame
                                 l'intendante}, M. et madame de Baye \&\up{a}. \bigskip


                     \end{diary}

                     \begin{diary}{06 Février 1766}{}


                           Mon frère m'écrit d'hier sur cet accident
                           et je vois par une autre lettre que le roi s'étant
                           levé pour poser sa pipe sur la cheminée, sa
                           robe de chambre ayant flotté vers le feu il
                           y avait pris et avait bientôt monté jusqu'au
                           bonnet de nuit. Le roi
                           ayant
                           appelé ses
                           valets de chambre se sont jetés sur lui et
                           ont éteint le feu ; mais il a laissé des
                           marques en plusieurs endroits. Le
                              roi seul
                           n'a point été effrayé, il ne tarit pas en
                           bons mots sur son aventure, sa gaieté
                           n'a fait qu'augmenter. Il garde la chambre
                           et on y fait sa partie. \bigskip


                         Le froid a repris depuis deux jours le
                           vent étant revenu Nord-Est. Il y a
                           en ce matin 7
                              h.
                           5 degrés
                              \up{1}/\textsubscript{2}, quoiqu'il tombe un peu de neige. \bigskip


                         Je reçois une lettre de l'abbé Expilly
                           d'Avignon
                           26 janvier 1766. Il dit :
                           \og depuis près d'un mois et demi il ne cesse
                              de geler ici. Ces jours passés le régiment
                                 de royal-italien a traversé le
                                 Rhône sur
                              la glace, à sec, de Beaucaire à Tarascon,
                              avec armes et bagages \fg{}. \bigskip


                         Mort du père de Menoux, jésuite ci-devant
                           supérieur des missions royales de Lorraine.
                           Il s'était démis le
                                 30 septembre de la place de supérieur.
                           Il sera demain inhumé au
                              noviciat
                        \bigskip


                     \end{diary}

                     \begin{diary}{07 Février 1766}{}


                           M. l'intendant revient de
                              Lunéville. Le
                              roi continue à bien se porter, et à plaisanter
                           d'une aventure qui fait encore frémir, quand
                           on pense qu'il pouvait périr en une minute. \bigskip


                         Service pour le Dauphin aux Jésuites du collège. L'oraison funèbre par le P. l'Enfant. \bigskip


                     \end{diary}

                     \begin{diary}{08 Février 1766}{}

                         Ce matin à l'hôtel de ville, sur ce qu'on avait
                           dit des dispositions de M.
                              l'évêque de Toul, Messieurs
                           Breton et Chapuis fils lui ont été députés,
                           pour demander que l'on fit gras pendant
                           ce carême. Cela est accordé. \bigskip


                     \end{diary}

                     \begin{diary}{10 Février 1766}{}

                         Je vais à Lunéville. \bigskip


                     \end{diary}

                     \begin{diary}{11 Février 1766}{}

                         Le lendemain matin j'ai vu
                              le roi dans
                           sa chambre. Il a le bras gauche enveloppé.
                           Les croutes du visage commencent à se former.
                           Il est sans inquiétude, sans fièvre et dort bien. Ce que j'ai appris de son accident par ceux qui
                           s'y sont trouvées le rendent encore plus effrayant.
                           La guerison sera longue. Il y a eu à Lunéville
                           beaucoup d'autres accidents de brûlures :
                           entre autres celui de madame de La Millière, qui
                           n'est pas encore entièrement guérie.
                           On travaille à l'ordonnance pour la levée de
                           la milice en Lorraine, qui sera cependant
                           datée du 31 janvier. \bigskip


                     \end{diary}

                     \begin{diary}{11 Février 1766}{}

                         Je repars le Mardi gras pour Nancy. \bigskip


                         Le dégel commençait, et on ne
                           voyait
                           déjà plus de neige sur les terres labourées. \bigskip


                     \end{diary}

                     \begin{diary}{12 Février 1766}{}

                         Assemblée à l'hôtel de ville après midi,
                           où étaient Messieurs
                           de Riocourt, de Marcol,
                           Thibault, Mengin et Durival, pour
                           entendre les comptes que le S.
                              Bechet rendait
                           pour les années 1764 et 1765 de la
                              fondation des maladies épidémiques.
                           Il lui a été accordé
                           1300\up{\#.} de Lorraine
                           pour le passé, et délibéré que pour
                           l'avenir il lui serait fixé une somme
                           annuelle, dans la prochaine assemblée,
                           pour peines, ports de lettres, pertes
                           d'argent, droits de sacs \&\up{a}. \bigskip


                     \end{diary}



                     \begin{diary}{17 Février 1766}{}

                         La situation du roi de Pologne est toujours la
                           même, c'est-à dire beaucoup de douleur dans
                           les pansements, surtout de la main gauche,
                           de la fièvre, et c'est ce dernier article qui
                           inquiète parce qu'on en craint des accidents
                           fâcheux. Des tâches noires se sont manifestées
                           sur la peau ; le quinquina
                           les a fait
                           disparaître, mais on craint le retour. \bigskip



                           Le roi a fait ce matin quelques
                           signatures
                           de chancellerie. \bigskip


                     \end{diary}

                     \begin{diary}{18 Février 1766}{}


                           Fleury, arrivé aujourd'hui de
                              Lunéville,
                           a rapporté que la nuit avait été moins
                           tranquille que la précédente. Le
                              roi a souffert
                           et s'est fait mettre dans son fauteuil. \bigskip



                           L'hôtel de ville envoyé à Lunéville pour avoir
                           des nouvelles de Sa Majesté et va établir pour
                           cela une correspondance journalière. \bigskip


                         On donne à l'impression l'arrêt pour la
                           liquidation des dettes d’État de Lorraine. \bigskip


                     \end{diary}

                     \begin{diary}{19 Février 1766}{}

                         Mort de Paul-Louis Protin, conseiller en la
                              Cour souveraine. Il était tombé en létargie
                           quelque jours auparavant. \bigskip


                         Mon exprès est parti de Lunéville après
                           les pansements du matin. Les nouvelles qu'il apporte sont très satisfaisantes. Le
                              roi a
                           eu une nuit très tranquille, les escarres
                           tombent. Il conserve sa sérénité et sa
                           gaieté. J'ai dépêché un autre exprès
                           pour avoir des nouvelles demain. \bigskip



                           Madame l'intendante est partie ce matin
                           pour Paris, où elle fera ses
                           couches. M.
                              l'intendant est allé à Lunéville. \bigskip


                     \end{diary}

                     \begin{diary}{20 Février 1766}{}

                         Après plusieurs jours de dégel, le
                           froid
                           a repris depuis hier soir. Il y avait
                           ce matin 3 degrés, à 7 h. par un
                           vent
                           Nord-Ouest
                        \bigskip


                        \begin{quote}\begin{flushright}Bulletin de Lunéville
                              le 20\end{flushright}
                              le roy
                                 eut hier à 10 h. du soir un
                                 frisson
                                 de quelques minutes, mais qui n'a point
                                 eu de suites, ce qui donne à penser qu'il ne
                              provient que de refroidissement sans principe
                              de fièvre. Les plaies ont été trouvées au
                              pansement de ce matin, encore en meilleur
                              état que dans ceux d'hier et donnent de
                              bonnes espérances pour les suivants, d'autant
                              que la fièvre de suppuration est fort
                              diminuée : on s'occupe avec soin des
                              moyens d'empêcher qu'il ne s'y mêle une
                              fièvre accidentelle qui pourrait avoir de
                              très fâcheuse suites ; mais il n'y a point
                              jusqu'à présent de symptômes qui fondent
                              cette appréhension. \bigskip

        \end{quote}
                         Mort de M.
                              Protin
                           conseiller à la Cour souveraine sera inhumé aux Minimes
                        \bigskip


                     \end{diary}


                     \begin{diary}{21 Février 1766}{}


                           Mes
                              frères m'ont écrit d'hier
                              soir par la
                           poste. L'affaissement est très sensible, la
                           fièvre continue, et plus forte la nuit que le
                           jour. Enfin l'état du malade (le roi de Pologne)
                           n'est rien moins que satisfaisant. M. le
                              chancelier
                           est dans la douleur. \bigskip


                        \begin{quote}\begin{flushright}Bulletin du 21, de
                                 Lunéville.\end{flushright}
                              Le prince, dont
                              l'affaissement pendant la journée
                              d'hier avait donné de l'inquiétude, se trouva beaucoup
                              mieux le soir, et tint son assemblée ordinaire, avec
                              la même gaieté qu'avant l'accident. \bigskip

         Le présage qu'on en tira pour une
                              nuit plus
                              tranquille que la précédente s'est confirmé en
                              partie ; le
                                    roi a passablement dormi depuis
                                    minuit jusqu'à six heures. \bigskip


                              Le pansement ne s'est fait qu'à huit, les chairs
                                 reprennent dans les parties découvertes ; on a levé
                                 de nouveaux escarre dans quelques autres ; ces
                                 derniers bien plus profonds qu'on ne l'avait cru,
                                 mais bien détachés malgré l'épaisseur. Beaucoup
                              des parties tenaces sont disposées à se détacher
                              aux pansements prochains. Dans celui de
                              ce matin les plaies ont été trouvées et
                              laissées dans le meilleurs état possible, et
                              sauf les accidents nous ne sommes pas
                              sans espérance. \bigskip

        \end{quote}

                           Mon frère ne m'a rien ajouté à ce
                           bulletin. \bigskip


                     \end{diary}


                     \begin{diary}{22 Février 1766}{}

                         Aujourd'hui samedi il a passé à une
                           heure
                           un quart après midi deux courriers de
                           Lunéville qui
                           partaient à Versailles la
                           nouvelle
                           que le roi de Pologne était très mal. À 4 h \up{1}/\textsubscript{2}
                           j'ai reçu de mon frère le billet
                           suivant \bigskip


                        \begin{quote}\begin{flushright}Lunéville
                              22 février 9 h. du matin\end{flushright}
                              Je vous marquais hier soir l'état du roi. Je
                              n'ai ce matin rien de consolant à vous
                              annoncer ; le malade respire, mais sa
                              situation ne laisse que peu d'esperance, et
                              peut être bientôt ..... Dieu veuille que je me trompe. \bigskip

        \end{quote}
                         D'autres nouvelles fâcheuses et encore
                           plus
                           positives sont arrivées. J'ai dépêché un
                           courrier pour en avoir de certaines. M.
                              le cardinal de Choiseul avait écrit pour
                           faire descendre la châsse de S.
                           Sigisbert.
                        \bigskip



                           L'évêque de Toul a passé à 7 h.
                           allant
                           à Lunéville. Il a ordonné
                           de sonner dans
                           toutes les églises pour les quarante heures
                           ce qui a fait croire que le
                              roi était mort,
                           et mis tout le monde en alarmes. Toutes
                           les personnes considérables sont venues me
                           demander des nouvelles jusqu'à minuit. \bigskip


                         J'ai reçu à onze heures et demie, par le
                           retour de mon courrier, le billet suivant. \bigskip


                        \begin{quote}\begin{flushright}Lunéville
                              22 février 8 h \up{1}/\textsubscript{2} du
                                 soir.\end{flushright}
                              J'allais vous écrire au moment où votre
                              courrier est arrivé. Notre maître
                                 respire encore. Après avoir
                                 reçu l'extrême
                                 onction vers 10 h. du
                                    matin, sans connaissance
                                 ni mouvement, il y a eu quelques instants
                                 lucides. À midi une moiteur salutaire.
                                 Elle
                                 s'est soutenue et a rétabli la suppuration.
                                 Quelques paroles sont sorties avec effort de la
                                 bouche du malade avant et après le pansement.
                              Ce soir la tête est plus libre, et je me hâte
                              de vous renvoyer votre courrier, en attendant
                              celui que je vous dépêcherai demain après
                              le pansement. Les courriers que vous avez
                              vu passer à Nancy, sont
                              effectivement partis d'ici. \bigskip

        \end{quote}

                           Mon frère m'ajoute séparément : \bigskip


                        \begin{quote}\begin{flushright}\end{flushright}
                              Voici ce que vous pouvez montrer sur l'état
                              du roi ; on n'espère
                              presque plus rien ; mais
                              enfin il vit encore et c'est beaucoup. On ne
                              pénètre plus dans la chambre du roi, excepté
                              les gens nécessaires et M. le
                                 chancelier qui
                              s'y renferme, peut être pour toute la nuit. \bigskip

         Je ne lui remettrai que demain la lettre
                              de M. Gallois et la vôtre.
                                 M. l'intendant
                              est arrivé ce soir de Neuviller. Je crains
                              bien de n'avoir pas demain de meilleures
                              nouvelles à vous donner. \bigskip

         Bonsoir. \bigskip

        \end{quote}
                     \end{diary}

                     \begin{diary}{23 Février 1766}{}

                         On a reçu ce matin des nouvelles de
                              minuit,
                           par Messieurs
                           \sout{Perrin} Charvet et d'Ubexi ; et de 3 h
                              du matin par le courrier. Les choses étaient au même état. À
                           midi M. le premier président
                              Du Rouvrois a reçu un courrier qui
                           annonçait qu'au pansement du matin
                           les plaies étaient gangrenées et qu'il n'y
                           avait plus d'espérance. À une
                              heure
                              et demie après midi, j'ai reçu ceci
                           de mon frère commissaire. \bigskip


                        \begin{quote}\begin{flushright}Le 23 8 h. du
                                 matin\end{flushright}
                              Il n'y a plus d'espérance de conserver
                              notre bon roi ; il n'a
                              plus qu'un souffle
                              de vie : vous recevrez encore de mes
                              nouvelles aujourd'hui. \bigskip

        \end{quote}
                         Il ajoute séparément ; \og
                              Les médecins ne donnent pas quatre heures
                              de vie au malheureux
                                 prince. M. le
                                 chancelier est depuis 7 heures avec M.
                                 l'intendant et les gens de service, dans
                              la chambre du mourant.  \fg{}
                        \bigskip


                         Je reçois ceci à 5 h. du soir \bigskip


                        \begin{quote}\begin{flushright}
                              Lunéville
                              23 février 11 h. du matin.\end{flushright}
                              Je n'ai rien de plus à vous dire sur l'état du roi,
                              que ce que je vous en ai marqué à 8 h. Sa Majesté
                              a donné quelques signes de connaissance
                              mais sa situation est absolument désespérée ;
                              Je ne vous parle pas de l'accablement de la
                              Cour. M. Chapuis va faire partir ceci
                              par le messager qui l'a suivi ; je lui ai conseillé de ne partir lui même
                              que ce soir.
                              Nous sommes tous dans la douleur. \bigskip

        \end{quote}

                           Mon frère ajoute séparément. \og
                              M. le chancelier, M. l'intendant et 3 ou 4
                              personnes ne quittent plus la chambre du roi.
                              vous verrez ce soir M. Chapuis.
                              Les plaies
                              sont séches et noires ; on reveille le
                              malade par de violents cordiaux.  \fg{}
                        \bigskip


                         À 6
                              h \up{1}/\textsubscript{4} du soir il
                           passe à Nancy un courrier
                           de M. le comte de Lucé, allant
                           porter à Versailles
                           la nouvelle de la mort du bon roi
                              Stanislas. \bigskip



                           M. le chevalier de Ludres arrive à 7 h. \up{1}/\textsubscript{2} qui
                           nous confirme que le roi
                           est mort à 4 h. et
                              quelques minutes. La désolation dans Lunéville. \bigskip


                        \begin{quote}\begin{flushright}Lunéville
                              23 février 4 h\up{1}/\textsubscript{4}\end{flushright}
                              M. Chapuis va vous
                              annoncer la perte que
                              nous venons de faire du meilleur des rois. Je
                              vous attends demain dans la matinée.
                                  Il expira
                                    à 4 h. et
                                       quelques minutes
                                       du soir
                                 .
                           \bigskip

        \end{quote}

                           M. Chapuis a ajouté de bouche
                           beaucoup de
                           détails, et m'a dit de la part de M.
                              l'intendant
                           de rester. Madame de Bouflers s'était présentée
                           pour voir le roi quelques
                           heures avant la
                           mort, elle n'a pu entrer. Un envoyé du roi
                              de Pologne Poniatowski a passé dans le même
                           temps à Lunéville. Il a
                           vu le roi Stanislas
                           mourant, qui a cependant entendu ce qu'on
                           lui disait, il n'a pu articuler et a tendu la main à cet
                           ambassadeur. L'agonie a été
                           longue et douloureuse. On a apporté le scellé
                           sur les papiers du roi,
                              M. de Lucé \&\up{a}
                           y ont mis leurs cachets. M.
                              Alliot a ordonné
                           tout haut de retirer la bougie \&\up{a}. sous pretexte
                           de n'avoir pas d'argent. On n'entendait que
                           cris, clameurs et gémissements dans toute
                           la ville. M. le chancelier
                           n'a pas quitté le
                              roi jusqu'au dernier moment. \bigskip


                     \end{diary}

                     \begin{diary}{24 Février 1766}{}


                           La Cour souveraine était assemblée de grand matin,
                           sur ce qu'elle aurait à faire ; on fouillait les
                           registres et on ne résolvait rien. \bigskip


                         À dix heures et demie du matin M. le
                              chancelier, M.
                              l'intendant et M. le comte
                              de Lucé sont arrivés à Nancy. M. le chancelier
                           s'est arrêté devant le
                              palais et s'est fait
                           conduire au parquet du procureur
                              général.
                           Il était porteur d'un ordre du roi, daté
                           de Compiègne
                           le 31 juillet 1765, portant
                           que le cas arrivant (ce que Dieu veuille
                           éloigner) de la mort du roi de Pologne ses testament
                           et codicile déposés à la Cour
                              souveraine
                           lui seraient remis cachetés, l'intention
                           du roi était que
                           l'ouverture en soit faite
                           en sa présence \&\up{a}. La Cour s'est assemblée
                           en robe, on adresse des actes, le testament et le codicile ont été remis ; cette
                           opération
                           a duré jusqu'à une heure après midi. M.
                              le chancelier est allé dîner à l'intendance,
                           accompagné de M. Marcol
                           procureur général
                           M. le
                              premier président et M. de Vigneron
                           président sont
                           venus l'y voir. Après le dîner M. le chancelier
                           a écrit et s'en est retourné à Lunéville.
                           M. de Lucé est parti pour
                              Versailles
                           emportant le testament et le codicile. M.
                              l'intendant est allé tout de suite à la
                              Malgrange où je l'ai accompagné, et y
                           a apposé les scellés. Il avait un ordre pour
                           cela, aussi bien que M. le
                              chancelier, dès
                           le 15 avril 1762. \bigskip


                         C'est en vertu de cela que M. le chancelier a apposé
                           les scellés sitôt la mort sur la cassette et
                           les papiers du roi,
                              la chancellerie aulique,
                           la caisse du trésorier.
                           Il a fait faire cette
                           opération à Chanteheux, Jolivet et autres
                           endroits par M. Viot.
                        \bigskip



                           La Cour souveraine a envoyé un huissier en
                           France pour avoir de nouveaux
                           sceaux.
                           Il faudra aussi un autre timbre. L'embarras
                           est grand. Un réquisitoire du procureur général était
                           tout prêt pour le deuil \&\up{a}. Le public
                           s'attend à de grand changements. Messieurs
                           Gallois, Feriet,
                              de Serre et de Brichambeau
                           conseillers d’État étaient partis le matin pour
                           Lunéville. Ils ont
                           rencontré M. le
                              chancelier sur la route, et sont revenus
                           sur leurs pas. \bigskip



                           Le prince de Beauvau a passé à 6 h
                              \up{1}/\textsubscript{2} du
                           soir allant à Lunéville.
                        \bigskip


                     \end{diary}

                     \begin{diary}{25 Février 1766}{}


                           M. l'intendant est allé à
                              Commercy
                           ce matin, pour apposer les scellés au château.
                           Il est revenu à 8 h \up{1}/\textsubscript{2} du
                           soir. \bigskip


                     \end{diary}

                     \begin{diary}{26 Février 1766}{}

                         À l'assemblée de l'hôtel de ville j'ai écrit la
                           mention à conserver dans les registres
                           de la mort déplorable de Stanislas I surnommé
                              le bienfaisant, dernier duc de Lorraine. \bigskip


                         Assemblée de l'Académie où étaient Messieurs
                           Du Rouvrois
                           directeur, de Sivry
                           sous-directeur
                           Thibault,
                           de Tervenus, Bagard, Cupers, de
                           Nicéville,
                           Gautier, P. Leslie, P. Husson,
                              Coster, Liebault,
                           Durival l'aîné. \bigskip


                         Délibéré que M. de Solignac serait chargé
                           de faire l'éloge funèbre du
                              roi à prononcer
                           dans l'assemblée publique du mois de
                              mars.
                           Que M. l'abbé Clément,
                           académicien, ferait
                           l'oraison funèbre du même prince, pour être
                           prononcée lors du service que l'Académie
                           ferait célébrer dans l’église
                              des Cordeliers,
                           pour le repos de l'âme de Sa Majesté
                        \bigskip


                         Et que M. le directeur écrivait à M.
                              le duc de Choiseul,
                           ministre d’État, que l'Académie
                              de Nancy étant de fondation royale, le roi
                           s'était supplié de prendre le titre de protecteur ;
                           qu'on écrirait en même temps à M. le duc de
                              Fleury gouverneur de la province et académicien
                           né, pour y joindre ses bons offices auprès de
                           Sa Majesté et du ministre. \bigskip



                           M. l'intendant est parti cet après
                           midi
                           pour Lunéville.
                           Les Jésuites de Pont à mousson et de
                              Nancy, et sans
                           doute les autres qui se trouvent en Lorraine,
                           se dépêchent de vendre et de faire argent de tout. \bigskip


                     \end{diary}

                     \begin{diary}{27 Février 1766}{}


                           Messieurs
                           de Levy et d'Amerague passent allant
                           à Lunéville. \bigskip



                           M. l'intendant est arrivé le
                           soir, apportant
                           l'édit adressé à la Cour souveraine et à la Chambre des Comptes.
                           pour continuer leur fonctions, et se servir en
                           attendant du sceau du roi
                              Stanislas. Avec des
                           lettres de cachet pour se
                           trouver aux obsèques. \bigskip


                     \end{diary}
                     \begin{diary}{28 Février 1766}{}

                         Ces édits ont été enrégistrés ce
                           matin, et M.
                              l'intendant est reparti pour Lunéville, avec les
                           réponses des 1.\up{rs} présidents et procureurs généraux.
                           M. le comte de Stainville
                           arrive à 6 h. du
                           soir, et dépêche un estafette à Lunéville, pour
                           avoir ses nouvelles lettres de commandant, qu'un
                           courrier de M. le Duc de
                              Choiseul y avait
                           laissé. \bigskip


                     \end{diary}
                  \chapter*{Mars 1766}\addcontentsline{toc}{chapter}{Mars 1766}


                     \begin{diary}{01 Mars 1766}{}

                         Dans l'assemblée de ce matin à
                              l'hôtel
                              de ville, délibéré que Messieurs
                           Breton, Guillon,
                           Richer et Rambois, en habit de cérémonie
                           et accompagnés de deux sergents de ville
                           en manteaux ou casaques,
                           iront demain
                           jeter l'eau bénite à Lunéville. \bigskip


                         J'ai eu l'après midi les préparatifs
                           de
                           l’église de
                              Bonsecours. Et M. Mique est
                           arrivé le soir pour tout préparer, avec
                           le mémoire de M. le cardinal de Choiseul,
                           sur l'ordre de la marche et de la
                           cérémonie à l’église. \bigskip



                           Mon frère écrit à M.
                              Coster, de la part
                           de son éminence pour se charger de composer
                           la relation de toute la pompe funèbre. \bigskip


                     \end{diary}

                     \begin{diary}{02 Mars 1766}{}

                         La députation de l'hôtel de
                              ville va à Lunéville,
                           et à 3 h après midi jette l'eau bénite dans
                           la chapelle ardente. la Cour
                              souveraine
                           y avait été la veille. La Chambre des Comptes
                           le
                              baillage et la
                              maitrise
                           le 2. \bigskip



                           M. le Comte de Stainville va à
                              Lunéville
                           l'après midi. \bigskip


                     \end{diary}

                     \begin{diary}{03 Mars 1766}{}

                         À six heures du soir le corps du
                           feu roi de Pologne, duc de
                           Lorraine et
                           de Bar, est parti de Lunéville, avec
                           un nombreux cortège, au milieu des larmes et des gémissements des bourgeois, dont plus
                           de 4000 ont suivi jusqu'à Léomont.
                           après le départ M. le
                              chancelier a mis
                           le scellé sur la chambre du roi. Il a ensuite
                           rejoint le convoi, qui est arrivé à Bonsecours
                           à minuit et demi. Le corps a été aussitôt
                           descendu dans le caveau où est déjà inhumée
                           feue reine de Pologne.
                           Cette cérémonie
                           a duré environ \up{3}/\textsubscript{4}
                           d'heure ; à Nancy dès
                           les 6 h. du soir, moment du départ, on tirait
                           un coup de canon des remparts de quart
                           d'heure en quart d'heure ; les décharges ont
                           redoublé à la descente du corps dans le tombeau.
                           Et le régiment du roi rangé dans le faubourgs
                           a fait des salves de mousqueterie. Le
                              régiment des gardes lorraines avait accompagné
                           le convoi. \bigskip


                     \end{diary}

                     \begin{diary}{04 Mars 1766}{}

                         Aujourd'hui s'est fait le service dans
                           l’église des Minimes de
                              Bonsecours. Il
                           devait commencer à 10 h. du matin, mais
                           ce n'a été qu'à onze. Le cardinal de Choiseul
                           officiait, avec 4 abbés crossés et mîttrés ;
                           tout s'y est passé avec beaucoup de dignité.
                           et de décence. la Cour souveraine
                           la Chambre des Comptes
                           le
                              baillage, la maitrise
                              des Eaux et Forêts et l'hôtel
                              de ville, la justice consulaire y étaient.
                           et au dehors un peuple immense. Tout
                           était fini à midi et un quart. \bigskip


                         Mort de M.
                              de Beauchamps, ancien lieutenant
                           de roi. Sera inhumé dans l’église de
                              S. Roch. \bigskip



                         Presque tout ce qui était venu de
                           Lunéville y est retourné immédiatement.
                           M. le chancelier et M. l'intendant y
                           sont allés. M. le chancelier
                           part demain
                           pour Versailles et emmène
                           mon jeune frère.
                           On dit que le testament et le codicile du
                           feu roi, qui avaient été
                           portés à
                           Versailles, par M. de Lucé, ont été renvoyés
                           à la Cour souveraine. \bigskip



                           M. Alliot est parti pour Versailles
                        \bigskip


                     \end{diary}
                     \begin{diary}{05 Mars 1766}{}

                         Le feu a pris vers trois heures du
                           matin
                           à une poutre dans le cabinet de toilette
                           d'une petite chambre à côté de l'appartement
                           de madame la comtesse de Stainville, pendant
                           que Mad. la duchesse de
                              Grammont
                           y était logée. J'y suis allé en robe de
                           chambre et j'y ai trouvé M. Mique
                           avec des ouvriers. Le dommage a eté
                           peu considérable, on n'a point sonné
                           le beffroi, et M. de
                              Stainville n'en a
                           pas été éveillé. \bigskip



                           Les minimes de Bonsecours
                           célébrent
                           aujourdhui un service pour le feu
                              roi de Pologne. \bigskip


                         Le testament du roi de Pologne a été lu et
                           homologué ce matin à la Cour
                              souveraine.
                           M. l'intendant est revenu de
                              Lunéville.
                           et M. le chancelier a passé
                           allant à
                           Versailles où mon jeune frère
                           l'accompagne. \bigskip


                     \end{diary}


                     \begin{diary}{Encart}{}\begin{Large}PRIÉRES\end{Large}\bigskip

                         A Dieu en triste complainte de douleur, chantons la
                              lentement,
                              d’un ton lugubre en pleures, de la perte que nous faisons de
                              notre bon souverain, qui
                              toujours sera regretté des fidèles
                              lorrains, sur l'air : je l'ai perdu. \bigskip

        \begin{verse}Venez vous en fidèles Lorrains,\\Soyons tous en tristesses ;\\De voir que notre
                                 souverain,\\Par sa mort nous délaisse :\\Une maladie de dix-neuf jours,\\Souffrant dans l’espérance ;\\Que Dieu dans l'éternel séjour,\\fera sa récompence.\\Avant mourir fit ses adieux,\\Au souverain pontif,\\Et faire pour lui prier à Dieu,\\Comme curé primitif,\\Etant très bien persuadé,\\Que ses saintes Prières,\\Fléchiront la divinité\\Pour son âme en lumière.\\Fit ses
                                 adieux à ses enfans,\\Son aimé frere \&
                                 gendre,\\A son épouse
                                 pareillement,\\Sa fille reine de
                                    France,\\De vouloir prier Dieu pour lui,\\Son ame la recevoir,\\Avec les saints en Paradis,\\Dans sa celéste gloire.\\\end{verse}
        \bigskip

        \begin{verse}Fit ses adieux pareillement,\\Aux Rois régnant sur terre,\\A l'Empereur
                                 premièrement,\\De vouloir Dieu prier,\\Recevoir son ame dans les cieux,\\Y chanter ses louanges,\\D'un ton le plus melodieux,\\En joyes avec les Anges.\\\end{verse}
        \bigskip

        \begin{verse}Faut espérer qu'il est heureux,\\Ayant été sur terre,\\Un roi dévot bien
                                 vertueux,\\Très-souvent en priére,\\A la sainte Vierge mére de Dieu,\\M’étant sa confiance,\\Sa fidéle avocate aux cieux,\\Ses prières elle présente.\\\end{verse}
        \bigskip

        \begin{verse}Présente de même à Dieu aussi,\\Qu’étant roi de Pologne,\\Tous les maux qu’il y a subit :\\En sa propre personne :\\L'armée formidable des Russiens,\\Faisoit tous ses efforts,\\de perdre ce pieux souverain,\\Et de le mettre à mort.\\\end{verse}
        \bigskip

        \begin{verse}La confiance qu'il avoit,\\A Dieu \& à la Vierge,\\Lui inspire le sécret,\\De vitte s'en évader,\\Avec aisance se travesti,\\A passer la vistule\\Au travers de ses ennemis,\\Qui ne le reconnure.\\\end{verse}
        \bigskip

        \begin{verse}La confiance qu'il avoit,\\A Dieu \& à la Vierge,\\Lui inspire le sécret,\\De vitte s'en évader,\\Avec aisance se travesti,\\A passer la vistule\\Au travers de ses ennemis,\\Qui ne le reconnure.\\\end{verse}
        \bigskip

        \begin{verse}Son règne de paix des plus heureux\\Nous l'aurons en mémoire,\\A travailler toujours pour Dieu,\\De fonder à sa gloire,\\Premièrement des missions,\\Des écoles dans les villes,\\Y enseigner les orphelins,\\Par sa bonté divine.\\\end{verse}
        \bigskip

        \begin{verse}Enfin l’est mort trop-tôt ainsi\\En Février le vingt-trois,\\De l'an 1766 :\\Nous n'avons plus d'espoir,\\Qu’à LOUIS XV roi très \\chré-
                                    tien, Des états de
                                    la France,\\Nous sommes à lui fidèle lorrains\\Par la sainte Providence.\\\end{verse}
        \bigskip

        \begin{verse}On fut huit jours à préparer,\\Cérémonie funèbre,\\De notre roi le bien
                                 aimé,\\Tous les jours faire priéres,\\Par des religieux bien dévots,\\À célébrer des messes,\\Pour son ame, qu'il soit en repos,\\Dans la gloire céleste.\\\end{verse}
        \bigskip

        \begin{verse}Le neuviéme jour tout préparé,\\On en fit le cortège,\\De Lunéville fut
                                 transporté,\\Aux pieds de la sainte Vierge,\\A Bonsecours près de
                                    Nancy,\\Avec sa chere compagne,\\Notre bonne reine est
                                 aussi,\\Prions Dieu pour leurs âmes.\\\end{verse}
        \bigskip

        \begin{verse}Ce cortège fut accompagné,\\De toute la noblesse,\\Tout un chacun y a pleuré,\\Avec grande tristesse,\\Seigneurs de la Cour de Nancy,\\Messieurs du bailliage,\\Celui de Lunéville
                                 aussi,\\Faire ce triste voyage.\\\end{verse}
        \bigskip

        \begin{verse}Les gentils-hommes, Sages \& \\Cadets,\\Messieurs les gardes du corps,\\Tous en flambeaux qui éclairoient,\\De chaque côté du corps,\\Les Suisses, héducs \& valets de
                                 pieds,\\En mains grande lumières,\\Et aussi tous les palferniers,\\Tant devant que derrière\\\end{verse}
        \bigskip

        \begin{verse}Le grand carosse fait expret,\\De drap noir enfublée,\\Au dessus une grande croix,\\De blan satin vermeille,\\Tous les chevaux en deuil aussi,\\Ah ! quel triste spéctacle,\\Chacun y faisoit de grands cris,\\De regrets déplorables.\\\end{verse}
        \bigskip

        \begin{verse}Le grand Héros marchoit devant\\Du carosse à cheval,\\Ayant en main certainement,\\Le beau bâton royal,\\A ce grand deuil chacun pleuroit,\\Même les gens des villages,\\Que sur la route ils abordoient,\\Tout étoit pitoyable.\\\end{verse}
        \bigskip

        \begin{verse}Religieux, carmes \& capucins,\\En furent de ce cortège,\\Ayant chacun des cierges en main,\\Hélas ! quel détresse,\\Les corps des congrégations,\\D'hommes, de garçons \& filles,\\Avec grande dévotions,\\Y furent hors de la ville.\\\end{verse}
        \bigskip

        \begin{verse}Soixante pauvres y furent aussi,\\Avec torches allumées,\\En robe noir sans contredit,\\Grossissoit l'essemblée,\\C’est pourquoi on les renvoya,\\N'y étant nécessaire,\\Que malheur ne leurs arrivas,\\Fini fut leurs priéres.\\\end{verse}
        \bigskip

        \begin{verse}Les officiers du régiment\\Des gardes de Lorraine,\\Avec un gros détachement,\\Du matin cette journée,\\Furent devant attendre à Bonsecours\\Recevoir le cortége,\\Qui fut mardi avant le jour,\\De mars le quatriéme.\\\end{verse}
        \bigskip

        \begin{verse}Le corp du roi fut déposé,\\Au milieu de l’église,\\Sur un mosolé préparé,\\De suite fait le service,\\Par prélats \& religieux,\\Qui chantérent ses obséques,\\Que son âme soit dedans les cieux,\\Jérusalem céleste.\\\end{verse}
        \bigskip

        \begin{verse}Dans le tombeau fut inhumé,\\A côté de la reine,\\A Dieu pour eux devons prier,\\Par toute la Lorraine,\\Avec humilité de cœur,\\Demander les suffrages\\À la Vierge mere du Sauveur,\\D'être leurs avantages.\\\end{verse}
        \bigskip

        \begin{verse}Ainsi soit-il.\\\end{verse}
        \bigskip

         Autres Complainte en prières à Dieu, pour STANISLAS Roi très-
                              vertueux : chantons la lentement d’un ton lugubre à la gloire,
                              Sur l'Air :
                              Pauvre mortél ou est votre mémoire. \bigskip

        \begin{verse}Venez vous en à moi Lorrains\\fidèles,\\Venez vous en pleurer avec moi,\\Et prier Dieu humblement avec zéle,\\Sur le tombeau de notre deffunt
                                    roi,\\Qui en ce monde toujours à Dieu\\fidèle, bis.\\A pratiquer les préceptes de sa loi.\\\end{verse}
        \bigskip

        \begin{verse}Pieux, dévot à la sainte Vierge,\\Qui en sa vie, l’a souvent protégé,\\Dedans le ciel sa fidéle concierge,\\Qu'en récompense de sa fidélité,\\Présente son ame allumée comme\\un cierge, bis\\Aux trois perfonnes de la Ste Trinité\\\end{verse}
        \bigskip

        \begin{verse}Notre grand Dieu tout rempli\\de clémence,\\Exaucera nos priéres \& nos pleurs,\\Pour notre roi qu'à mis sa
                                 confiance,\\Toujours en lui avec humble ferveur,\\Et à la Vierge avocate clémente, bis.\\Près de son fils Jesus notre Sauveur.\\\end{verse}
        \bigskip

        \begin{verse}Hélas ! mon Dieu, nous faisons\\grande perte,\\d'un roi si bon si doux
                                 \& bienfaisant,\\Envoyez-nous un roi qui nous\\gou-
                                    verne,\\Comme lui en paix toujours bien-
                                 sagement,\\Bien
                                 vertueux \& d'un long heureux\\régne, bis.\\Qu’en ses états l’on bannisse les\\mé-
                                    chants.\\\end{verse}
        \bigskip

        \begin{verse}Pour quant à moi j'aurai toujours\\mémoir,\\De Stanislas notre roi bon
                                 chrétien,\\A prier Dieu que son ame soit en\\gloire,\\Dedans les cieux, avec ses plus\\grands Saints,\\Hélas ! Mon Dieu, s'est là tout\\mon espoir, bis.\\Me confiant à votre amour divin.\\Ainsi soit-il.\\\end{verse}
        \bigskip

        \begin{verse}Requies cat in pace.\\\end{verse}
        \bigskip

        \begin{Large} Complainte nouvelle
                              de monseigneur le Dauphin.
                           \end{Large}\bigskip

                         Mort à Versailles
                              le 17 Décembre 1765, Sur l'air : Catin
                              petite brune. \bigskip

        \begin{verse}Divine Providence,\\Qui décidez mon sort,\\Mon cœur en défaillance,\\N'attend plus que la mort,\\Roi cher papa,\\Ne vous alarmez pas,\\C’est le vouloir,\\Et de Dieu le pouvoir.\\\end{verse}
        \bigskip

        \begin{verse}Sa Majesté Louis XV.\\Perle de ma couronne,\\Dont l'éclat ravissant ;\\Seule charme ma personne,\\Digne issu de mon sang,\\Mon cher enfant,\\S'en retourne au néant,\\Quelle douleur,\\S'empare de mon coeur.\\\end{verse}
        \bigskip

        \begin{verse}Monseigneur le Dauphin.\\Reine, ma très chère
                                    mere,\\Ne versez pas des pleurs,\\Ce passage il faut faire,\\C'est l’ordre du Seigneur,\\N’oubliez pas,\\Les pauvres à mon trépas,\\Soit jeunes ou vieux,\\Ils sont membres de Dieu.\\\end{verse}
        \bigskip

        \begin{verse}La Reine très chrétienne de
                                 France.\\Beau fruit de mes entrailles,\\Que j'ai porté neuf mois,\\Au Louvre de
                                    Versailles,\\Mon fils
                                 regardez-moi,\\Un doux baiser,\\Et pour me consoler,\\Mon cher enfant,\\Que j'aime tendrement.\\\end{verse}
        \bigskip

        \begin{verse}Madame la Dauphine.\\Adieu époux fidel,\\Adieu charmant Dauphin,\\Comme la tourterelle,\\Je gémirai sans fin,\\D'un cœur navré,\\De près je vous suivrai,\\Embrassez-moi,\\Pour la dernière fois.\\\end{verse}
        \bigskip

        \begin{verse}Monseigneur le Dauphin.\\Epouse cessez vos larmes,\\C’est un décret de Dieu,\\Qui veut placer mon âme,\\s'il lui plaît dans les cieux,\\Par sa bonté,\\Pour toute l'éternité,\\Les larmes aux yeux,\\Chère compagne adieu.\\\end{verse}
        \bigskip

        \begin{verse}Les Enfans de Monseigneur.\\Donnez-nous très-cher
                                    Pere,\\La bénédiction,\\Les deux genoux à terre,\\Nous vous en supplions,\\Je vous bénis,\\Au nom du Pere \& Fils,\\Et du Saint-Esprit,\\Que Dieu soit votre appui.\\\end{verse}
        \bigskip

        \begin{verse}A Louis Duc de Berry.\\Sire, je vous recommande,\\Louis Duc de Berry,\\Mon fils, je vous commande,\\De lui être soumis,\\Doux aux soldats,\\Chérissez les prélats,\\Et soutenez l’Etat,\\De votre cher papa,\\\end{verse}
        \bigskip

        \begin{verse}Les dames \& les Princesses, \\ver-
                                    sant un torrent de pleurs, justes aux\\combats Noblesse, et chevaliers\\d'honneur, le deuil au cœur, sont\\saisis de douleurs, gardes-du-\\corps, sont presque à demi mort.\\\end{verse}
        \bigskip

        \begin{verse}Les pauvres de Paris \&
                                    Versailles.\\Prince très charitable, aux \\veu-
                                    ves, aux orphelins, nous que\\vieil-
                                    lesse accable, nous prierons sans\\fin, pour vos bontés, \& \\libérali-
                                    tés : cauelle mort que tu nous fait\\de tort.\\\end{verse}
        \bigskip

        \begin{verse}Dans un morne silence, chacun\\pleure \& gémit, son corps mort est\\à Sens, son cœur à Saint Denis, \&\\l'ame aux cieux, avec les \\bienheu-
                                    reux, pour tout jamais, dans ce\\séjour de paix. FIN\\\end{verse}
        \bigskip

         Vu permis d’imprimer, à Lunéville ce 11 Mars
                                 1766. Signé, VIOT.
                           \bigskip

        \end{diary}



                     \begin{diary}{07 Mars 1766}{}

                         Assemblée du bureau de l'aumône, où étaient
                           Messieurs
                           de Riocourt, de Morey, abbé de
                              Tervenus,
                           abbé de Bressey, Dombâle, François, de
                              Maisonneuve et Durival. On a décidé sur
                           plusieurs requêtes, et délibéré de laisser à bail
                           à vie au curé de S.
                              Nicolas, à 124.\up{\#} de
                              Lorraine par année, le jardin
                              de la maison de force.
                        \bigskip



                           M. Noverre, arrivé de Stuttgart est venu
                           me voir avec une lettre de M.
                              Uriot
                        \bigskip


                     \end{diary}

                     \begin{diary}{08 Mars 1766}{}

                         Sur de mauvaises nouvelles de la santé
                           de la reine,
                              M. le cardinal de Choiseul
                           a ordonné des prières cet après midi
                           dans l’église
                              primatiale, ce qui a fort
                           alarmé tout le monde. \bigskip


                     \end{diary}

                     \begin{diary}{09 Mars 1766}{}

                         Aujourd'hui assemblée à l'hôtel de ville
                           pour la fondation des maladies
                              épidémiques,
                           Messieurs
                           de Riocourt, de Marcol, Mengin
                           et moi. Mais M. Du Rouvrois n'y ayant
                           pu venir, il n'a rien été fait, et l'assemblée
                           a été remise. \bigskip


                     \end{diary}

                     \begin{diary}{10 - 11 Mars 1766}{}


                           M. l'intendant arrive à Nancy, y reste
                           le 11 et va voir la pépinière, dont on
                           défonce le terrain ; le 12 il y est
                           encore ;
                           il partira le 13 pour Neuviller
                        \bigskip


                     \end{diary}

                     \begin{diary}{12 Mars 1766}{}


                           M. l'intendant a reçu des lettres
                           de M.
                              son père du 9, il
                           n'avait pas encore remis
                           les sceaux de Lorraine, ni
                           vu la reine
                           quoiqu'elle se porte mieux. Il doit y avoir
                           devant le roi une assemblée où
                           seront Messieurs
                           le duc de Choiseul, prince de Beauvau,
                           le contrôleur général, de La
                              Galaizière et
                           Alliot pour règler le sort des
                           gens du
                           feu roi de Pologne.
                        \bigskip


                         J'ai reçu aussi une lettre de mon
                              frère
                           du même jour 9. M. de La Galaizière dit
                           plaisamment que ce qui a été lu au
                           Parlement
                           le 3 est un plagiat de ce qu'il
                           a fait en Lorraine. \bigskip


                     \end{diary}

                     \begin{diary}{12 Mars 1766}{}

                         Aujourd'hui assemblée de l'Académie
                           où étaient Messieurs
                           Du Rouvrois directeur,
                           de Sivry
                           sous-directeur, de Solignac
                           secrétaire
                           de Tervenus, Cupers, André, Liebault,
                           P. Leslie, P. Husson, abbé
                              Gautier et
                           Durival l'aîné. On a proposé
                              M. de
                              Montal chanoine de Toul pour académicien
                           mais cela a souffert difficulté, parce qu'il
                           n'avait point demandé en régle, et qu'il
                           faut voir de quel mérite sont ses ouvrages
                           imprimés, entre autres l'\emph{Anti-réformateur}.
                           convenu aussi qu'on lirait les ouvrages de
                           M. Le Bas. Que l'abbé Crédo serait effacé de la classe des titulaires, et que M. Liebault
                           y serait mis. M. de Solignac
                           a promis
                           de travailler à l'éloge du roi.
                        \bigskip


                     \end{diary}
                     \begin{diary}{13 Mars 1766}{}


                           M. l'intendant repart pour
                              Neuviller. \bigskip


                         Demain service aux Carmes pour
                           Stanislas le bienfaisant. \bigskip


                     \end{diary}

                     \begin{diary}{15 Mars 1766}{}

                         À l'assemblée de l'hôtel de ville on a
                           arrêté à 120\up{\#.} de France le logement en
                           argent de l'abbé Marquet son bibliothécaire. \bigskip


                         Je reçois des lettres de M. de La Galaizière et
                           de mon jeune frère, datées de Versailles
                           le 12. Le
                              dimanche 9 au soir M. de La Galaizière
                           remet les sceaux de Lorraine
                           au roi.
                           Le 13 il a du y avoir une
                           assemblée devant
                           Sa Majesté sur les affaires de Lorraine.
                           Quoique la reine
                           continue d'aller de mieux
                           en mieux elle ne reçoit encore personne
                           et M. de La Galaizière ni M. Alliot
                           ne l'avaient
                           pas encore vu. \bigskip


                     \end{diary}

                     \begin{diary}{16 Mars 1766}{}


                           M. de Lucé arrive de Paris. Il ne savait
                           encore rien de l'arrangement fait pour la
                           maison du feu roi de Pologne. \bigskip


                     \end{diary}

                     \begin{diary}{17 Mars 1766}{}

                         On reprend les ouvrages de maçonnerie
                           au corps-royal des casernes Nancy. \bigskip


                     \end{diary}

                     \begin{diary}{17 Mars 1766}{}

                         Je reçois une lettre de mon frère
                           datée
                           de Versailles
                           le 14. Dans le travail de la
                           veille fait devant le roi,
                           on a reglé le sort
                           de la maison du roi de Pologne. La livrée conserve ses
                           gages. Il y aura peut-être des retranchements
                           sur le surplus. Le 15
                           M. de la Galaizière a du
                           retourner à Paris. Il doit
                           arriver en
                           Lorraine en même temps que M. Alliot pour
                           la levée des scellés et la délivrance des legs.
                           M. l'intendant après avoir
                           reçu ses lettres
                           est parti pour Lunéville,
                           où M. de Lucé
                           le dévance. \bigskip


                         Un huissier pour les Leonis, commence
                           à saisir chez
                           les Jésuites de Lorraine. \bigskip


                     \end{diary}

                     \begin{diary}{18 Mars 1766}{}


                           Le P. Charles, jésuite qui s'était chargé de
                           l'oraison funèbre du roi de Pologne est malade et
                           a remercié. \bigskip



                           M. le chancelier passe à 7 h \up{1}/\textsubscript{2} du soir
                           allant à Lunéville. \bigskip


                     \end{diary}

                     \begin{diary}{19 Mars 1766}{}

                         À l'assemblée de l'hôtel de ville délibéré
                           de se pourvoir pour la conservation des
                           droits de la ville sur le collège, l’église \&\up{a}.
                           Et à l'occasion de partie des 50000\up{\#.} de
                              France
                           prêtées au roi de Pologne et destinés
                           à la reconstruction du collège de Nancy
                           et à l'entretien de celui de
                              Bar. \bigskip


                     \end{diary}


                     \begin{diary}{19 Mars 1766}{}


                           La Cour souveraine a reçu aujourd'hui l'opposition
                           des Jésuites, permis d'assigner
                           les saisissants
                           à leur domicile élu à Nancy,
                           toutes choses
                           demeurant en état. \bigskip



                           M. de Stainville m'a montré une
                           lettre qui
                           donne au commandant général de
                              la province
                           la Malgrange pour maison de
                           campagne,
                           avec pouvoir de faire démolir telles parties
                           qu'il jugera à propos. \bigskip


                         On dit dans Nancy qu'on lui donne aussi
                           pour logement l'hôtel de
                              l'intendance.
                           M. Alliot passe l'après midi
                           allant à
                           Lunéville, et s'arrête chez
                           M. de Stainville. \bigskip


                         Les gages et appointements de la maison du
                           roi Stanislas sont conservés
                           en pensions.
                           Il y aura des retranchements sur les pensions
                           de la musique. Il y a apparence qu'on donnera
                           des pensions aux officiers des gardes du corps
                           et des cadets. Qu'on
                           renverra les gardes avec
                           leurs chevaux, uniformes, armes et bagages.
                           Et les cadets avec leur uniforme, armes et
                           500\up{\#.} de gratification. \bigskip


                     \end{diary}

                     \begin{diary}{20 Mars 1766}{}


                           M. le chancelier a commencé
                           aujoud'hui
                           l'éxécution testamentaire, en remettant à
                           M. le maréchal de Bercheni les chevaux et
                           voitures dont la plupart sont déjà vendus.
                           Il a aussi levé les scellés de l'appartement
                           du roi, de la cassette, de l'appartement de
                           la reine et de
                           quelques autres parties. Demain
                           on commencera la délivrance des legs en argent. \bigskip


                     \end{diary}

                     \begin{diary}{21 Mars 1766}{}


                           M. le comte de Stainville m'a appris ce
                           matin que le roi
                           accordait cent
                              mille livres
                           sur la succession du roi de Pologne pour la
                           construction des
                              casernes, et que le surplus
                           serait pris sur le fonds des fortifications :
                           M. de Stainville logera à
                              l'intendance
                           au bout de la carrière. Et l'intendant dans
                           son hôtel sur la place
                              royale. \bigskip



                           M. l'intendant lève les scellés à
                              la
                              Malgrange. Service aux Premontés pour
                           le roi Stanislas. \bigskip


                     \end{diary}

                     \begin{diary}{22 Mars 1766}{}


                           M. l'intendant à Commercy pour y
                           faire la même opération. \bigskip



                           L'hôtel de ville a adjugé
                           des bois de la
                              côte Sainte-Catherine et au dessus de Maréville.
                           Il y a été résolu un service solennel
                           pour le feu roi de Pologne dans l’église S.
                              Roch,
                           après la 15\up{e.} de Pâques, sans oraison
                           funèbre
                           M. l'évêque de Toul a promis
                           d'y venir officier,
                           et le vicaire de S. Roch
                           est allé convenir du jour. \bigskip


                         On m'écrit de Lunéville
                           le 22. : \og notre cour
                              se dépeuple petit à petit. Tout le monde se
                              plaint. Il n'y a encore rien de décidé
                              définitivement en attendant on délivre les legs \fg{}. \bigskip


                     \end{diary}


                     \begin{diary}{23 Mars 1766}{}

                         Il neige un peu pendant le jour, et il gèle
                           pendant la nuit. \bigskip


                     \end{diary}

                     \begin{diary}{24 Mars 1766}{}

                         Il neige très fort pendant le jour et
                           dégèle.
                           M. l'intendant retourne à
                              Lunéville. Il a
                           écrit ces jours-ci, de concert avec M. le comte
                              de Stainville, à M. le duc
                              de Choiseul sur les
                           appointements de gouverneur de
                              la Malgrange,
                           et offre en même temps de remettre la capitainerie
                              des chasses. \bigskip


                     \end{diary}

                     \begin{diary}{25 Mars 1766}{}

                         Il a beaucoup neigé la nuit
                           dernière,
                           et la neige continue de tomber par un vent
                           Sud-Ouest. Il y avait encore un dégré de
                              froid
                           à 7 h \up{1}/\textsubscript{2} du matin. À 3 h après midi
                           3 degrés
                              de chaud. \bigskip


                         On m'écrit d'hier de Lunéville. M. le
                              chancelier n'a encore rien reçu de positif
                           sur le sort de la maison du roi
                              de Pologne
                           En attendant on prépare la dissolution
                           pour le 1.\up{er}
                              avril. \bigskip


                     \end{diary}

                     \begin{diary}{26 Mars 1766}{}


                           M. le premier président m'envoie dire le
                           matin par son secrétaire, qu'on discontinuera
                           de sonner à la Quasimodo, et qu'on pourra
                           le lendemain reprendre les spectacles. \bigskip



                           L'abbé Sallet est venu me demander
                           que
                           la foire S.\up{t} Georges soit remise après le
                           service
                           du 25 avril. Et le soir l'abbé Ragot
                           me montre une lettre de
                              l'éveque de Toul
                           qui fixe au 14 mai le service de la
                           ville
                           et me prie d'écrire à l'abbé
                              Clement
                           pour l'oraison funèbre du roi Stanislas.
                        \bigskip


                     \end{diary}

                     \begin{diary}{27 Mars 1766}{}


                           Mon frère m'écrit d'hier de Lunéville que
                           le sort de la maison du roi
                           est à peu près
                           décidé. Celui du conseil, des
                           greffes \&\up{a}
                           la chancellerie. Le régiment des gardes lorraine
                           partira le 31 par le Dauphiné. On tirera
                           la milice dans le cours
                           du mois prochain. \bigskip



                         On m'écrit d'hier de Lunéville que le sort
                           du conseil est réglé. Les membres
                           qui
                           le composaient conservent leurs appointements.
                           Mes frères
                           ont chacun 1500\up{\#.} de
                           pension M. Guire
                           3000\up{\#.}
                           le S. Bremont
                           800\up{\#.}
                           le S. Petit Jean
                           800\up{\#}. \bigskip


                     \end{diary}

                     \begin{diary}{28 Mars 1766}{}

                         Je reçois une lettre d'aujourd'hui de
                           M. l'évêque de Toul, suivant
                           laquelle
                           le service de la ville est renvoyé au 27
                              mai
                        \bigskip


                     \end{diary}

                     \begin{diary}{30 Mars 1766}{}


                           M. le comte de Stainville ordonne
                           une
                           garde bourgeoise de 50 hommes par jour au
                           château de
                              Lunéville. Et un détachement
                           du régiment du roi pour relever à la Malgrange
                           celui des gardes
                           lorraines. \bigskip


                     \end{diary}


                     \begin{diary}{31 Mars 1766}{}


                           M. le comte de Stainville a
                           ordonné un
                           détachement de 12 hommes du régiment du roy pour
                           y remplacer à la
                              Malgrange
                           celui des gardes lorraines.
                           Et 50 hommes de milice bourgeoise
                           pour garder
                           le Château de
                              Lunéville, que le régiment gardes
                              lorraines quitte aujourd'hui pour aller à
                           Briançon en Dauphiné. Ils
                           partiront demain
                           de Nancy, pour Colombey \&\up{a}. \bigskip



                         C'est de demain aussi que les tables cessent
                           au Château de
                              Lunéville. On a levé le scellé
                           des papiers des greffes et de la
                              chancellerie
                           et on en forme des inventaires. Ils seront
                           transportés au greffe du Conseil de France
                           à Paris ; et il y aura arrêt à
                           cette
                           occasion. \bigskip



                           M. le duc de Fleury m'a écrit du
                              29
                           touchant ses nouvelles lettres de bailly
                           de Nancy. \bigskip


                     \end{diary}
                  \chapter*{Avril 1766}\addcontentsline{toc}{chapter}{Avril 1766}


                     \begin{diary}{01 Avril 1766}{}

                         Toutes les tables ont cessé au château de
                              Lunéville. \bigskip



                           Madame Prevost part de Nancy, avec sa
                              fille. \bigskip


                     \end{diary}

                     \begin{diary}{02 Avril 1766}{}

                         Je vais à Lunéville. M. le
                              chancelier
                           a proposé une pension de 2000\up{\#.} pour
                           moi, et autant pour chacun de mes frères. \bigskip


                         J'ai signé la levée des scellés de
                              la Chancellerie. \bigskip


                     \end{diary}

                     \begin{diary}{03 Avril 1766}{}

                         Je reviens à Nancy, où M.
                              l'intendant
                           était arrivé de Neuviller. \bigskip


                     \end{diary}

                     \begin{diary}{04 Avril 1766}{}


                           M. le comte de Stainville et
                              M. l'intendant
                           visite l'hôtel sur la
                              carrière, et conviennent
                           de leurs arrangements pour l'échange
                           réciproque de logement. \bigskip


                     \end{diary}

                     \begin{diary}{05 Avril 1766}{}


                           M. l'intendant retroune à
                              Lunéville, où
                           on prépare le départ des papiers. \bigskip



                           M. de Stainville
                           envoie un détachement du régiment du roi à Lunéville. \bigskip


                     \end{diary}

                     \begin{diary}{06 Avril 1766}{}

                         Je fais publier à Nancy l'arrêt du
                           10 mars concernant la liquidation
                           des dettes d’État de Lorraine. Et celui du
                           21 qui renvoie aux conseils de France,
                           les matières et affaires qui se jugeaient
                           aux conseils de Lorraine. \bigskip


                         Le régiment de fiffer suisse passe à Nancy. \bigskip



                           La mère de M. de S. Lambert
                           est morte ces jours-ci, sur
                              la paroisse Notre-Dame
                        \bigskip



                           M. le comte de Stainville est
                           parti à
                           minuit pour Paris. \bigskip


                     \end{diary}

                     \begin{diary}{07 Avril 1766}{}

                         On m'écrit de Paris
                           le 3 : \og M. le marquis de
                                 Castries, beau frère de M. le duc de Fleury,
                              vient d'avoir la lieutenance générale du
                              Barrois \fg{}. \bigskip



                           M. Sallet
                           subdélégué à Neufchateau est
                           mort ces jours derniers. \bigskip


                     \end{diary}

                     \begin{diary}{08 Avril 1766}{}


                           M. l'intendant est revenu hier de
                              Lunéville, avec
                           mon frère le commissaire. On a hier commencé
                           le transport de ses meubles de l'hôtel du gouvernement
                           qui sera sur la carrière, à l'intendance
                           qui sera
                           près de l'hôtel de
                              ville,
                              place royale.
                        \bigskip


                     \end{diary}

                     \begin{diary}{09 Avril 1766}{}

                         J'ai aujourd'hui à 4 h. du matin fait
                           publier l'ordonnance du roi du 27 novembre 1765
                           et l'extrait qui en a été fait pour la
                              Lorraine. \bigskip


                         Après midi assemblée particulière de
                              l'Académie
                           où étaient Messieurs
                           Du Rouvrois, de Sivry, de
                              Solignac, Thibault, de
                           Tervenus, abbé
                              Gautier, André,
                              P. Husson, Durival l'aîné
                           On y a lu quelques lettres sur la mort
                           déplorable du roi de Pologne écrites de la part
                           de plusieurs  académies. Un jardinier a presenté
                           un modèle de petite maison à mettre en sûreté
                           contre les entreprises des voleurs, et dont la porte se ferme au moyen d'une serrure à
                           secret, tellement que quand on l'a fermé avec
                           \sout{cette} la clé, on ne peut sans le
                           secret la
                           rouvrir avec la même clé. \bigskip


                         Délibéré que l'assemblée publique de la
                           S. Stanislas qui aurait du se faire le
                           8 mai prochain, jour de l'ascenssion,
                           sera
                           remise au dimanche suivant \bigskip


                     \end{diary}

                     \begin{diary}{13 Avril 1766}{}


                           M. l'intendant arrivé hier de
                              Neuviller
                           est parti ce matin pour Lunéville. \bigskip



                           Colin a gravé des sceaux pour la
                              Cour souveraine.
                           Et les deux Chambres des Comptes mais on
                           ne les remet pas encore, parce qu'il y a
                           dispute entre le
                              vice-chancelier, le
                              duc de Choiseul et le contrôleur général à qui
                           les fera remettre aux cours. \bigskip


                         On a ordonné aujourd'hui dans les
                           églises des prières pour la reine,
                           et j'ai fait cesser le spectacle. \bigskip



                           M. Viot fait hier son opération de
                           milice pour Lunéville, et
                           y tira 25
                           miliciens. \bigskip


                     \end{diary}

                     \begin{diary}{15 Avril 1766}{}


                           M. l'intendant revient de
                              Lunéville, d'où
                           les papiers de la
                              chancellerie et des greffes
                           sont prêts à partir ; ce qui ne sera décidé
                           qu'à son arrivé à Paris. \bigskip


                     \end{diary}


                     \begin{diary}{16 Avril 1766}{}

                         Les nouvelles lettre de M. le duc de fleury,
                           du 1.\up{er} mars
                              1766, en qualité de bailli de
                           Nancy, enregistrées en la Cour souveraine et au
                           baillage, ont été aussi
                           registrées en l'hôtel
                              de ville.
                        \bigskip


                         Ce matin la Cour souveraine a été assemblée sur
                           l'affaire des Jésuites. Arrêt qui donne main-
                              levée des saisies : signifié aux dépositaires
                           et gardes de la Conétablie, ils ont refusé de
                           se retirer des maisons où on avait saisi. \bigskip


                         Sur quoi ordre à l'huissier de les
                           y obliger,
                           et de se faire accompagner par la maréchaussée.
                           Ils se sont laissés conduire en prison. \bigskip


                     \end{diary}

                     \begin{diary}{18 Avril 1766}{}


                           M. l'intendant est allé ce matin
                           à Neuviller
                           et reviendra ce soir. \bigskip



                           M. de Silly, major de la place, s'est
                           battu
                           cet après midi contre M. de
                              Lescoure, capitaine
                           officier au régiment du roi, et a reçu un coup
                           d'épée. \bigskip



                           M. le contrôleur
                              général avait écrit le
                              10
                           à M. de la Galaizière ancien chancelier de Lorraine
                           touchant les pensions accordées aux commis
                           de la Chancellerie du roi de Pologne et aux greffiers
                           des conseils, et M. de la Galaizière y avait repondu
                           le 14. Le
                              17 il a ecrit de sa main à
                           M. de Beaumont intendant des
                           finances, en faveur de
                              mes frères
                           et de moi, pour
                           augmenter nos pensions. \bigskip


                     \end{diary}

                     \begin{diary}{19 Avril 1766}{}


                           Mes
                              frères arrivent de Lunéville. Je donne
                           l'ordre pour le tirage des 35 miliciens de
                           la ville de Nancy. \bigskip


                     \end{diary}

                     \begin{diary}{20 Avril 1766}{}


                           M. l'intendant part pour
                              Paris, avec
                           mon frère le plus
                              jeune, à 4 h. du matin. \bigskip


                     \end{diary}

                     \begin{diary}{21 Avril 1766}{}

                         L'assemblée pour la milice de Nancy, commence
                           à 6 h. du matin. Il s'est presenté près de
                           800 garçons, mais par le grand nombre
                           d'étrangers et de ceux qui n'avaient pas ou
                           l'âge ou la taille \&\up{a}. le nombre des miliciables
                           a été réduit à environ 250. Il y a eu
                           beaucoup de tumulte, en sorte que l'examen
                           n'a pu être fait qu'à 8 h. du soir, et que
                           j'ai été obligé de remettre le sort au lendemain. \bigskip


                     \end{diary}

                     \begin{diary}{22 Avril 1766}{}

                         J'ai fait publier une ordonnance à 6 h. du matin
                           pour le rassemblement de ceux qui avaient
                           été séparés pour le sort. M. le marquis de Choiseul
                           m'a accordé une forte garde. L'assemblée
                           avait commencé à 8 h. du matin mais les
                           garçons refusaient d'entrer dans la salle
                           du concert, demandant que les fils de marchands
                           et autres qui avaient été exemptés soient
                           soumis au sort. Je l'ai refusé. Les garçons
                           se sont mutinés. Un soldat ayant bourré l'un d'eux, tous se
                           sont avancés pour le soutenir.
                           Il protestaient de ne pas tirer au sort si la
                           garde ne se retirait. Je l'ai encore refusé.
                           Vers dix heures la sédition a augmenté, et
                           il y avait à craindre ; mais par la fermeté
                           et la sagesse du sergent de garde, tout est
                           devenu plus calme. Surtout après que
                           j'ai eu déclaré aux garçons que j'allais
                           déclarer miliciens environ une trentaine
                           qui étaient entrés ; et eux tous miliciens
                           de droit pour servir surtout à la décharge
                           des premiers. J'ai pris la résolution de
                           tirer par paroisses, à commencé par
                           celle de Notre-Dame alors les mutins à la vue
                           de ceux que le sort avait affranchis ont
                           demandés d'être admis à tirer, et l'opération
                           s'est faite assez
                           tranquillement. J'ai tiré
                           36 miliciens sur 227 miliciables.
                           Des domestiques de directeurs des fermes
                           \&\up{a}. ont été soumis au sort. J'ai été
                           obligé de faire tirer des soldats congédiés
                           de 16 ans de service, mais le sort leur
                           a été favorable. \bigskip


                     \end{diary}

                     \begin{diary}{23 Avril 1766}{}

                         Mort de madame Malcuit, veuve du maître des
                              comptes.\bigskip


                     \end{diary}

                     \begin{diary}{26 Avril 1766}{}


                           Mon jeune frère a donné de
                           ses nouvelles
                           de Paris
                           le 22 il est logé chez
                           M. l'intendant
                              de Lorraine.
                        \bigskip


                     \end{diary}

                     \begin{diary}{27 Avril 1766}{}

                         Assemblée à 3 h. après midi à l'hôtel de ville
                           où étaient les 2
                              premiers présidents, les
                           2 procureurs généraux
                           le lieutenant général du
                              Bauge
                           et moi, directeur de la fondation des
                              maladies épidémiques, grêle et incendie.
                           Après la distribution de l'argent à des laboureurs
                           incendiés on a fait une délibération qui accorde
                           300\up{\#.} de France par année au S. Béchet, receveur
                           de la fondation ; règle que les comptes seront
                           rendus chaque année dans le 15 premiers
                           jours de mai ; et que les drogues seront
                           prises de la pharmacie de la veuve
                              Virion.
                           On a résolu aussi de demander au roi la
                           confirmation et l'exécution du legs de 100000\up{\#.}
                           fait par le testament du feu
                              roi de Pologne,
                           pour augmenter de 5000\up{\#.} les rentes de la
                           même fondation. \bigskip


                         J'ai fait des représentations sur
                           lesquelles M. le
                              procureur général doit faire ordonner que les corps des
                           hopitaux S.
                              Julien, et S.
                              Charles seront inhumés
                           au cimetière
                              extérieur de la porte S. Jean. \bigskip


                         On m'amène la nouvelle charrue que
                           M. Genneté a été me faire faire
                           à
                           Champigneulle. \bigskip


                     \end{diary}

                     \begin{diary}{28 Avril 1766}{}

                         Je tire à Nancy
                           la milice pour la ville
                           de Saint-Nicolas. Cinq
                           miliciens \bigskip


                     \end{diary}

                     \begin{diary}{29 Avril 1766}{}

                         Je tire celle des cantons de Varangeville
                           et d'Ency. 3 miliciens
                           chaque. \bigskip


                     \end{diary}



                     \begin{diary}{29 Avril 1766}{}


                           M. Jean-Charles François, conseiller en l'hôtel de ville,
                           est mort ce matin, pendant mon opération
                           de milice. Il sera inhumé demain aux religieuses
                              de Sainte-Élisabeth. Il laisse 3 filles et un fils. \bigskip


                     \end{diary}

                     \begin{diary}{30 Avril 1766}{}

                         Je tire le matin la milice du canton de
                           Bouxieres-aux chênes ; 30
                           soumis au sort, 3 miliciens. \bigskip


                         L'après midi celui de Faux 52 soumis au
                           sort, 4 miliciens \bigskip


                     \end{diary}
                  \chapter*{Mai 1766}\addcontentsline{toc}{chapter}{Mai 1766}


                     \begin{diary}{01 Mai 1766}{}

                         J'ai l'après midi tiré la milice du canton
                           de Malzéville. 48 soumis au
                           sort, 4 miliciens. \bigskip


                     \end{diary}

                     \begin{diary}{02 Mai 1766}{}

                         Ce matin le canton de Grondreville. 3
                           miliciens.
                           L'après midi celui de Pont-Saint-Vincent 3. \bigskip


                     \end{diary}

                     \begin{diary}{03 Mai 1766}{}

                         Aujourd'hui matin le canton de Flavigny. 3.
                           Cet après midi celui de
                              Lupcourt. Il devait
                           aussi fournir 3 miliciens. Mais ne m'étant resté
                           que 8 hommes réservés pour le sort, j'ai pris
                           sur moi de ne tirer qu'un milicien. \bigskip


                         J'ai reçu par cet ordinaire des
                           lettres de Paris.
                           Une de mon jeune frère du 30 avril. Il avait vu M.
                              La Pierre. M.
                              Dailly qui de tous les ouvrages présentés
                           sur l'impôt ne trouve de bon que celui
                              de mon frère.
                           M. le chancelier de Lorraine sera appelé à Nancy dans
                           une douzaine de jours. Les procédures de la Cour souveraine de Lorraine sur les Jésuites vont être annulées.
                           M. le comte de Stainville
                           m'écrit du même jour,
                           sur les ouvrages de l'hôtel
                              du gouvernement
                           et M. l'intendant du 1.\up{er}
                              mai
                           approuve fort
                           mon opération de milice de Nancy. Je lui
                           envoie
                           le bon mémoire de M. Guerre,
                           demandé par le duc de Choiseul,
                           sur les
                           abbayes et prieurés de Lorraine. \bigskip


                     \end{diary}


                     \begin{diary}{04 Mai 1766}{}


                           M. l'intendant sur les ordres de
                              M. le
                              vice-chancelier, a fait remettre aujoud'hui
                           par M. le Changeur, au procureur général de la Cour souveraine
                           et à celui de la Chambre des Comptes, les nouveaux
                           sceaux gravés ici par Collin.
                        \bigskip


                     \end{diary}

                     \begin{diary}{07 Mai 1766}{}

                         Assemblée de l'Académie où étaient Messieurs
                           Du Rouvrois, Thibault, de
                              Tervenus, Bagard,
                           Cupers, Sozzi, P. Husson,
                              Durival l'aîné.
                           on a proposé pour académiciens Messieurs
                           de La
                              Fargue, François et Mittié,
                           admis. \bigskip


                         Il est arrivé un arrêt du conseil qui casse
                           ceux de la Cour souveraine rendus en faveur des
                              Jésuites
                           défenses d'en rendre des semblables. Ordre au
                           commandant de donner main forte. \bigskip


                     \end{diary}

                     \begin{diary}{08 Mai 1766}{}

                         Je pars pour Lunéville et revient le même
                           jour. Il venait d'arriver à Nancy des
                           archers de la connétablie,
                           comme je rentrais. \bigskip



                           M. le chancelier fait état de partir pour toujours
                           de Lunéville mardi
                           prochain. M. l'intendant
                           qui est à la
                              Galaizière n'en reviendra à
                           Paris que le 12. \bigskip


                     \end{diary}

                     \begin{diary}{09 Mai 1766}{}

                         J'ai vu ce matin M.
                              le cardinal de Choiseul
                           arrivé d'hier. L'après midi j'ai reçu cet
                           avertissement. \bigskip


                        \begin{quote}M. le lieutenant
                                    général de police\bigskip


                              Je suis chargé Monsieur de la part de la Cour
                              de vous avertir qu'elle assistera au service funèbre et solennel
                              qui sera célébré par ordre
                              du roi en l’église primatiale
                              demain samedi
                                 dix mai (1766) neuf heures et demie du matin
                              pour le repos de l'âme de feu Sa Majesté le roi de Pologne,
                              et de vous inviter de vous y trouver, à la tête de
                              Messieurs vos collègues en habits de cérémonie selon
                              l'usage \bigskip

        \begin{flushright}\emph{signé}
                                 Collenel.\end{flushright}\end{quote}
                     \end{diary}

                     \begin{diary}{10 Mai 1766}{}


                           M. de la Galaizière père, M. de
                              Lucé, madame et
                           M.\up{lle} de la
                              Millière sont arrivés à 9 h \up{1}/\textsubscript{4}
                           l'hôtel de ville s'est rendu à
                              la primatiale
                           à 9 \up{1}/\textsubscript{2}. La Cour souveraine
                           la Chambre des Comptes
                           le bailliage,
                           la maitrise, l'hôtel de ville, les juges consuls,
                           l'Académie, étaient
                           placés après 10 h. Le
                           prélat officiant avait un dais du côté de
                           l'épitre, le marquis de Choiseul commandant un fauteuil, les
                           gens de la Cour du feu roi,
                           dans le sanctuaire
                           et près de la chaire. Les dames dans la
                           nef d'un côté, l’état-major et les officiers
                           de l'autre en très bel ordre. L'\emph{oraison
                                 funèbre}
                           a été prononcée par le P. Elisée
                           et a duré
                           1 h \up{1}/\textsubscript{2}. On a fini par les
                           obséques, et on
                           n'a pu sortir de l’église que vers 2 h. après
                           midi. \bigskip



                           M. de la Galaizière est reparti pour Lunéville
                           après dîner, passant par Fléville. \bigskip



                           Mon frère le commissaire a reçu par cet
                           ordinaire 3 lettres datées de Versailles
                           le 5.
                           Une de M. l'intendant, une du
                              duc
                              de Choiseul qui lui marque de se rendre
                           aussitôt auprès de lui, la 3.\up{e} de M. de La Ponce,
                           secrétaire de confiance qu'il y a apparence
                           que mon frère
                           remplacera auprès du ministre.
                        \bigskip


                     \end{diary}
                     \begin{diary}{11 Mai 1766}{}

                         Les académiciens se sont rendus
                           à dix h.
                           du matin aux
                              Cordeliers, où il y a eu une
                           grande messe, pendant laquelle le P.
                           a prononcé l'éloge de S. Stanislas, évêque de
                              Cracovie. Il y a inseré très bien celui du
                           Stanislas que nous pleurons.
                           On n'est sorti
                           qu'à midi pour aller dîner chez
                           M. le premier président.
                           À trois heures à l'hôtel de
                              ville, où une
                           grande et belle assemblée était impatiente
                           d'entendre l'éloge historique du feu roi de Pologne
                           par M. de Solignac. Les
                           académiciens étaient
                           Messieurs
                           Du Rouvrois, de Sivry, de
                           Solignac, cardinal
                              de Choiseul, ch.\up{er} de Bouflers, Thibault, André,
                           Bagard, Cupers, Harmant,
                              P. Husson, P. Leslie,
                           Devaux, Liébault, abbé
                              Montignot, Porquet,
                           Foliot, de
                              Sozzi, Coster, Custine,
                           Durival l'aîné.
                           On n'a pas été content du discours ; on y a
                           trouvé déplacé, Jupiter,
                              Pallas, Prométhée,
                           Achilles, Enée,
                              Deucalion et Pirrha, et
                           beaucoup
                           de comparaisons inutiles. L'orateur n'a point
                           touché, et madame de Bouflers qui l'a entendu dit qu'on lui a
                           l'obligation de nous avoir
                           empêcher de pleurer. Ce discours n'avait point
                           été lu dans les assemblées particulières. \bigskip


                         Les Consuls sont allés voir M. le chancelier
                           à Lunéville. \bigskip


                         On a representé \emph{l'Écossaise} qui a fait
                           repandre des larmes. \bigskip


                         Mort de Aime-Claire Perault
                           femme du S. Mathieu
                              Lallement, intéressé
                           dans les affaires du roi.
                           Elle sera inhumée
                           dans l’église de la paroisse S. Nicolas.
                        \bigskip


                     \end{diary}

                     \begin{diary}{12 Mai 1766}{}

                         Service solennel pour Le roi de Pologne
                           dans l’église
                              primatiale, de la part du chapitre.
                           L'abbé de Dombasle l'un de
                           ses membres
                           a prononcé l'\emph{oraison
                              funèbre}. \bigskip


                         J'ai reçu de Paris des lettres de mon jeune
                              frère des 8 et
                              10. Il y a une commission
                           nommée et il doit y avoir un travail
                           sur les choses importantes de la Lorraine. \bigskip



                           M. de Lally, condamné le 6 à être
                           décapité a été éxécuté le 9. Il
                           avait
                           cherché à éviter le supplice en se donnant la
                           mort ; sa famille a sollicité inutilement
                           sa grace. \bigskip


                     \end{diary}
                     \begin{diary}{13 Mai 1766}{}


                           M. de La Galaizière, ancien
                           chancelier de
                           Lorraine, est parti ce matin
                           de Lunéville, à 5 h. moins un quart et est arrivé à
                           Nancy à 7 h \up{1}/\textsubscript{2} avec mon frère le commissaire
                           ils dîneront à Saint-Aubin. La
                              reine est fort
                           impatiente de voir M. de La Galaizière et on comptait
                           qu'il arriverait le 10 à
                              Paris. \bigskip


                     \end{diary}

                     \begin{diary}{15 Mai 1766}{}

                         Service des marchands à S. Roch, pour le feu
                              roi de Pologne
                           M. Coster de Remiremont a prononcé
                           l'\emph{oraison funèbre}. \bigskip


                     \end{diary}

                     \begin{diary}{16 Mai 1766}{}


                           Madame M.  me fait part à Heillecourt
                           de sa résolution d'entrer au couvent. \bigskip


                     \end{diary}

                     \begin{diary}{17 Mai 1766}{}

                         Je reçois des nouvelles de mes deux frères
                           qui sont à Paris. M. le chancelier de Lorraine
                           y est arrivé le 14 à 9. h. du soir.
                              Mon
                              frère le commissaire avec lui, il était attendu
                           avec impatience. \bigskip


                     \end{diary}

                     \begin{diary}{19 Mai 1766}{}


                           M. le cardinal de Choiseul part
                           aujoud'hui
                           de Nancy
                        \bigskip



                           Suivant les lettres de Paris des
                              16 et 17,
                           le roi était à
                              S. Hubert, et M. le duc de
                              Choiseul aussi. M.
                              le chancelier et l'intendant
                              de Lorraine devaient aller à Versailles
                           le 18.
                           M. l'intendant me demande des éclaircissements
                           sur la Cour souveraine et les 2 chambres
                        \bigskip


                     \end{diary}

                     \begin{diary}{20 Mai 1766}{}

                         Service des Jésuites à S.
                              Roch, pour le feu
                              roi de Pologne. L'évêque de Toul
                           y officie et
                           le P. Coster prononce
                                 l'\emph{oraison funèbre}
                        \bigskip


                     \end{diary}

                     \begin{diary}{21 Mai 1766}{}


                           M. Dorly m'écrit de Versailles
                           le 18, sur ce
                           qu'il appelle l'exaltation de mon
                              frère, dont
                           je n'ai encore point d'autres nouvelles. \bigskip


                     \end{diary}

                     \begin{diary}{22 Mai 1766}{}


                           M. Mirebeck, qui a l'agrément d'une
                           place
                           d'avocat au conseil, et en attendant
                           permission de M. le
                              vice-chancelier d'y
                           travailler, passe avec la voiture de M.
                              de La Galaizière chancelier allant à Paris.
                        \bigskip


                     \end{diary}

                     \begin{diary}{24 Mai 1766}{}

                         Le feu dans une baraque de planches
                           rue de
                              l'Esplanade à tennant aux maisons de
                           M. de Riocourt.
                        \bigskip



                           Mon frère qui m'avait
                           prévenu dès le 19
                           m'écrit de Versailles
                           le 21
                           \og Je suis en exercice
                              de la place de \emph{secrétaire des affaires
                                 étrangères}.
                              Je partage aussi avec M. de La
                                 Ponce les détails
                              du secrétariat de la guerre, comme il partage
                              avec moi ceux du secrétariat des affaires
                              étrangères, afin de pouvoir mutuellement
                              nous suppléer auprès du ministre \fg{}
                        \bigskip



                           Mon jeune frère m'avait appris du 18 que
                           M. de La Galaizière chancelier avait un bureau au conseil
                           ; que la reine lui
                           aurait donné
                           une tabatière d'or enrichie de son portrait. \bigskip


                     \end{diary}

                     \begin{diary}{25 Mai 1766}{}


                           L'abbé Clément arrive de
                              Versailles pour
                           l'\emph{oraison funèbre} de demain. \bigskip



                           L'évêque de Toul a aujoud'hui
                           fait la bénédiction
                           de la nouvelle église des Sœurs Grises. \bigskip


                     \end{diary}

                     \begin{diary}{26 Mai 1766}{}

                         Tout étant prêt de grand matin et
                              M.
                              l'évêque de Toul
                           ayant fixé l'heure du service
                           on a allumé les cierges et les bougies
                           dès les 9 heures. Tout le monde étant placé
                           dans un bel ordre. Clérgé, état major, officiers
                           de la garnison et autres, les dames de qualité,
                           les autres dames sur des chaises et des
                           gradins, les gens de robe et autre,la
                           noblesse, tous en deuil et l’église étant
                           entièrement remplie M.
                              l'évêque est entré
                           peu après dix heures, précédé du Clergé des
                           sept paroisses, et de trois abbés crossés et
                           mitrés. On a commencé à 10 h \up{1}/\textsubscript{4} la
                           messe avec la musique de la primatiale.
                           M. l'abbé Clément a
                           prononcé l'\emph{oraison funèbre}
                           qui a duré une heure et demie et quelques
                           minutes. La messe étant achevée on
                           a fait les obsèques autour du catafalque.
                           Éclairé de six urnes d'un grand effet.
                           La figure du roi de 7 pieds de proportion
                           était levée a demi sur le tombeau, soutenue
                           par la religion. À un bout sur les gradins d'en bas la ville de Nancy en pleurs, à l'autre
                           la charité. Tout a été trouvé très bien
                           et de meilleur goût que les services de la
                           primatiale. Personne de marque ne s'est
                           dispensé d'y assister, quoiqu'il n'y eut que
                           le corps municipal en habit de cérémonie.
                           Un peuple immense attendait à la porte
                           et lorsqu'on a été sorti de l’église vers
                           deux heures après midi. Elle a été sur le
                           champs remplie de spectateurs, pour la
                           satisfaction desquels on a laissé tout
                           éclairé avec une heure de plus \bigskip


                     \end{diary}

                     \begin{diary}{27 Mai 1766}{}

                         On a détendu l’église S. Roch et les
                              principales piéces du mausolée ont été
                           transportées dans un cabinet à côté
                           de la salle de l'hôtel de ville. Le sculpteur
                           qui a fait ces beaux morceaux se nomme Jean-Joseph
                              Sontken de Coesfeld, évêché de Munster Wesphalie. \bigskip


                     \end{diary}
                     \begin{diary}{28 Mai 1766}{}

                         Assemblée particulière de l'Académie où étaient
                           Messieurs
                           Du Rouvrois, de Solignac, Thibault, André
                           de Sozzi, Coster, P. Husson,
                              P. Leslie et Durival
                              l'aîné. Il a été d'abord question de l'admission
                           du P. Elisée, proposée chez
                           M. le premier le jour
                           de S. Stanislas ; elle a fait quelque difficulté.
                           Ensuite on a parlé du service que l'Académie
                           doit faire célébrer aux Cordeliers, où M.
                              l'abbé Guyot prononcera l'\emph{oraison funèbre} de Stanislas.
                           Délibéré que l'on prendra 300\up{\#.} de France sur
                           les fonds de l'Académie, et
                           que les académiciens suppléeraient au surplus. Que le P. Husson
                           serait chargé des détails ; et qu'on écrirait
                           une circulaire aux académiciens pour y assister
                           en nombre. \bigskip



                           M. de Sozzi a lu une dissertation sur deux
                              odes d'Horace, pour montrer l'accord de leurs
                           parties. À la fin un remerciement à l'Académie
                           dans la supposition qu'il n'aura plus
                           occasion d'y paraître. \bigskip



                           M. de Lucé arrive de Metz, et nous apprend
                           qu'une partie des casernes de
                              Chambière
                           a été incendiée hier vers dix heures du soir
                           on croit que c'est le feu du ciel, il a
                           péri quelques personnes. \bigskip


                     \end{diary}

                     \begin{diary}{29 Mai 1766}{}

                         La procession de la Fête-Dieu s'est
                           faite
                           sortant de la
                              primatiale à S.
                              Roch, à S.
                              Sébastien, au S. Sacrement. La Cour souveraine
                           y était, le 1.\up{er} président y manquait. Il y
                           avait aussi le bailliage,
                              la maitrise,
                           l'hôtel-de ville. Il n'a point paru
                           de
                           commandant de la province. Cette cérémonie
                           a duré de puis 8 h du matin jusqu'à 11 \up{1}/\textsubscript{2}. \bigskip


                         Par les lettres reçues d'hier M. l'intendant
                           a arrangé ses bureaux. M. le
                              Changeur est
                           le chef de tous. \bigskip


                     \end{diary}
                  \chapter*{Juin 1766}\addcontentsline{toc}{chapter}{Juin 1766}


                     \begin{diary}{04 Juin 1766}{}

                         Sur mon procès verbal du
                              2 juin, la commission
                           accordée jusqu'à bon plaisir le 30 mars
                              1757 à
                           Pierre Robinet, a été
                           révoqué, avec défense à
                           lui de plus s'immiscer dans les fonctions
                           de commissaire de police.
                           Le vent Sud, 20 degrés de chaud, à 2 h \up{1}/\textsubscript{2}
                              après midi.
                           Assemblée de l'Académie où
                           étaient Messieurs
                           Du Rouvrois, de Sivry, de
                              Solignac, de Tervenus,
                           Thibault, André, Bagard, Harmant,
                              Cupers,
                           P. Leslie, Coster, abbé
                              Gautier, Durival
                              l'aîné.
                           Dans la supposition que M. l'abbé Guyot,
                           qui doit prononcer l'\emph{oraison funèbre} du roi de Pologne,
                           arrivera aujoud'hui, l'Académie a fixé à
                           mardi prochain, le jour du service qu'elle fera
                           dans l’église des
                              Cordeliers. Il n'y aura ni
                           draperie ni catafalque ;
                           ainsi qu'en use
                           l'Académie
                              française. Les académiciens
                           payeront. \bigskip


                     \end{diary}

                     \begin{diary}{05 Juin 1766}{}


                           21 degrés de chaud à 5 h. du soir. \bigskip



                           M. le comte de Stainville arrive à
                              Nancy
                           et loge pour la 1\up{re} fois au gouvernement sur
                           la carrière. Il s'était arreté quelques jours
                           à Metz. \bigskip


                     \end{diary}

                     \begin{diary}{06 Juin 1766}{}

                         Je l'ai vu aujoud'hui.
                           Et j'ai signé l'état des encaissements des
                           papiers de Lunéville, en
                           qualité de second 1.\up{er}
                           commis du sceau. \bigskip


                     \end{diary}



                     \begin{diary}{07 Juin 1766}{}


                           L'hôtel de ville a nommé commissaire au
                           faubourg
                              Saint-Pierre, Darche, pour
                           remplacer
                           Robinet. \bigskip


                         Service de l'hôtel de ville à S. Roch, pour
                           feu M. François conseiller
                           et aux Minimes pour Le roi de Pologne. \bigskip


                     \end{diary}

                     \begin{diary}{08 Juin 1766}{}

                         Je signe trois états de l'envoi des papiers
                           de Lunéville. Ils sont datés du 9. \bigskip


                     \end{diary}

                     \begin{diary}{10 Juin 1766}{}

                         Service aux Cordeliers pour le roi de Pologne
                           pendant
                           lequel M. l'abbé Guyot,
                           aumônier de M. le
                              duc d'Orléans a prononcé l'\emph{oraison funèbre}. Elle a
                           duré 5 quarts d'heure, et le sujet a été très
                           bien traité. C'est l'Académie
                           et les académiciens
                           qui en font les frais. \bigskip


                     \end{diary}

                     \begin{diary}{11 Juin 1766}{}

                         Service aux Cordeliers pour le duc Léopold
                           et pour le roi Stanislas.
                        \bigskip


                     \end{diary}

                     \begin{diary}{17 Juin 1766}{}

                         Sur un ordre de M. Alliot que M.
                              le comte de
                              Stainville m'a remis, les cent mille livres de France
                           de la succession du feu roi de Pologne pour les casernes
                              de Nancy, ont été payées à M.
                              Richer, trésorier
                           de l'hôtel de ville par M. Michel, et par M.
                              Richer remises au S.
                              Mique. \bigskip


                     \end{diary}

                     \begin{diary}{18 Juin 1766}{}

                         Service aux Capucins de Nancy pour le
                              feu roi de Pologne
                        \bigskip


                     \end{diary}

                     \begin{diary}{19 Juin 1766}{}

                         Service aux Orphelines, pour Sa Majesté Polonaise
                        \bigskip


                     \end{diary}

                     \begin{diary}{20 Juin 1766}{}

                         Les papiers de la Chancellerie et des greffes
                           de Lunéville partent pour
                              Paris. \bigskip


                         Mort de Mutlot architecte. \bigskip


                         Assemblée d'une partie de la noblesse 28 gentilhommes
                           chez
                           M. le comte de Stainville, à
                           l'occasion
                           du Parlement et des bruits
                           desavantageux
                           qui s'en répandaient à Nancy.
                        \bigskip


                     \end{diary}

                     \begin{diary}{22 Juin 1766}{}


                           M. Brémont part de Nancy, pour suivre les
                           papiers de Lunéville, qui
                           arriveront à Paris
                           le 1.\up{er}
                           juillet. \bigskip


                     \end{diary}

                     \begin{diary}{23 Juin 1766}{}

                         Les avocats de Nancy s'assemblent deux fois au
                           palais pour résoudre s'il
                           ne convient pas d'envoyer
                           en France deux députés de
                              l'ordre, dans la
                           circonstance où on craint la translation d'un
                           Parlement. Le bâtonnier et
                           trois autres viennent
                           me dire ce qui s'était passé ; invitant l'hôtel
                              de ville à délibérer là dessus et prendre un
                           parti, puisque les cours n'envoient point de
                           députés. Je n'ai voulu m'engager à rien,
                           sinon à écrire à Messieurs
                           de la Galaizière ce que
                           j'ai fait à 9 h. du soir. \bigskip


                     \end{diary}
                     \begin{diary}{24 Juin 1766}{}

                         Le lendemain matin j'en ai rendu
                           compte
                           à M. le compte de
                              Stainville,
                           qui m'a dit
                           que les cours étaient en diligence là dessus,
                           et que des députés de corps particuliers
                           seraient mal reçus. Qu'il avait écrit à son frère \&\up{a}. \bigskip


                     \end{diary}


                     \begin{diary}{24 Juin 1766}{}

                         La troupe de Brisson, qui part demain et
                           après, a donné une dernière représentation
                           qui a commencé à 9 h. du soir. M. de
                              Stainville
                           a accordé qu'on passerait sans lumière passé les
                           dix heures. La foire de S. Jean le
                              vieil-âtre
                           se
                           tenait. \bigskip


                         Le  de ce mois Marianne Breton
                           ma nièce a épousé le S.
                              Bouchon, fils d'un marchand de Ligny
                        \bigskip


                     \end{diary}

                     \begin{diary}{25 Juin 1766}{}

                         Le ban des fenaisons sur le finage
                           de Nancy, ouvert pour demain 26
                        \bigskip


                         Assemblée particulière de l'Académie, où étaient
                           Messieurs
                           de Solignac,
                           de Sivry, André, de Tervenus,
                           Coster, abbé Gautier, Cupers et
                              Durival l'aîné.
                        \bigskip


                         On a agité si c'était sur les
                           fonds de la bibliothèque
                           que devaient être pris les frais du service fait
                           pour le roi de Pologne et le voyage de l'orateur. J'ai
                           soutenu que non ; on a remis à examiner la
                           chose dans une assemblée plus nombreuse.
                           Ainsi que la question si on proposera des
                           sujets à traiter pour les prix. \bigskip


                     \end{diary}

                     \begin{diary}{26 Juin 1766}{}


                           M. le comte de Stainville a fait
                           ces jours-ci
                           barrer les chemins
                              d'Heillecourt par la Malgrange
                           qui tourne la grotte. Les 3 capucins de
                           la Malgrange ont déjà
                           rejoint leur couvent
                           de Nancy. \bigskip


                         La troupe de Brisson est partie pour Dijon. Où vont s'assembler les états. \bigskip


                     \end{diary}

                     \begin{diary}{27 Juin 1766}{}

                         Et il nous vient en place la petite
                           troupe
                           d'opéra comique du S.
                              Chapuis
                           qui était à Toul. \bigskip


                     \end{diary}

                     \begin{diary}{28 Juin 1766}{}

                         Vers qui m'ont été envoyés par
                              M.
                              l'abbé Parquet, pour être mis au bas
                           de la statue du feu roi de
                           Pologne.
                        \bigskip


                        \begin{quote}\begin{verse}Il n'est point de vertus que son nom ne rappelle\\Philosophe et guerrier, monarque et citoyen,\\son génie attendit l'art de faire du bien.\\Charles fut son ami, Trajan fut son modèle.\\\end{verse}
        \bigskip

        \end{quote}
                     \end{diary}
                     \begin{diary}{29 Juin 1766}{}


                           M. le prince heréditaire de
                              Brunswic est
                           arrivé à onze heures du soir, venant de Metz,
                           avec Messieurs
                           d'Armentieres, de Rochouart et
                           autres. M. le comte de
                              Stainville leur a
                           donné un grand souper d'environ 45 couverts.
                           Le prince a couché au gouvernement
                           Messieurs
                           d'Armantières et de Rochouart à l'intendance
                        \bigskip


                     \end{diary}

                     \begin{diary}{30 Juin 1766}{}


                           Le régiment du roi s'est formé vers dix heures
                           du matin, sur les 4 faces de la
                              place royale.
                           Il a défilé devant le
                              prince, qui a vu ensuite
                           la bibliothèque, les
                              casernes \&\up{a}. Il est parti de là
                           pour Strasbourg. Messieurs
                           de Bouflers et de Cambis
                           avaient été du souper et sont partis pour
                           Lunéville en même temps. \bigskip


                     \end{diary}
                  \chapter*{Juillet 1766}\addcontentsline{toc}{chapter}{Juillet 1766}




                     \begin{diary}{01 Juillet 1766}{}

                         J'ai donné congé à Anne-Marie Demange,
                           qui était ma concierge à Heillecourt, depuis
                           le 14 mai 1765. \bigskip


                     \end{diary}

                     \begin{diary}{05 Juillet 1766}{}


                           M. le marquis du Chastelet,
                           notre ambassadeur
                           à Vienne, passe allant à
                              Versailles. \bigskip


                         Paraît l'édit de juin, enregistré
                           le 30 au Parlement de
                              Paris, qui fixe l'intérêt à 4 % \bigskip


                     \end{diary}

                     \begin{diary}{06 Juillet 1766}{}

                         J'installe dans ma maison d'Heillecourt, pour
                           jardinier et concierge, le nommé Lallemant, sa
                           femme et sa fille. \bigskip


                     \end{diary}

                     \begin{diary}{07 Juillet 1766}{}

                         On m'écrit du 7 de Lunéville : \og M.
                                 Mengin
                              procureur du roi est mort la nuit dernière d'apoplexie \fg{}. \bigskip


                     \end{diary}

                     \begin{diary}{09 Juillet 1766}{}


                           Le S. Pierrot adjoint commissaire de
                           police au
                              faubourg
                              Saint-Pierre. Maurice
                              Henry suisse
                           de la paroisse S.\up{t} Nicolas.
                        \bigskip


                         Assemblée particulière de l'Académie où
                           étaient Messieurs
                           Thibault, de Tervenus, André,
                           Harmant, Cupers, Gautier,
                              P. Husson,
                           et Durival l'aîné. M. l'abbé de Tervenus
                           a lu un morceau sur les peintres lorrains,
                           presque tout tiré de D. Calmet et
                           rempli
                           de beaucoup d'erreurs et d'omissions. \bigskip


                     \end{diary}

                     \begin{diary}{10 Juillet 1766}{}

                         Je suis allé par Bouxieres, Clevant, Custine,
                           Morey, à Duhautoy
                           ci-devant
                           Clemery, lever les
                           scellés mis sur l'armoire de l'abbé
                              Royer ; et j'ai repassé à Manoncourt-sur-Seille avec M.
                              de Marcol, conseiller à la Cour, qui en est
                           seigneur. \bigskip


                     \end{diary}

                     \begin{diary}{11 Juillet 1766}{}

                         J'ai envoyé dans une caisse par le carrosse
                           à M. de Sartine les effets
                           réclamés par l'abbé
                              Finatory.
                        \bigskip


                         Cet après midi j'ai vu à la Malgrange
                           l'abattis des vieux chêne dont quelques uns
                           avaient 3, 4, 5 et 6 piés de diamètre, la
                           pupart creux. On démolissait l'appartement du
                           roi, et la plupart des
                           matériaux étaient
                           enlevés et vendus. \bigskip


                     \end{diary}

                     \begin{diary}{12 Juillet 1766}{}

                         J'envoie à M. de Sartine le procès
                           verbal que j'ai
                           dressé le 10 à Du Hautoy, et j'en rends compte
                           à M. l'intendant. \bigskip


                     \end{diary}

                     \begin{diary}{13 Juillet 1766}{}

                         La pluie commencée le 11, qui a continué
                           hier et presque tout aujourd'hui, a fait déborder
                           les rivières. La prairie était toute couverte
                           d'eau, et tous les foins ne sont pas enlevés \bigskip


                     \end{diary}

                     \begin{diary}{14 Juillet 1766}{}

                         Il n'a pas plu aujoud'hui. J'ai vu
                           de la chaussée
                              Sainte-Catherine toute la prairie couverte d'eau
                           quoiqu'à 3 h. après midi elle eut déjà baissé
                           de près d'un pied. Il y a eu des tas entiers de
                           foin entrainés, on a vu flotter des armoires
                           et autres meubles sur la
                              Meurthe. Les deux quarreaux
                           planté de la pépinière royale n'ont
                           pas été innondés, mais l'eau avait couvert
                           tout l'autre côté de la chaussée. \bigskip



                           M. de Vassimont
                           conseiller en la Cour souveraine est
                           parti aujourd'hui avec M. Coster
                           comme
                           député. Mais on assure que le
                              premier
                              président a écrit pour les désavouer de la
                           part de sa compagnie. Il y a une réponse
                           de plusieurs gentilshommes au mémoire
                              indécent des Messins, plus  que le
                              mémoire de la Cour souveraine sans finesse, sans
                           recherches et sans pédanteries. \bigskip



                           La Chambre des Comptes sur une lettre écrite par son
                           député se presse d'envoyer quelques éclaircissements
                           sur la population de Nancy ; on
                           m'en a
                           fait demander que j'ai remis a M. de
                              Maisonneuve
                        \bigskip



                           M. le comte de Stainville arrive
                           de Strasbourg. \bigskip


                     \end{diary}

                     \begin{diary}{15 Juillet 1766}{}


                           M. de Riocourt
                           premier président de la Chambre des Comptes
                           part pour Paris, passant par
                              Troyes. Il
                           laissera madame de Riocourt en Champagne,
                           elle nourrit son dernier enfant. \bigskip


                         La pluie a recommencé cette nuit. \bigskip



                           M. de Vigneron, président à la
                           Cour, part comme
                           député pour Paris. Mort de
                              madame de Malleloy.
                        \bigskip


                     \end{diary}

                     \begin{diary}{16 Juillet 1766}{}


                           Messieurs
                           Grandjean, bâtonnier des avocats,
                              Olivier,
                           Jaquemin et Husson, partent aussi comme
                           députés de leur ordres. \bigskip


                         Il a plu hier. La pluie a recommencé aujourdhui, et on ne parle que de malheurs arrivés
                           par les innondations subites de la
                              Moselle
                           et de nos autres rivières. La route de Metz
                           a été interceptée près de Corny,
                              M. Friant
                           procureur du roi a pensé y être noyé. On dit que le
                           carrosse de voiture a eu hier le même accident
                           et que le cocher a été noyé. \bigskip


                         Aujoud'hui à l'assemblée de l'hôtel de ville
                           M. de Maisonneuve a demandé
                           une attestation
                           au bas du mémoire que je lui avait remis,
                           nous l'avons signé. \bigskip


                         J'ai
                           reçu le serment d'un commis à l'exercice
                           des droits de 2 francs par rezal, du pied
                           fourché \&\up{a}. et des nouveaux bédeaux de
                           Notre-Dame et de S.
                              Epvre. \bigskip


                         J'ai lu aujourd'hui le mémoire de
                           plusieurs
                           gentilshommes de Lorraines,
                           sur Parlement
                        \bigskip


                         Les marchands de Nancy ont député à Paris. \bigskip


                     \end{diary}

                     \begin{diary}{18 Juillet 1766}{}


                           M. le comte de Stainville part le
                           matin
                           pour Metz. M. le comte de Guerchi
                           arrive le soir. \bigskip



                           M. Mathis, ancien prévôt de
                              Boucquemont
                           employé depuis longtemps au réglement des
                           limites de la Lorraine, avec
                           les princes
                           voisins, est venu me voir aujourd'hui. Il
                           arrivait de Versailles,
                           où on vient de consommer
                           déjà quelques échanges avec le prince
                              de
                              Nassau. Le beau temps a succédé à la pluie
                           et on fait des foins. \bigskip


                         J'ai vu la continuation de la ruine
                           de
                           la Malgrange, dont il ne restera
                           presque rien. \bigskip


                     \end{diary}

                     \begin{diary}{19 Juillet 1766}{}

                         On a commencé aujourd'hui à abattre les
                           tilleuls des ramparts, qui seront remplacés
                           en automne par des jeunes ornes, et des frênes. \bigskip


                     \end{diary}

                     \begin{diary}{21 Juillet 1766}{}

                         Grand chaud. Il était encore à
                              20 degrés à 6 h.
                              du soir. \bigskip



                           Mon jeune frère m'écrit
                           du 16 que ce n'est
                           pas le moment de faire changer à nos pensions
                           que j'aurai 1000\up{\#.} et chacun de mes frères
                              1500\up{\#}. \bigskip


                     \end{diary}
                     \begin{diary}{23 Juillet 1766}{}

                         La pluie qui a recommencé hier continue
                           aujourd'hui. Le vent change d'un moment à l'autre. \bigskip


                         Assemblée particulière de l'Académie où étaient
                           Messieurs
                           de Solignac, Thibault, André, de Tervenus,
                           abbé Gautier, P. Husson, et Durival l'aîné. Il ne
                           s'y est rien traité d'intéressant. M.
                              de Solignac
                           a reparlé de M. Le Bas qui
                           sollicite une place
                           dans l'Académie, et de
                              M. Marin, autre
                           médecin, qui demande aussi d'en être. \bigskip


                         Hier le
                              prince de Hesse passa à Nancy
                           venant de Metz. Il fut reçu chez
                           M.
                              le comte de Stainville, où était M. de Guerchi.
                        \bigskip


                     \end{diary}

                     \begin{diary}{26 Juillet 1766}{}


                           Temps couvert, vent Sud-Ouest 21 degrés de chaud. \bigskip


                     \end{diary}

                     \begin{diary}{29 Juillet 1766}{}

                         Depart de M. de
                              Guerchi. \bigskip


                     \end{diary}

                     \begin{diary}{30 Juillet 1766}{}


                           M. Mique de Lunéville passe allant à
                           Versailles. \bigskip


                     \end{diary}

                     \begin{diary}{31 Juillet 1766}{}

                         Le lendemain M. de
                              Lucé
                        \bigskip


                         J'envoie à mes
                              frères, copie du mémoire
                              de 1738 sur la liquidation des dettes de
                              Lorraine. \bigskip


                         L'alarme est grande à Nancy, sur de
                           prétendus avis de Paris qu'il
                           n'y aura
                           point de parlement que celui de
                              Metz. \bigskip


                     \end{diary}
                  \chapter*{Août 1766}\addcontentsline{toc}{chapter}{Août 1766}



                     \begin{diary}{01 Août 1766}{}

                         J'ai vu aujourd'hui Bernard Gilly, italien,
                           haut de 7 pieds 8 pouces de roi. Il est âgé de
                           26 ans, n'a pas encore de barbe et est bien
                           proportionné. \bigskip


                     \end{diary}

                     \begin{diary}{02 Août 1766}{}


                           L'hôtel de ville a aujourd'hui décidé dans l'affaire
                           d'entre le nommé Chaudelot adjudicataire de la ferme
                           des drois sur les bois \&\up{a} et le S.
                              Mougenot, qu'il
                           n'y avait point de droits sur les bois de chauffage. \bigskip



                           M. le comte de Stainville, qui ne
                           devait partir
                           qu'après souper pour Paris, est
                           parti ce matin
                           après l'arrivée des lettres. \bigskip



                           Madame l'intendante est accouchée d'une fille
                           à Paris, la nuit du 29 au 30 Juillet. \bigskip


                     \end{diary}

                     \begin{diary}{04 Août 1766}{}

                         À Heillecourt les laboureurs coupent chacun
                           deux de froment, les autres habitants
                           chacun un. Le jour suivant sera le jour
                           du Seigneur. Le 6 la moisson ouverte
                           pour tous. \bigskip


                     \end{diary}

                     \begin{diary}{05 Août 1766}{}


                           La princesse Christine comtesse d'Henneberg, passe retournant
                           à Remiremont. \bigskip


                     \end{diary}

                     \begin{diary}{06 Août 1766}{}

                         Assemblée particulière de l'Académie où
                           étaient Messieurs
                           de Solignac, Thibault, de
                              Tervenus,
                           André, abbé Gautier, Durival
                              l'ainé.
                           On y a agité si l'assemblée publique se tiendrait
                           à l'avenir le 20 octobre jour de la naissance du feu roi ; il a été trouvé qu'il
                           n'était plus
                           temps de faire ce changement pour cette
                           année : mais qu'à l'avenir cela serait
                           transféré à la S. Louis 25 août. \bigskip


                     \end{diary}

                     \begin{diary}{07 Août 1766}{}

                         Plusieurs particuliers du faubourg de Bonsecours
                           rempiètent à leurs frais, la croix antique
                           \sout{des Bourguignons} de Jarville ou Geirville
                        \bigskip


                     \end{diary}

                     \begin{diary}{08 Août 1766}{}

                         Assemblée du bureau de l'aumône, où étaient
                           Messieurs
                           Du Rouvrois, Joli de Morey, de
                              Tervenus, de
                              Bressey, de
                              Dombale, et Durival. On
                           y a
                           condamné à 100l d'amende la
                              fille de Thouvenin,
                           huilier, sur un procès verbal du commissaire Rochette. \bigskip


                         On m'écrit de Metz
                           le 6 que le jugement
                           de Metz contre le S. Husson
                           subdélégué à Sedan a
                           été cassé et déclaré monstrueux et indécent,
                           au Conseil du roi, en
                           présence de 51 juges,
                           entre lesquels étaient M.
                              le chancelier, les
                           ministres d’État. Chaleur 22 degrés.
                        \bigskip


                     \end{diary}

                     \begin{diary}{09 Août 1766}{}


                           M. le premier président m'a fait remettre un
                           exemplaire du mémoire de la Cour souveraine contre
                              le parlement de Metz
                              in 4\degre, 36 pages. \bigskip


                         Le thermomètre de Réaumur marque
                              24 degrés
                              de chaleur, à 3 h. après midi. \bigskip


                         J'ai été d'une assemblée chez
                           l'avocat du roi,
                           pour la vente de Montaigu. \bigskip



                           Madame Durival retourne à Ubexi. \bigskip



                           Suivant les lettres du 7.
                           Le duc de Choiseul
                           aura vu ce jour là le camp de
                              Soissons
                           d'où celui de Compiègne,
                           où toute la Cour sera. \bigskip


                     \end{diary}



                     \begin{diary}{11 Août 1766}{}

                         On me mande de Paris
                           le 9, que la reine
                           est vivement affectée de la destruction
                           précipitée des bâtiments du roi son père. \bigskip



                           Le roi était à Compiègne dès le 7. \bigskip



                           la Cour souveraine me fait avertir par M.
                              Collenel
                           qu'elle s'assemblera vendredi 15 à
                              l'hôtel du
                              grand doyen de la primatiale, à 4 h\up{1}/\textsubscript{2} du soir
                           pour la procession du vœu de Louis
                              XIII. Et
                           de donner les ordres à tous les corps de métiers
                           de Nancy, qui sont en jurande de
                           s'y trouver
                           avec leur bannière dans l'ordre et le rang qui
                           leur convient ainsi qu'il est d'usage. \bigskip


                         Je me suis expliqué sur les
                           difficultés de ce
                           dernier article, avec M. le premier président,
                           M. Collenel, M. l'abbé Antoine.
                        \bigskip


                     \end{diary}

                     \begin{diary}{13 Août 1766}{}

                         Nous commençons l'audition du compte
                           de
                           1764. M. de Blair intendant de Strasbourg
                           arrive de Paris à 11 h. du
                           matin, pour aller
                           coucher à Joui
                           chez
                           M. le président Pierre.
                           Pendant que je lui montrais l'hôtel de
                              ville
                           le feu prend à la cheminée de la cuisine
                           chez
                           madame de Grandville, près du palais,
                           ce qui cause une grande inquiétude à
                           cause de la proximité des archives. \bigskip


                     \end{diary}

                     \begin{diary}{15 Août 1766}{}


                           Mon frère le commissaire m'écrit de Compiègne
                           le 11 \og Le camps des suisses se
                              séparera dans
                              deux jours. Celui des français s'assemblera
                              ensuite composé de 16 bataillons, et puis un troisiéme du seul
                                 régiment de navarre. Tous
                              cela sera fini pour le 28 \fg{}. \bigskip


                         Tout était prêt pour
                           faire aujourd'hui la
                           procession du vœu de Louis XIII, depuis la
                              primatiale
                           passant sur la
                              carrière, jusqu'à \sout{la ville neuve}
                           Notre-Dame et de là revenant par les places Saint-Èpvre et des
                              Dames. On s'est assemblé
                           à 4 h \up{1}/\textsubscript{2}
                           après midi, un moment après il a
                           plu, puis le temps s'est remis. Enfin vers les
                           5 h. on s'est mis en marche les confréries,
                           les paroisses, la primatiale, vue Notre-Dame
                           d'argent postée sur les épaules de deux enfants
                           de chœur ; puis la Cour
                              souveraine et
                           la Chambre des Comptes
                           côte à côte, le bailliage,
                           la maitrise des eaux et fôrets. l'hôtel de
                           ville,
                           la justice consulaire. Il n'y avait que
                           la maréchaussée et un détachement de
                           la garnison pour troupes. Sortant de
                           la primatiale par
                              la rue de la congrégation
                           on a passé entre l'hôtel de ville et la statue
                           de Louis XV. de là rue des Jacobins, on est
                           entré dans leur église et retourné à
                           celle de la
                           primatiale. La tête de la
                           procession y rentrait quand le queue
                           en sortait. \bigskip



                           Madame Prevost est partie ce matin pour
                           Paris. Elle va vendre Montaigu et
                           ne compte plus revenir en Lorraine. \bigskip


                     \end{diary}



                     \begin{diary}{16 Août 1766}{}

                         Le vœu de la
                              ville de Nancy et la fête de S. Roch
                           se sont célébrés aujoud'hui à l'ordinaire, par
                           une messe aux Minimes de
                              Bonsecours. Le repas
                           s'est fait dans cette maison. Les deux procureurs
                           généraux y étaient. \bigskip


                         Suivant des lettres du 14
                           M. de La Galaizière
                           intendant
                           de Lorraine était allé la veille à Etiole, d'où à
                           Montigny
                           chez
                           M. Trudaine, où il restera 8 à 10
                           jours. Il a envoyé son mémoire
                              d'observations
                              sur les 2 parlements ; étant probable que cette
                           affaire se décidera bientôt à Compiègne. \bigskip


                     \end{diary}

                     \begin{diary}{17 Août 1766}{}

                         J'ai reçu ce matin de M. de Marcol, procureur général
                           la lettre suivante. \bigskip


                        \begin{quote}\begin{flushright}Nancy
                                 le 15 août 1766\end{flushright}\bigskip

                     Monsieur les sentiments d'amour et de respect
                              pour l'auguste personne du
                                 roi, étant profondément
                              gravé dans le cœur de tous ses sujets, des provinces
                              de Lorraine et Barrois, on ne peut saisir avec trop
                              d'empressement toutes les occasions d'en donner des
                              marques et des démonstrations publiques ; c'est
                              pourquoi l'intention de la Cour
                                 souveraine est que
                              le 25 du présent mois, fête de S.
                              Louis dont Sa Majesté
                              porte le glorieux nom, il soit fait dans toutes les
                              villes et bourgs du reffort, les réjouissances et
                              observé les solennités qui sont de coutume à la
                              fête de nos souverains, et que la même chose soit
                              renouvellée chaque année à pareil jour. \bigskip

         En vous faisant part, Monsieur, des intentions
                              de la cour, je ne doute pas que l'hôtel de
                                 ville de
                                 Nancy, à qui vous voudrez bien communiquer ma lettre, ne se porte à donner des marques du
                              plus grand zèle dans cette occasion. \bigskip

         J'ai l'honneur d'être très parfaitement
                                 Monsieur votre très humble et très obeïssant
                                 serviteur \begin{flushright}\emph{signé}
                                 Marcol.\end{flushright}\end{quote}
                     \end{diary}

                     \begin{diary}{17 Août 1766}{}

                         La procession des
                           hommes s'est faite l'après midi,
                           peut être pour la dernière fois. Une pauvre
                           vieille femme y a été écrasé sous les roues de
                           carrosse. On a été obligé de différer la comédie,
                           et elle n'a commencé qu'après neuf heures. \bigskip


                     \end{diary}

                     \begin{diary}{18 Août 1766}{}

                         Je vais coucher à Neuviller, que je trouve
                           fort embelli. On fait les murs du parc, qui
                           sont bien avancés. L'étang est empoissonné,
                           les peupliers plantés en différents endroits s'élevent.
                           La digue ou jetée de pierres sur la Moselle a
                           sauvé les clos dans le dernier débordement.
                           Le village de Flavigny n'en a
                           point souffert non
                           plus et a été préservé par d'autres jetées,
                           sans lesquelles l’église et partie du village
                           étaient emportées. \bigskip


                     \end{diary}

                     \begin{diary}{19 Août 1766}{}

                         Le lendemain je vais dîner à
                              Ubexi. \bigskip


                     \end{diary}

                     \begin{diary}{20 Août 1766}{}


                           Le vingt j'en pars avec M.\up{lles}
                           de Tilly et
                           de Jevaincourt pour
                              la verrerie de
                           Porcieux.
                           Madame Durival qui était à cheval à côté
                           de la voiture, tombe sous son cheval et entres
                           les roues de la voiture, dans un grand danger
                           d'y périr : mais les chevaux ayant été arrêtés
                           tout court par mon cocher, aux cris des
                           demoiselles, cela s'est borné à des contusions, foulures de nerfs et à quelques petites
                           écorchures, dont on ne s'est aperçu qu'à
                           la verrerie. Nous en avons vu tout le
                           travail avec M. Plassiard l'un
                           des entrepreneurs,
                           et sommes convenu coucher à Ubexi, où madame
                              Durival a renoncé au cheval. \bigskip


                     \end{diary}

                     \begin{diary}{21 Août 1766}{}

                         Revenu dîner à Neuviller, et de là coucher
                           à Nancy, où je trouve des
                           nouvelles de
                           mes
                              frères par leurs lettres des 17 et 18. \bigskip



                           Le 19 on avait affiché à Nancy un arrêt
                           du Parlement de Paris du
                              11 juillet, touchant
                           les dépositaires et rétentionnaires des effet
                           des Jésuites de Lorraine. \bigskip


                     \end{diary}

                     \begin{diary}{22 Août 1766}{}

                         J'ai vu aujourd'hui le jeune Puiseur, conseiller de
                           l'hôtel de ville, il a quitté
                           depuis trois jours
                           la chartreuse de
                              Bosserville, où il a demeuré
                           un an dans le dessein d'y faire profession. \bigskip


                     \end{diary}

                     \begin{diary}{23 Août 1766}{}

                         On parle beaucoup d'un mémoire attribué
                           à la Chambre des Comptes dans lequel la Cour
                              souveraine
                           est fort maltraitées ; elle s'est assemblée plusieurs
                           fois à cette occasion. \bigskip


                     \end{diary}

                     \begin{diary}{24 Août 1766}{}

                         J'avais permis aux pantonimes de
                           jouer
                           demain fête du roi. Les officiers de la garnison
                           ont désirés qu'il y eut plutôt opéra bouffon,
                           et bal. J'avais refusé l'un et l'autre mais
                           j'ai parlé ce soir à M. de
                              Montesquiou
                           et il a été arrangé que les bouffons joueraient mais que les autres
                           seraient indemnisés sur
                           le produit du bal. La troupe de Fleury
                           revient après demain. \bigskip


                     \end{diary}

                     \begin{diary}{25 Août 1766}{}


                           Madame
                               de Beauvau, marquise
                              Des Armoises est morte à quatre heures
                           du matin, au château
                              Fléville, des suites
                           d'un cancer au sein, contre lequel elle prenait
                           des remèdes depuis trois mois. M.
                              Des Armoises
                           est revenu sur le champ à Nancy.
                              Madame
                                 Des Armoises était sœur du marquis de
                              Beauvau tué devant Ypres. \bigskip



                           M. Gallois, cy-devant
                           secrétaire d’État en Lorraine
                           partit
                           hier pour aller résider à Paris
                        \bigskip


                         Voici comment la fête du
                           roi a été célébrée
                           à Nancy. Hier on sonna en volée
                           à toutes les
                           églises à 6 h. du soir, aujourd'hui à 6 h.
                           du matin, à midi et
                           à 6 h. du soir. La comédie a commencé à 4 h.
                           À six le régiment du roi a pris les armes. À 7 h.
                           plusieurs décharges d'artillerie ; ensuite deux
                           salves de mousqueterie à feu roulant sur tout
                           le cordon des remparts de la vieille ville. Un peu
                           avant huit heures le feu d'artifice préparé
                           sur la place royale au pied et autour de la
                           statue de Louis XV à été exécuté, ce qui a
                           duré un peu plus de demie heure. Un
                           cercle de grenadiers était à 20 pas de
                           distance, et derrière eux, aux fenêtres
                           sur les balcons de l'hôtel de
                              ville un
                           peuple
                           immense. On a beaucoup crié \og vive le roi \fg{},
                           la musique du roi a joué
                           pendant tout
                           ce temps. À dix heures le bal a commencé et a duré toute la nuit. La comédie
                           n'était
                           point finie à la première décharge de
                           mousqueterie, quand elle s'est faite vis
                           à vis de la comédie, elle y a fait un bruit
                           comme si le théâtre s'enfonçait, tout le monde
                           s'est mis en disposition de fuir ; et on s'est
                           enfin rassuré. \bigskip


                     \end{diary}

                     \begin{diary}{26 Août 1766}{}

                         Il y a quelque temps qu'on a tenu conseil
                           de guerre, contre deux soldats du régiment du roi.
                           Les juges avaient ordonné un plus amplement
                           informé, contre l'avis de M. de
                              Montesquiou
                           du commandant du corps et des officiers
                           majors persuadés de la nécessité d'un exemple.
                           On a écrit à la Cour et il est arrivé une
                           lettre du ministre sur
                           laquelle on a tenu
                           aujourd'hui un nouveau conseil de guerre.
                           Les deux déserteurs ont eu aujourdhui
                           la tête cassée à 3 h. après midi. \bigskip



                           Madame Dèsarmoises a été inhumée
                           aujourd'hui dans la chapelle du
                              château
                              de Fléville. Il s'y est trouvé quelques
                           gentilshommes et les gens du village. \bigskip


                     \end{diary}

                     \begin{diary}{27 Août 1766}{}

                         Je reçois de Compiègne une lettre du 24
                           qui marque que tous les députés de la Lorraine
                           et du Barrois y sont
                           rassemblés, et qu'on y
                           a vu les S.\up{rs}
                           Coster et Doyen. \bigskip


                     \end{diary}

                     \begin{diary}{29 Août 1766}{}

                         J'ai payé au S. Mengin à la
                              Malgrange
                           90\up{\#.} 10\up{s.} pour 133 pieds de taille, 2 colonnes de de bois, leurs bases et chapiteaux. Et je lui avais
                           payé précédemment 12\up{\#.} 10\up{s.}
                           pour des marches. \bigskip


                         J'ai fait arrêter aujourd'hui et
                           mettre a la
                              conciergerie
                           le S. Chapuis, directeur de
                           l'opéra
                           bouffon, qui ne payait pas les bourgeois ; et
                           un de ses pensionnaires qui m'avait menacé. \bigskip


                         Au bâtiment de ville à la Vènerie un ouvrier a été dans
                           les fondations couvert de vingt tombereaux de
                           terre, presque à ma vue, par un bonheur
                           singulier il n'a point eu de mal, après avoir
                           été découvert. \bigskip


                     \end{diary}

                     \begin{diary}{30 Août 1766}{}


                           Montaigu et le bien de madame Prevost à
                           Jarville, on été adjugés aujourd'hui à M.
                              Launay fils à la somme de 62000\up{\#.} de
                              France. Il n'en avait d'abord offert que
                           50000\up{\#.} mais il a eu des concurrents. \bigskip


                         La troupe de Fleury, de retour de Dijon,
                           a recommencé à jouer aujourd'hui. \bigskip


                     \end{diary}

                     \begin{diary}{31 Août 1766}{}


                           M. l'intendant, qui partit hier
                           de Paris
                           à 6 h. du matin, est arrivé ici aujourd'hui
                           à 8 \up{1}/\textsubscript{2} du soir, avec
                              M. Bremont. Mon
                              jeune frère reste à Paris, et sera
                           auprès de M. de La Galaizière père. \bigskip



                           Cela n'a pas eu lieu.
                        \bigskip


                     \end{diary}
                  \chapter*{Septembre 1766}\addcontentsline{toc}{chapter}{Septembre 1766}



                     \begin{diary}{01 Septembre 1766}{}

                         Mort d’Anne-Françoise de Foucoult-Saint-Germain
                              Beaupré, marquise d’Auroy. Sera inhumée
                           dans l’église des Minimes de
                              Nancy. \bigskip


                     \end{diary}

                     \begin{diary}{03 Septembre 1766}{}


                           \emph{La Gazette de France} du 1.\up{er} septembre annonce
                           la mort du Bailly de
                              Froullay, ambassadeur de la
                           religion auprès du roi,
                           arrivé le 26 août, dans
                           sa 73\up{e} année de son âge. Et d’Anne-Catherine
                              Gerard de Viet, maréchale de Bercheny
                           le 24. \bigskip


                     \end{diary}

                     \begin{diary}{04 Septembre 1766}{}

                         On a fait tirer au sort aujourd'hui,
                           trois
                           déserteurs du régiment du roi, dont l’un a eu
                           la tête cassée sur la place de grève cet après
                           midi. \bigskip


                     \end{diary}

                     \begin{diary}{05 Septembre 1766}{}

                         On fait beaucoup de conjectures sur
                           deux
                           hommes arrêtés avec de grandes précautions
                           il y a cinq à six jours à l’auberge des
                              Dames de France. Un exempt de la police
                           les conduit à la Bastille ;
                           on prétend qu’il
                           y a cinq ans qu’on les suit, et qu’ils ont
                           beaucoup coûté au roi. Un de
                           ces
                           hommes se nomme La Coste,
                           l’autre Renaut. \bigskip


                     \end{diary}

                     \begin{diary}{11 Septembre 1766}{}


                           M. le comte de Stainville est
                           arrivé cet
                           après midi pendant la comédie, vers 7h. \bigskip


                     \end{diary}

                     \begin{diary}{13 Septembre 1766}{}


                           M. l’intendant arrive de
                              Lunéville, avec M.
                              Bremont.
                        \bigskip


                     \end{diary}

                     \begin{diary}{14 Septembre 1766}{}

                         Ce matin à 7 h. le curé d’Heillecourt en
                           surplus à la tête de ses paroissiens s’est
                           rendu à la Malgrange, d’où
                           la belle croix a été
                           transposée en cérémonie
                           au nouvel emplacement
                           qui lui avait été préparé près de l’église
                              de Bonsecours, et de la croix des bourguignons
                           entre quatre arbres où il y avait eu une croix
                           de mission. Quoique tout eut été préparé
                           en secret il s’y est trouvé 3 à 400 personnes. \bigskip


                         L’après midi les congrégations sont venues
                           à ce nouvel emplacement, comme on allait
                           autrefois à la Malgrange le
                           jour de l’invention
                           S.\up{te} Croix. Il y a eu des murmures et des
                           injures contre M. le comte de Stainville de la part
                           des fanatiques. Et le soir un de ces zélés
                           était venu me demander la permission de
                           faire la quête, pour faire un nouveau
                           dôme à cette croix, ce que j’ai refusé. \bigskip


                         J’ai adjugé définitivement aujourd'hui
                           le foin provenant des tilleuls des remparts. \bigskip


                     \end{diary}

                     \begin{diary}{15 Septembre 1766}{}

                         J’apprends par une réponse de M. de La Galaizière
                              père, datée de Compiègne
                           le 11, qu’il
                           a présenté à la
                              reine notre estampe
                           gravée par Collin, du mausolée du
                              roi
                              de Pologne. \bigskip


                         Il entre dans les projets de M. l’intendant
                           de faire à Nancy un hôpital
                           militaire
                           dont le plan a été examiné aujourd'hui
                           par lui et M. le comte de
                              Stainville ; un
                           manège des chevaux de la province ; et
                           un hôpital général. \bigskip


                     \end{diary}



                     \begin{diary}{16 Septembre 1766}{}

                         A 5 h \up{1}/\textsubscript{2} du matin, il n’y avait que deux
                              degrés de chaud, et il avait un peu gelé
                           à la campagne. \bigskip



                           M. l’intendant est parti ce matin à 8 h \up{1}/\textsubscript{2}
                           pour Neuviller. \bigskip


                     \end{diary}

                     \begin{diary}{21 Septembre 1766}{}

                         On arrête en l’hôtel de ville la déclaration à
                           faire au Parlement de
                           Paris, pour satisfaire
                           aux arrêts concernant les
                              Jésuites. \bigskip


                         On régle aussi qu’il y aura un des six
                           petits
                           bouchers au faubourg
                              Saint-Fiacre.
                        \bigskip


                     \end{diary}

                     \begin{diary}{22 Septembre 1766}{}

                         Les comédiens jouent avec succès
                              \emph{la Partie
                              de chasse de Henri IV}. Messieurs les comte de Stainville,
                           et marquis de Choiseul y
                           étaient avec madame
                              l’abbesse de S. Pierre. Le spectacle était plus
                           rempli qu’il n’est ordinairement dans cette saison. \bigskip



                           M. le comte de Stainville a
                           arrêté ce matin
                           comment serait faite la chambre de l’officier
                           de garde à la porte royale. \bigskip



                           M. Launay
                           cy-devant
                           commissaire des guerres, acquéreur
                           de Montaigu auprès de
                              Nancy , qui avait
                           été laissé à bail à M. le comte de
                              Stainville
                           et par lui relaissé au marquis de
                              Gerbéviller
                           est en pleine possession de cette jolie maison,
                           par la résiliation du bail de M. de
                              Gerbéviller. \bigskip


                     \end{diary}

                     \begin{diary}{23 Septembre 1766}{}


                           M. Alliot part de Nancy le matin pour aller
                           de nouveau à Paris. \bigskip



                           M. Mique passe venant de
                              Compiègne. \bigskip


                     \end{diary}


                     \begin{diary}{27 Septembre 1766}{}

                         Le pain augmenté d’un denier et demi
                           par livre, en sorte que blanc, à
                           commencer demain, est à 2\up{s.} 3\up{d.} le bis
                           à 1\up{s.} 7\up{d}
                           \up{1}/\textsubscript{2}. \bigskip


                         Les vendanges fixées au lundi 6 octobre
                           pour les privilégiés et au 7 pour
                           tous autres. \bigskip


                     \end{diary}

                     \begin{diary}{29 Septembre 1766}{}


                           M. de Rutant de Saulxures épouse
                              la fille
                              de M. Cheneau.
                        \bigskip



                           M. l’intendant arrive du voyage
                           qu’il a
                           fait en Alsace pour les haras.
                        \bigskip


                     \end{diary}

                     \begin{diary}{30 Septembre 1766}{}

                         Il choisit l’arsenal pour y placer les étalons,
                           le manège, \&\up{a}. Le S. Roissy
                           se presente
                           pour traiter de la fourniture des
                              casernes. \bigskip



                           Madame Durival arrive et voit M. l’intendant. \bigskip


                         Convenu que les fournitures de bois à
                              l’intendance
                           sera de 100 cordes par années ; et celles
                           des bureaux 20 cordes. \bigskip


                     \end{diary}
                  \chapter*{Octobre 1766}\addcontentsline{toc}{chapter}{Octobre 1766}




                     \begin{diary}{01 Octobre 1766}{}

                         M. l’intendant part pour Neuviller, d’où
                           il ira à Remiremont. \bigskip


                         Il pleut un peu après beaucoup de
                           jours
                           d’une extrême sécheresse qui empêchaient
                           les froments de lever. \bigskip


                     \end{diary}

                     \begin{diary}{02 Octobre 1766}{}

                         Assemblée particulière de l’Académie où
                           étaient Messieurs
                           Du Rouvrois, de Sivry, de
                              Solignac, Thibault, de
                           Tervenus, André,
                           Gautier, Coster, P. Husson et
                              Durival l’aîné.
                           On y a distribué à examiner six ouvrages
                           présentés pour les prix. M.
                              Gandoger
                           a été proposé pour académicien et admis
                           tout d’une voix. Il a encore été question
                           d’autres sujets proposés auparavant entre autres
                           de M. Le Bas chirurgien, fameux
                           par sa
                           dispute contre M. Louis, touchant
                           les
                           naissances tardives. On a remarqué
                           que M. Mittié n’avait pas
                           encore envoyé
                           de discours. Après la séance on a fait
                           l’essai de la pompe présentée par Despois. \bigskip


                         J’écris à M. le comte
                              de Stainville une
                           grande lettre pour l’engager à rouvrir
                           le chemin public qui passait par la
                              Malgrange, et qui est une communication
                           nécessaire entre Jarville
                           et Heillecourt,
                           Heillecourt et Nancy. \bigskip


                     \end{diary}


                     \begin{diary}{04 Octobre 1766}{}

                         Je reçois une lettre de mon
                              frère. Il
                           marque du 1\up{er}
                           que M. le duc de
                              Choiseul arriva de sa terre la veille,
                           ressentit dans la matinée des douleurs
                           de gravelle, dont il a
                           dejà eu plusieurs
                           attaques. Il a rendu 2 pierres. Et a
                           du se rendre à Choisy
                           le 1\up{er}
                              jusqu’au 4. \bigskip


                     \end{diary}

                     \begin{diary}{07 Octobre 1766}{}

                         Je vais à Heillecourt, où on commençait
                           la vendange. \bigskip


                     \end{diary}

                     \begin{diary}{08 Octobre 1766}{}

                         Aujourd'hui mercredi j’y reçois une
                           lettre
                           de mon jeune frère du
                              5 qui me marque qu’il
                           s’est logé rue Feydeau. M.
                              d’Ormesson se trouve
                           embarassé touchant l’ouvrage de mon
                              frère sur l’impôt, parce que mon frère
                           ne veut pas se contenter d’une permission
                           tacite pour l’impression, et qu’il inciste
                           sur une permission formelle qui marque
                           l’aveu du gouvernement. \bigskip


                         Je reçois en même temps pour les officiers
                           municipaux la lettre ci-après. \bigskip


                        \begin{quote}Messieurs,\bigskip


                              La confiance dont la Cour
                                 souveraine m’a honoré,
                              le service important que j’ai eu le bonheur de
                              lui rendre en lui procurant la communication
                              des mémoires clandestins du parlement de Metz
                              dont elle a par les siens si puissamment
                              repoussé les \sout{attaques}
                              tentatives ; enfin la
                              confiance générale dont par suite votre province m’honore aujourd'hui, par
                              le nombre
                              des affaires différentes qu’elle m’adresse, me fait
                              par reconnaissance naturaliser Lorrain ; recevez
                              donc Messieurs dans cette lettres la protestation
                              que je fais en vos mains à tous vos concitoyens,
                              non d’un zèle borné aux seuls devoir de l’avocat,
                              mais étendu à tout ce que chaque membre parait
                              demander, on pourrait attendre d’un Lorrain,
                              d’un Barrois domicilié à Paris qui cherchait
                              sa patrie. \bigskip


                              Permettez moi de joindre ici quelques mémoires
                              nouveaux de ma façon, dont les circonstances
                              singulières amuseront agréablement votre
                              esprit et intéresseront votre cœur. \bigskip

         Messieurs
                              C’est M. le President
                                 Vigneron
                              deputé de la Cour souveraine qui
                              a bien voulu se charger de
                              ce paquet pour vous le faire
                              remettre.
                           \bigskip


                                 Je suis avec respect
                                 votre très humble et très
                                 obéissant serviteur. \begin{flushright}
                                 D’Hermond de Clery, avocat
                                 au parlement et ès conseils du
                                 roi, \end{flushright}

                                    rue Jacob faubourg S.\up{t} Germain.

                                    à Paris

                                 le 1\up{er} octobre
                                    1766
                              \end{quote}

                           Cet après midi comme les vendanges rentraient
                           à Heillecourt et
                           criaient haillons, comme
                           on fait par plaisanterie au temps des vendanges,
                           sans que personne s’en fâche, ils ont rencontrés
                           quatre gardes de la ferme, dont l’un, qui
                           avait une capote et paraissait le brigadier,
                           a donné un coup d’épée au nommé
                              Michel,
                           habitant d’Heillecourt ;
                           les misérables ont
                           ensuite traversé les champs, et sont
                           entrés dans un bois. \bigskip


                     \end{diary}


                     \begin{diary}{11 Octobre 1766}{}

                         Aujourd'hui délibéré en l’hôtel de ville
                           d’emprunter 150000\up{\#.} cours de Lorraine à
                           8 % de rentes viagères, exempts de retenues,
                           pour contribuer de 30000\up{\#.} de France aux
                           casernes, et payer
                           divers entrepreneurs. \bigskip


                     \end{diary}

                     \begin{diary}{15 Octobre 1766}{}

                         Assemblée particulière de l’Académie
                           où étaient Messieurs
                           Du Rouvrois, de Solignac,
                           de Tervenus, André, Coster, P. Husson et
                           Durival l’aîné. On a éxaminé
                           le
                           discours de réception que doit
                           prononcer
                           M. Gandoger ; il est sur la
                           conservation
                           des grains. On a lu ensuite le discours
                              de M.
                                 François, et son héroïde de Charles I.
                           Elle a été trouvée supérieure au discours.
                           M. de Solignac a ensuite lu
                           presque
                           tout l’éloge qu’il destinait au P. de
                              Menoux.
                           mais quoique ce discours dit du révérend père tout
                           le bien qu’on en pouvait dire sans mentir,
                           et n’en dit point de mal, qu’il fut même
                           beaucoup mieux fait que celui du roi de Pologne.
                           Il a été désapprouvé par tous excepté
                           M. Du Rouvrois et moi. On a
                           voulu
                           soumettre l’auteur à des retranchements
                           à des corrections : il s’y refuse et il
                           est incertain que ce discours soit prononcé. \bigskip


                     \end{diary}

                     \begin{diary}{17 Octobre 1766}{}


                           Mon jeune frère me marque
                           du 12  \og
                              J’appris hier qu’on a établi M.
                                 Cochin, avocat parent
                              de M. le contrôleur
                                 général, dépositaire
                              des papiers de la
                              Lorraine, et qu’on attendra
                              M. Bremon pour lui en faire la
                              remise  \fg{}. \bigskip


                     \end{diary}

                     \begin{diary}{19 Octobre 1766}{}

                         Le thermomètre au degré zéro, à 7 h \up{1}/\textsubscript{2}
                              du matin. Aussi il a gelé. \bigskip


                     \end{diary}

                     \begin{diary}{20 Octobre 1766}{}


                           M. l’intendant est venu de
                              Neuviller pour
                           la séance publique de l’Académie.
                              M.
                              Peronnet de l’Académie des sciences s’étant trouvé ici
                           a pris séance parmi nous et a eu son
                           jeton. La séance a commencé par
                           l’éloge du P. de Menoux, de la
                           façon de M.
                              de Solignac. M.
                              François, jeune homme
                           de 15 ans né à Saffais a fait
                           ensuite
                           son remerciement et lu une héroïde
                              de
                              Charles 1 roi
                              d’Angleterre. M.
                              Gandoger
                           a prononcé le sien et un discours sur
                              la
                              conservation des grains. La séance a
                           été terminée par un discours du directeur
                           aux deux recipiendaires. Il y avait Messieurs
                           Du Rouvrois, de Solignac, de
                              La Galaizière,
                           Perronet, de Tervenus, Gautier, Bagard,
                           André, Harmant, \sout{Gautier}
                           Coster, Thibault,
                           Montignot

                           Gandoger, François. Le public a paru
                           content et l’assemblée était belle. \bigskip


                     \end{diary}

                     \begin{diary}{21 Octobre 1766}{}

                         Aujourd'hui Préville a joué Crispin dans
                           \emph{le Légataire}, et dans \emph{Crispin Rival
                              de son maitre}
                        \bigskip


                     \end{diary}

                     \begin{diary}{22 Octobre 1766}{}

                         Le lendemain mercredi le brutal dans
                           \emph{Les Menechmes}, et Sganarelle dans \emph{le Médecin
                              malgré lui}. \bigskip



                           M. l’intendant est reparti ce
                           matin pour
                           Neuviller. \bigskip


                     \end{diary}

                     \begin{diary}{23 Octobre 1766}{}

                         On a donné \emph{le
                              Philosophe sans le savoir},
                           où Préville a fait le rôle d’Antoine. Et
                           \emph{les Trois frères rivaux} où il a joué
                           Merlin. \bigskip


                     \end{diary}

                     \begin{diary}{24 Octobre 1766}{}

                         Dans \emph{les
                              Confidences réciproques}
                           du Bois, et dans
                           \emph{Crispin médecin}
                        \bigskip


                     \end{diary}

                     \begin{diary}{26 Octobre 1766}{}

                         La nuit dernière vers onze heures
                           et demis,
                           Monblot milicien de Nancy, a été tué
                           par une sentinelle à la
                           porte du bal de la
                           S.\up{t} Crépin. \bigskip


                     \end{diary}

                     \begin{diary}{27 Octobre 1766}{}


                           M. Drion se charge à ses risques du
                           recouvrement des arrérages
                           de la chancellerie
                           et des greffes du conseil, au
                           moyen d’une
                           remise de 10% Il y en avait pour 27248\up{\#.} 19\up{s.} 4\up{d}.
                           Il délivre ses reconnaissances à cette déduction. \bigskip


                     \end{diary}
                  \chapter*{Novembre 1766}\addcontentsline{toc}{chapter}{Novembre 1766}



                     \begin{diary}{01 Novembre 1766}{}

                         On commence à allumer les lanternes
                           pour
                           éclairer les rues et les places de Nancy.
                        \bigskip


                     \end{diary}

                     \begin{diary}{03 Novembre 1766}{}

                         Je reçois des lettres de mon jeune frère des
                           30 octobre et 1\up{er}
                              novembre. On lui a refusé net
                           la permission d’imprimer son excellent
                              ouvrage
                              sur l’impôt. Il n’est pas sûr qu’il reste auprès
                           de M. de La Galaizière père, ni qu’il passe
                           l’hiver à Paris. \bigskip


                     \end{diary}

                     \begin{diary}{05 Novembre 1766}{}

                         Assemblée particulière de l’Académie, où
                           étaient Messieurs
                           Du Rouvrois, de Solignac, Thibault,
                           de Tervenus, abbé Gautier, P.
                              Husson, Harmant,
                           Coster, Gandoger, P. Leslie, et
                              Durival l'aîné
                           on y a rendu compte des ouvrages présentés
                           au concours, et quelques uns ont été rejetés. \bigskip




                           M. Gautier a rendu compte d’une
                           méthode nouvelle
                           de former l’ennéagome ou figure de 9 côtés.
                           L’auteur la prétend géométrique, elle n’est a certains
                           égards que méchanique, mais fort approchante et
                           utile. Ensuite de la pompe de Despois, dans
                           laquelle on a trouvé de l’invention ; au moins
                           dans l’application de ce qui avait été trouvé
                           par d’autres. M. Le Bas, proposé
                           il y a
                           longtemps a été enfin admis. M. de
                              Mortal
                           chancelier de Toul renvoyé a de nouveaux ouvrages
                           s’il en fait, ceux qu’on connait de lui jusqu’à
                           présent n’étant pas suffisant. On a proposé
                           M.  professeur en l’École royale
                              militaire
                           et auteur d’une histoire romaine. \bigskip



                           M. Alliot est parti pour Paris passant à Pont-à-Mousson. \bigskip


                     \end{diary}


                     \begin{diary}{15 Novembre 1766}{}

                         Mort d’Elisabeth-Henriette le Texier, veuve
                           de Léopold de Bourcier de
                              They, lieutenant colonnel
                           des gardes du duc Léopold.
                           Sera inhumée
                           aux Dames du S.
                              Sacrement. Elle laisse
                           deux filles. \bigskip


                     \end{diary}

                     \begin{diary}{16 Novembre 1766}{}


                           Madame la marquise de Bouflers, qui quitte
                           Lunéville a passé
                           aujourdhui allant en
                           Languedoc joindre le prince de Beauvau son frère. \bigskip


                         J’assemble les imprimeurs et libraires
                           de
                           Nancy, à l’occasion des mauvais
                           livres
                           qui se répandent à Paris, et
                           qu’on prétend
                           venir d’eux. \bigskip


                     \end{diary}

                     \begin{diary}{18 Novembre 1766}{}


                         Assemblée au bureau de l’aumône, où étaient
                           Messieurs
                           Du Rouvrois, de Tervenus, de
                              Bressey,
                           François, de Dombâle, de Maisonneuve et moi.
                           On y a continué pour Boulanger la veuve du
                           dernier mort. Et accordé une réduction de
                           près de 400\up{\#.} à la caution de celui qui
                           tenait autrefois le jardin. \bigskip


                     \end{diary}

                     \begin{diary}{20 Novembre 1766}{}

                         Mort du S. Pierre François Huguet de
                              Montaran,
                           écuyer avocat en Parlement de
                              Paris. Sera inhumé
                           dans l’église de S. Roch. \bigskip



                           Messieurs Renault partent aujourd'hui de Paris
                           pour revenir en Lorraine.
                        \bigskip


                     \end{diary}

                     \begin{diary}{23 Novembre 1766}{}


                           M. Mique passe retournant à
                              Paris, par
                           ordre de la reine. \bigskip


                     \end{diary}



                     \begin{diary}{25 Novembre 1766}{}


                           M. Renault et M. d’Ubexi sont arrivés
                           de Paris. Les privilèges de
                              la manufacture
                              de bains sont confirmés \&\up{a}. \bigskip


                         Ces jours derniers la Cour souveraine
                           a fait le grands changements dans ses greffes.
                           M. Coster est
                           secrétaire-greffier en chef. \bigskip


                     \end{diary}

                     \begin{diary}{27 Novembre 1766}{}


                           M. l’intendant arrive de Neuviller pour
                           rester à Nancy jusqu’à son
                           voyage de Paris. \bigskip


                     \end{diary}

                     \begin{diary}{28 Novembre 1766}{}

                         On achève aujourd'hui de couvrir
                              le
                              quartier royal des casernes. \bigskip


                     \end{diary}

                     \begin{diary}{29 Novembre 1766}{}


                           M. Bremont part pour Paris, afin d’y
                           achever la remise des papiers avant
                           l’arrivée de M. l’intendant.
                           Il sera joint
                           dans la route par M. Girardet
                           peintre
                           du feu roi de Pologne que la
                              reine veut voir. \bigskip


                     \end{diary}

                     \begin{diary}{30 Novembre 1766}{}


                           M. l’intendant vient me voir.
                           J’avais
                           encore un peu de goutte. \bigskip


                     \end{diary}
                  \chapter*{Décembre 1766}\addcontentsline{toc}{chapter}{Décembre 1766}


                     \begin{diary}{01 Décembre 1766}{}

                         Je reçois des lettres de mes
                              frères du
                           27 novembre. Les dernières
                           tentatives
                           sur nos pensions ont été inutiles. Elles
                           resteront telle qu’elles ont été fixées d’abord.
                           Mon jeune frère doit
                           en faire de nouvelles
                           sur l’impression de son ouvrage ; et quand
                           tout cela sera arrangé il viendra demeurer
                           à Nancy. \bigskip


                         J’ai appris ces jours-ci que le président
                              Vigneron demandait des mémoires pour
                           que les appels de la polices de Nancy
                           soient
                           postés à la Cour souveraine \&\up{a}. \bigskip


                         On m’écrit de Metz que M. de
                              Bernage
                           en était parti avec sa maison le 24
                              novembre et qu’on n’y attend pas sitôt M.
                              de Calonne, qui le remplace comme intendant. \bigskip


                     \end{diary}

                     \begin{diary}{03 Décembre 1766}{}


                           M. Vigneron avocat
                           général, madame la présidente
                              Vigneron, et M.
                              Coster partent pour Paris. \bigskip


                     \end{diary}

                     \begin{diary}{04 Décembre 1766}{}


                           La dernière épouse du S. Busquet de Caumont,
                           conseiller au parlement de
                              Rouen, est décédée
                           hier à Nancy, chez
                           M. d’Authieulles
                           lieutenant de
                           roi. Elle sera inhumée dans l’église
                              de S. Epvre.
                        \bigskip


                     \end{diary}



                     \begin{diary}{08 Décembre 1766}{}


                           M. l’intendant arrive de Remiremont
                           où il était déja allé pour la réception d’une
                           dame, sur laquelle il y a opposition. \bigskip


                     \end{diary}

                     \begin{diary}{09 Décembre 1766}{}


                           M. de Farémont, gentilhomme du
                              Barrois,
                           épouse M.\up{lle}
                              Dureteste. \bigskip


                     \end{diary}

                     \begin{diary}{10 Décembre 1766}{}

                         Le froid a été très violent ces
                           jours-ci.
                           Il y en avait encore près de 5 degrés
                           ce matin à 9
                              heures. \bigskip


                     \end{diary}

                     \begin{diary}{11 Décembre 1766}{}

                         On était hier dans la
                           crise. L’eau des
                           rivières et des ruisseaux, diminuée encore
                           par les gelées, faisait chômer la plupart
                           des moulins. Les 9 tournants de ceux de
                           Nancy ne faisaient que 100 rézaux au lieu
                           de 200. Il n’y avait de farine que pour
                           environ 14 jours chez nos boulangers.
                           M. l’intendant avait
                           ordonné des moulins
                           à bras. Heureusement il a commencé à
                           dégeler aujourd'hui, par 4 degrés de chaud. \bigskip


                     \end{diary}

                     \begin{diary}{12 Décembre 1766}{}

                         Mort du S.
                              Noël, ancien secrétaire de
                           l’hôtel de ville de Nancy... \bigskip


                     \end{diary}

                     \begin{diary}{13 Décembre 1766}{}

                         Je reçois une lettre de M.
                              Bremont du 10. Il
                           était arrivé le 7 à Paris. \bigskip


                     \end{diary}


                     \begin{diary}{15 Décembre 1766}{}


                           M. l’intendant part à 10 h \up{1}/\textsubscript{2} du matin
                           pour Paris, passant par
                              Metz. \bigskip


                         Il y avait cinq jours que l’air
                           était fort chargé
                           et couvert : enfin vers la nuit la pluie a
                           commencé ; elle a été abondante ensuite
                           ce qui fait cesser de grandes alarmes, surtout
                           dans les campagnes où on ne trouvait plus
                           d’eau pour les bestiaux. \bigskip


                     \end{diary}

                     \begin{diary}{17 Décembre 1766}{}

                         Déliberation pour échanger le corps de garde
                              S.\up{te} Catherine, contre la maison de Perbal rue
                              des champs, pour une entrée au jardin de botanique
                        \bigskip


                         Arrêt de la Cour souveraine qui surseoit l’exécution des
                           lettres patentes surprises par
                           l’exécuteur. \bigskip


                         Assemblée particulière de l’Académie,
                           où je n’ai pu me trouver. On y a adjugé
                           le prix des belles lettres à l’éloge du roi de Pologne n\degre1.
                           Et celui des arts a été partagé par moitié
                           entre la pompe de Despois et une méthode
                              presque géometrique de former l’Ennéagone. \bigskip


                     \end{diary}

                     \begin{diary}{20 Décembre 1766}{}

                         Il a commencé à neiger vers dix heures
                           du matin. Il avait un peu gelé la nuit \bigskip


                         On me mande de Versailles
                           le 17, que ce jour là
                           tout le Parlement de Paris
                           y était. \bigskip


                     \end{diary}


                     \begin{diary}{21 Décembre 1766}{}

                         Je fais publier à Nancy l’ordonnance du 20
                              octobre 1766, concernant la milice. \bigskip


                     \end{diary}

                     \begin{diary}{22 Décembre 1766}{}

                         Assemblée chez
                           M. l’abbé de Grandchamps,
                           où j’étais et Messieurs les curés, pour disposer
                           des 6 places au refuge de la fondation
                              de l’abbé de Bousey, pour 1767.
                           Elles seront remplies par
                           Marianne Renel
                           Francoise Antoine
                           La Duclos pour 2 ans
                           Jeanne Edelin de Nancy
                           Anne Caprès, âgée de 18 ans
                           Marguerite Mengin,
                              f\up{e}. Séguin.
                           Marie Fendrick pour la
                              1\up{re}. vacante. \bigskip


                         Mort de M. Claude de Bavillier, chevalier de S. Louis,
                           ingénieur en chef à Nancy. Il
                           était dans un
                           âge très avancé, et sera inhumé le 24
                           dans
                           l’église paroisse de Notre-Dame
                        \bigskip


                     \end{diary}

                     \begin{diary}{24 Décembre 1766}{}

                         On a accordé une augmentation d’un
                           denier
                           et demi sur la livre de pain. ainsi à commencer
                           du 25 le blanc sera 2\up{s} 4\up{d}\up{1}/\textsubscript{2} le bis 1\up{s} 9. \bigskip


                     \end{diary}


                     \begin{diary}{30 Décembre 1766}{}

                         Mort de
                           Jacob, femme
                           du S. Alix de
                              Pixérécourt, cy-devant
                           grand maître
                           des eaux et forêts. Elle était jeune encore
                           et laisse quatre enfants. 28 jours de mal \bigskip


                     \end{diary}

                     \begin{diary}{31 Décembre 1766}{}

                         Un enfant a été brulé aujourd'hui
                           chez sa nourrice qui était allée au marché
                           pendant que son mari était au bois. C’est
                           dans une maison du faubourg
                              Saint-Fiacre,
                           et par le feu d’un fourneau près duquel
                           on avait laissé cet enfant, et un autre
                           qui a pensé avoir le même sort. \bigskip


                         Il y avait aujourd'hui cinq degrés de
                              froid à huit heures du matin. À midi
                           0. \bigskip


                         Suivant l’état commencé à la fin de
                              novembre et fini dans le présent mois de
                              décembre il s’est trouvé à Nancy 26989 âmes
                           savoir : \begin{itemize}\item hommes 5517\item femmes 6428\item Enfans \begin{itemize}\item mâles 4474\item femelles 5284\end{itemize}

                              \item Domestiques
                                 \begin{itemize}\item mâles 1142\item femelles 1897\end{itemize}

                              \item Penssionnaires \begin{itemize}\item mâles 1157\item femelles 1090\end{itemize}
                              \end{itemize}
                        \bigskip


                     \end{diary}
